%\documentclass[defaultstyle,11pt]{thesis}
%\documentclass[]{report}
%\documentclass[]{article}
%\usepackage{aastex_hack}
%\usepackage{deluxetable}
%\documentclass[preprint]{aastex}
%\documentclass{aa}
\newcommand\arcdeg{\mbox{$^\circ$}\xspace} 

\pdfminorversion=4


%%%%%%%%%%%%%%%%%%%%%%%%%%%%%%%%%%%%%%%%%%%%%%%%%%%%%%%%%%%%%%%%
%%%%%%%%%%%  see documentation for information about  %%%%%%%%%%
%%%%%%%%%%%  the options (11pt, defaultstyle, etc.)   %%%%%%%%%%
%%%%%%%  http://www.colorado.edu/its/docs/latex/thesis/  %%%%%%%
%%%%%%%%%%%%%%%%%%%%%%%%%%%%%%%%%%%%%%%%%%%%%%%%%%%%%%%%%%%%%%%%
%		\documentclass[typewriterstyle]{thesis}
% 		\documentclass[modernstyle]{thesis}
% 		\documentclass[modernstyle,11pt]{thesis}
%	 	\documentclass[modernstyle,12pt]{thesis}

%%%%%%%%%%%%%%%%%%%%%%%%%%%%%%%%%%%%%%%%%%%%%%%%%%%%%%%%%%%%%%%%
%%%%%%%%%%%    load any packages which are needed    %%%%%%%%%%%
%%%%%%%%%%%%%%%%%%%%%%%%%%%%%%%%%%%%%%%%%%%%%%%%%%%%%%%%%%%%%%%%
\usepackage{latexsym}		% to get LASY symbols
\usepackage{graphicx}		% to insert PostScript figures
%\usepackage{deluxetable}
\usepackage{rotating}		% for sideways tables/figures
\usepackage{natbib}  % Requires natbib.sty, available from http://ads.harvard.edu/pubs/bibtex/astronat/
\usepackage{savesym}
\usepackage{pdflscape}
\usepackage{amssymb}
\usepackage{morefloats}
%\savesymbol{singlespace}
\savesymbol{doublespace}
%\usepackage{wrapfig}
%\usepackage{setspace}
\usepackage{xspace}
\usepackage{color}
\usepackage{multicol}
\usepackage{mdframed}
\usepackage{url}
\usepackage{subfigure}
%\usepackage{emulateapj}
\usepackage{lscape}
\usepackage{grffile}
\usepackage{standalone}
\standalonetrue
\usepackage{import}
\usepackage[utf8]{inputenc}
\usepackage{longtable}
\usepackage{booktabs}
\usepackage[yyyymmdd,hhmmss]{datetime}
\usepackage{fancyhdr}







%\renewcommand\ion[2]{#1$\;${%
%\ifx\@currsize\normalsize\small \else
%\ifx\@currsize\small\footnotesize \else
%\ifx\@currsize\footnotesize\scriptsize \else
%\ifx\@currsize\scriptsize\tiny \else
%\ifx\@currsize\large\normalsize \else
%\ifx\@currsize\Large\large
%\fi\fi\fi\fi\fi\fi
%\rmfamily\@Roman{#2}}\relax}% 

\newcommand{\paa}{Pa\ensuremath{\alpha}}
\newcommand{\brg}{Br\ensuremath{\gamma}}
\newcommand{\msun}{\ensuremath{M_{\odot}}\xspace}			%  Msun
\newcommand{\mdot}{\ensuremath{\dot{M}}\xspace}
\newcommand{\lsun}{\ensuremath{L_{\odot}}}			%  Lsun
\newcommand{\lbol}{\ensuremath{L_{\mathrm{bol}}}}	%  Lbol
\newcommand{\ks}{K\ensuremath{_{\mathrm{s}}}}		%  Ks
\newcommand{\hh}{\ensuremath{\textrm{H}_{2}}\xspace}			%  H2
\newcommand{\dens}{\ensuremath{n(\hh) [\percc]}\xspace}
\newcommand{\formaldehyde}{\ensuremath{\textrm{H}_2\textrm{CO}}\xspace}
\newcommand{\formaldehydeIso}{\ensuremath{\textrm{H}_2~^{13}\textrm{CO}}\xspace}
\newcommand{\methanol}{\ensuremath{\textrm{CH}_3\textrm{OH}}\xspace}
\newcommand{\ortho}{\ensuremath{\textrm{o-H}_2\textrm{CO}}\xspace}
\newcommand{\para}{\ensuremath{\textrm{p-H}_2\textrm{CO}}\xspace}
\newcommand{\oneone}{\ensuremath{1_{1,0}-1_{1,1}}\xspace}
\newcommand{\twotwo}{\ensuremath{2_{1,1}-2_{1,2}}\xspace}
\newcommand{\threethree}{\ensuremath{3_{1,2}-3_{1,3}}\xspace}
\newcommand{\threeohthree}{\ensuremath{3_{0,3}-2_{0,2}}\xspace}
\newcommand{\threetwotwo}{\ensuremath{3_{2,2}-2_{2,1}}\xspace}
\newcommand{\threetwoone}{\ensuremath{3_{2,1}-2_{2,0}}\xspace}
\newcommand{\fourtwotwo}{\ensuremath{4_{2,2}-3_{1,2}}\xspace} % CH3OH 218.4 GHz
\newcommand{\methylcyanide}{\ensuremath{\textrm{CH}_{3}\textrm{CN}}\xspace}
\newcommand{\Rone}{\ensuremath{\para~S_{\nu}(\threetwoone) / S_{\nu}(\threeohthree)}\xspace}
\newcommand{\Rtwo}{\ensuremath{\para~S_{\nu}(\threetwotwo) / S_{\nu}(\threetwoone)}\xspace}
\newcommand{\JKaKc}{\ensuremath{J_{K_a K_c}}}
\newcommand{\water}{H$_{2}$O}		%  H2O
\newcommand{\feii}{\ion{Fe}{2}}		%  FeII
\newcommand{\uchii}{UC\ion{H}{ii}\xspace}
\newcommand{\UCHII}{UC\ion{H}{ii}\xspace}
\newcommand{\hchii}{HC\ion{H}{ii}\xspace}
\newcommand{\HCHII}{HC\ion{H}{ii}\xspace}
\newcommand{\hii}{H~{\sc ii}\xspace}
\newcommand{\hi}{H~{\sc i}\xspace}
\newcommand{\Hii}{H~{\sc ii}\xspace}
\newcommand{\HII}{H~{\sc ii}\xspace}
\newcommand{\Xform}{\ensuremath{X_{\formaldehyde}}}
\newcommand{\kms}{\textrm{km~s}\ensuremath{^{-1}}\xspace}	%  km s-1
\newcommand{\nsample}{456\xspace}
\newcommand{\CFR}{5\xspace} % nMPC / 0.25 / 2 (6 for W51 once, 8 for W51 twice) REFEDIT: With f_observed=0.3, becomes 3/2./0.3 = 5
\newcommand{\permyr}{\ensuremath{\mathrm{Myr}^{-1}}\xspace}
\newcommand{\pers}{\ensuremath{\mathrm{s}^{-1}}\xspace}
\newcommand{\tsuplim}{0.5\xspace} % upper limit on starless timescale
\newcommand{\ncandidates}{18\xspace}
\newcommand{\mindist}{8.7\xspace}
\newcommand{\rcluster}{2.5\xspace}
\newcommand{\ncomplete}{13\xspace}
\newcommand{\middistcut}{13.0\xspace}
\newcommand{\nMPC}{3\xspace} % only count W51 once.  W51, W49, G010
\newcommand{\obsfrac}{30}
\newcommand{\nMPCtot}{10\xspace} % = nmpc / obsfrac
\newcommand{\nMPCtoterr}{6\xspace} % = sqrt(nmpc) / obsfrac
\newcommand{\plaw}{2.1\xspace}
\newcommand{\plawerr}{0.3\xspace}
\newcommand{\mmin}{\ensuremath{10^4~\msun}\xspace}
%\newcommand{\perkmspc}{\textrm{per~km~s}\ensuremath{^{-1}}\textrm{pc}\ensuremath{^{-1}}\xspace}	%  km s-1 pc-1
\newcommand{\kmspc}{\textrm{km~s}\ensuremath{^{-1}}\textrm{pc}\ensuremath{^{-1}}\xspace}	%  km s-1 pc-1
\newcommand{\sqcm}{cm$^{2}$\xspace}		%  cm^2
\newcommand{\percc}{\ensuremath{\textrm{cm}^{-3}}\xspace}
\newcommand{\perpc}{\ensuremath{\textrm{pc}^{-1}}\xspace}
\newcommand{\persc}{\ensuremath{\textrm{cm}^{-2}}\xspace}
\newcommand{\persr}{\ensuremath{\textrm{sr}^{-1}}\xspace}
\newcommand{\peryr}{\ensuremath{\textrm{yr}^{-1}}\xspace}
\newcommand{\perkmspc}{\textrm{km~s}\ensuremath{^{-1}}\textrm{pc}\ensuremath{^{-1}}\xspace}	%  km s-1 pc-1
\newcommand{\perkms}{\textrm{per~km~s}\ensuremath{^{-1}}\xspace}	%  km s-1 
\newcommand{\um}{\ensuremath{\mu \textrm{m}}\xspace}    % micron
\newcommand{\mum}{\um}
\newcommand{\htwo}{\ensuremath{\textrm{H}_2}}    % micron
\newcommand{\Htwo}{\ensuremath{\textrm{H}_2}}    % micron
\newcommand{\HtwoO}{\ensuremath{\textrm{H}_2\textrm{O}}}    % micron
\newcommand{\htwoo}{\ensuremath{\textrm{H}_2\textrm{O}}}    % micron
\newcommand{\ha}{\ensuremath{\textrm{H}\alpha}}
\newcommand{\hb}{\ensuremath{\textrm{H}\beta}}
\newcommand{\so}{SO~\ensuremath{5_6-4_5}\xspace}
\newcommand{\SO}{SO~\ensuremath{1_2-1_1}\xspace}
\newcommand{\ammonia}{NH\ensuremath{_3}\xspace}
\newcommand{\twelveco}{\ensuremath{^{12}\textrm{CO}}\xspace}
\newcommand{\thirteenco}{\ensuremath{^{13}\textrm{CO}}\xspace}
\newcommand{\ceighteeno}{\ensuremath{\textrm{C}^{18}\textrm{O}}\xspace}
\def\ee#1{\ensuremath{\times10^{#1}}}
\newcommand{\degrees}{\ensuremath{^{\circ}}}
% can't have \degree because I'm getting a degree...
\newcommand{\lowirac}{800}
\newcommand{\highirac}{8000}
\newcommand{\lowmips}{600}
\newcommand{\highmips}{5000}
\newcommand{\perbeam}{\ensuremath{\textrm{beam}^{-1}}}
\newcommand{\ds}{\ensuremath{\textrm{d}s}}
\newcommand{\dnu}{\ensuremath{\textrm{d}\nu}}
\newcommand{\dv}{\ensuremath{\textrm{d}v}}
\def\secref#1{Section \ref{#1}}
\def\eqref#1{Equation \ref{#1}}
\def\facility#1{#1}
%\newcommand{\arcmin}{'}

\newcommand{\necluster}{Sh~2-233IR~NE}
\newcommand{\swcluster}{Sh~2-233IR~SW}
\newcommand{\region}{IRAS 05358}

\newcommand{\nwfive}{40}
\newcommand{\nouter}{15}

\newcommand{\vone}{{\rm v}1.0\xspace}
\newcommand{\vtwo}{{\rm v}2.0\xspace}
\newcommand\mjysr{\ensuremath{{\rm MJy~sr}^{-1}}}
\newcommand\jybm{\ensuremath{{\rm Jy~bm}^{-1}}}
\newcommand\nbolocat{8552\xspace}
\newcommand\nbolocatnew{548\xspace}
\newcommand\nbolocatnonew{8004\xspace} % = nbolocat-nbolocatnew
\renewcommand\arcdeg{\mbox{$^\circ$}\xspace} 
\renewcommand\arcmin{\mbox{$^\prime$}\xspace} 
\renewcommand\arcsec{\mbox{$^{\prime\prime}$}\xspace} 

\newcommand{\todo}[1]{\textcolor{red}{#1}}
\newcommand{\okinfinal}[1]{{#1}}
%% only needed if not aastex
%\newcommand{\keywords}[1]{}
%\newcommand{\email}[1]{}
%\newcommand{\affil}[1]{}


%aastex hack
%\newcommand\arcdeg{\mbox{$^\circ$}}%
%\newcommand\arcmin{\mbox{$^\prime$}\xspace}%
%\newcommand\arcsec{\mbox{$^{\prime\prime}$}\xspace}%

%\newcommand\epsscale[1]{\gdef\eps@scaling{#1}}
%
%\newcommand\plotone[1]{%
% \typeout{Plotone included the file #1}
% \centering
% \leavevmode
% \includegraphics[width={\eps@scaling\columnwidth}]{#1}%
%}%
%\newcommand\plottwo[2]{{%
% \typeout{Plottwo included the files #1 #2}
% \centering
% \leavevmode
% \columnwidth=.45\columnwidth
% \includegraphics[width={\eps@scaling\columnwidth}]{#1}%
% \hfil
% \includegraphics[width={\eps@scaling\columnwidth}]{#2}%
%}}%


%\newcommand\farcm{\mbox{$.\mkern-4mu^\prime$}}%
%\let\farcm\farcm
%\newcommand\farcs{\mbox{$.\!\!^{\prime\prime}$}}%
%\let\farcs\farcs
%\newcommand\fp{\mbox{$.\!\!^{\scriptscriptstyle\mathrm p}$}}%
%\newcommand\micron{\mbox{$\mu$m}}%
%\def\farcm{%
% \mbox{.\kern -0.7ex\raisebox{.9ex}{\scriptsize$\prime$}}%
%}%
%\def\farcs{%
% \mbox{%
%  \kern  0.13ex.%
%  \kern -0.95ex\raisebox{.9ex}{\scriptsize$\prime\prime$}%
%  \kern -0.1ex%
% }%
%}%

\def\Figure#1#2#3#4#5{
\begin{figure*}[htp]
\includegraphics[scale=#4,angle=#5]{#1}
\caption{#2}
\label{#3}
\end{figure*}
}


\def\RotFigure#1#2#3#4#5{
\begin{sidewaysfigure*}[htp]
\includegraphics[scale=#4,width=#5]{#1}
\caption{#2}
\label{#3}
\end{sidewaysfigure*}
}

\def\FigureSVG#1#2#3#4{
\begin{figure*}[htp]
    \def\svgwidth{#4}
    \input{#1}
    \caption{#2}
    \label{#3}
\end{figure*}
}

% originally intended to be included in a two-column paper
% this is in includegraphics: ,width=3in
% but, not for thesis
\def\OneColFigure#1#2#3#4#5{
\begin{figure}[htpb]
\epsscale{#4}
\includegraphics[scale=#4,angle=#5]{#1}
\caption{#2}
\label{#3}
\end{figure}
}

\def\SubFigure#1#2#3#4#5{
\begin{figure*}[htp]
\addtocounter{figure}{-1}
\epsscale{#4}
\includegraphics[angle=#5]{#1}
\caption{#2}
\label{#3}
\end{figure*}
}

%\def\FigureTwo#1#2#3#4#5{
%\begin{figure*}[htp]
%\epsscale{#5}
%\plottwo{#1}{#2}
%\caption{#3}
%\label{#4}
%\end{figure*}
%}

\def\FigureTwo#1#2#3#4#5#6{
\begin{figure*}[htp]
\subfigure[]{ \includegraphics[scale=#5,width=#6]{#1} }
\subfigure[]{ \includegraphics[scale=#5,width=#6]{#2} }
\caption{#3}
\label{#4}
\end{figure*}
}

\def\FigureTwoAA#1#2#3#4#5#6{
\begin{figure*}[htp]
\subfigure[]{ \includegraphics[scale=#5,width=#6]{#1} }
\\
\subfigure[]{ \includegraphics[scale=#5,width=#6]{#2} }
\caption{#3}
\label{#4}
\end{figure*}
}

\newenvironment{rotatepage}%
{}{}
   %{\pagebreak[4]\afterpage\global\pdfpageattr\expandafter{\the\pdfpageattr/Rotate 90}}%
   %{\pagebreak[4]\afterpage\global\pdfpageattr\expandafter{\the\pdfpageattr/Rotate 0}}%


\def\RotFigureTwoAA#1#2#3#4#5#6{
\begin{rotatepage}
\begin{sidewaysfigure*}[htp]
\subfigure[]{ \includegraphics[scale=#5,width=#6]{#1} }
\\
\subfigure[]{ \includegraphics[scale=#5,width=#6]{#2} }
\caption{#3}
\label{#4}
\end{sidewaysfigure*}
\end{rotatepage}
}

\def\RotFigureThreeAA#1#2#3#4#5#6#7{
\begin{rotatepage}
\begin{sidewaysfigure*}[htp]
\subfigure[]{ \includegraphics[scale=#6,width=#7]{#1} }
\\
\subfigure[]{ \includegraphics[scale=#6,width=#7]{#2} }
\\
\subfigure[]{ \includegraphics[scale=#6,width=#7]{#3} }
\caption{#4}
\label{#5}
\end{sidewaysfigure*}
\end{rotatepage}
\clearpage
}

\def\FigureThreeAA#1#2#3#4#5#6#7{
\begin{figure*}[htp]
\subfigure[]{ \includegraphics[scale=#6,width=#7]{#1} }
\\
\subfigure[]{ \includegraphics[scale=#6,width=#7]{#2} }
\\
\subfigure[]{ \includegraphics[scale=#6,width=#7]{#3} }
\caption{#4}
\label{#5}
\end{figure*}
}


\def\TallFigureTwo#1#2#3#4#5#6{
    \FigureTwo{#1}{#2}{#3}{#4}{#5}
    }

\def\SubFigureTwo#1#2#3#4#5{
\begin{figure*}[htp]
\addtocounter{figure}{-1}
\epsscale{#5}
\plottwo{#1}{#2}
\caption{#3}
\label{#4}
\end{figure*}
}

\def\FigureFour#1#2#3#4#5#6{
\begin{figure*}[htp]
\subfigure[]{ \includegraphics[width=3in,type=png,ext=.png,read=.png]{#1} }
\subfigure[]{ \includegraphics[width=3in,type=png,ext=.png,read=.png]{#2} }
\subfigure[]{ \includegraphics[width=3in,type=png,ext=.png,read=.png]{#3} }
\subfigure[]{ \includegraphics[width=3in,type=png,ext=.png,read=.png]{#4} }
\caption{#5}
\label{#6}
\end{figure*}
}

\def\FigureFourPDF#1#2#3#4#5#6{
\begin{figure*}[htp]
\subfigure[]{ \includegraphics[width=3in,type=pdf,ext=.pdf,read=.pdf]{#1} }
\subfigure[]{ \includegraphics[width=3in,type=pdf,ext=.pdf,read=.pdf]{#2} }
\subfigure[]{ \includegraphics[width=3in,type=pdf,ext=.pdf,read=.pdf]{#3} }
\subfigure[]{ \includegraphics[width=3in,type=pdf,ext=.pdf,read=.pdf]{#4} }
\caption{#5}
\label{#6}
\end{figure*}
}

\def\FigureThreePDF#1#2#3#4#5{
\begin{figure*}[htp]
\subfigure[]{ \includegraphics[width=3in,type=pdf,ext=.pdf,read=.pdf]{#1} }
\subfigure[]{ \includegraphics[width=3in,type=pdf,ext=.pdf,read=.pdf]{#2} }
\subfigure[]{ \includegraphics[width=3in,type=pdf,ext=.pdf,read=.pdf]{#3} }
\caption{#4}
\label{#5}
\end{figure*}
}

\def\Table#1#2#3#4#5{
%\renewcommand{\thefootnote}{\alph{footnote}}
\begin{table}
\caption{#2}
\label{#3}
    \begin{tabular}{#1}
        \hline\hline
        #4
        \hline
        #5
        \hline
    \end{tabular}
\end{table}
%\renewcommand{\thefootnote}{\arabic{footnote}}
}


%\def\Table#1#2#3#4#5#6{
%%\renewcommand{\thefootnote}{\alph{footnote}}
%\begin{deluxetable}{#1}
%\tablewidth{0pt}
%\tabletypesize{\footnotesize}
%\tablecaption{#2}
%\tablehead{#3}
%\startdata
%\label{#4}
%#5
%\enddata
%\bigskip
%#6
%\end{deluxetable}
%%\renewcommand{\thefootnote}{\arabic{footnote}}
%}

%\def\tablenotetext#1#2{
%\footnotetext[#1]{#2}
%}

\def\LongTable#1#2#3#4#5#6#7#8{
% required to get tablenotemark to work: http://www2.astro.psu.edu/users/stark/research/psuthesis/longtable.html
\renewcommand{\thefootnote}{\alph{footnote}}
\begin{longtable}{#1}
\caption[#2]{#2}
\label{#4} \\

 \\
\hline 
#3 \\
\hline
\endfirsthead

\hline
#3 \\
\hline
\endhead

\hline
\multicolumn{#8}{r}{{Continued on next page}} \\
\hline
\endfoot

\hline 
\endlastfoot
#7 \\

#5
\hline
#6 \\

\end{longtable}
\renewcommand{\thefootnote}{\arabic{footnote}}
}

\def\TallFigureTwo#1#2#3#4#5#6{
\begin{figure*}[htp]
\epsscale{#5}
\subfigure[]{ \includegraphics[width=#6]{#1} }
\subfigure[]{ \includegraphics[width=#6]{#2} }
\caption{#3}
\label{#4}
\end{figure*}
}

		% file containing author's macro definitions


\begin{document}

\title{The (proto) O-star population of W51}
\titlerunning{W51 EVLA H2CO}
\authorrunning{Ginsburg et al}
% for future reference, this is probably a better approach:
% https://github.com/dfm/peerless/blob/af483ced97045c213650ed807c430b2f87d2c8c9/document/ms.tex#L104
% assuming it's compatible with A&A
\newcommand{\eso}{$^{1}$}
\newcommand{\nrao}{$^{2}$}
\newcommand{\radboud}{$^{3}$}
\newcommand{\allegro}{$^{4}$}
\newcommand{\morelia}{$^{5}$}
\newcommand{\excellence}{$^{6}$}
\newcommand{\casa}{$^{7}$}
\newcommand{\cfa}{$^{8}$}
\newcommand{\lasp}{$^{9}$}
\newcommand{\jodrell}{$^{11}$}
\newcommand{\sofia}{$^{12}$}
\newcommand{\mpia}{$^{13}$}
\newcommand{\iah}{$^{14}$}


\author{
Adam Ginsburg{\eso},
W.M. Goss{\nrao}, Ciriaco Goddi{\radboud$^{,}$\allegro}, Roberto Galvan-Madrid{\morelia}, Jim Dale\excellence, John Bally{\casa}, Cara
Battersby{\cfa}, Allison Youngblood{\lasp}, Ravi Sankrit{\sofia}, Rowan Smith{\jodrell}, Jeremy Darling\casa, J.~M.~Diederik Kruijssen{\mpia$^{,}$\iah},
Hauyu Baobab Liu{\eso}
        }
%
% Randolf Klein, Jin Koda,  nick scoville,  Allison
% Youngblood, Eric Becklin,  Adam Ginsburg,

%\institute{
%      {$^\casa$}{\it{CASA, University of Colorado, 389-UCB, Boulder, CO 80309}}}
%      {$^\eso$}{\it{European Southern Observatory, Karl-Schwarzschild-Strasse 2, D-85748 Garching bei München, Germany}}}
%      {$^\cfa$}{\it{CfA}}}
%      {$^\mpifr$}{\it{Max Planck Institute for Radio Astronomy, auf dem Hugel, Bonn}}}
%      {$^\nrao$}{\it{National Radio Astronomy Observatory, Socorro}}}
%      {$^\oxford$}{\it{Oxford}}}
%      {$^\chalmers$}{\it{Chalmers}}}
%}
\institute{
    {\eso}{
           \it{
               European Southern Observatory, Karl-Schwarzschild-Stra{\ss}e 2, D-85748 Garching bei München, Germany\\
                      \email{Adam.Ginsburg@eso.org}
               }
           } \\ 
    %{\saudi}{\it{Astron. Dept., King Abdulaziz University, P.O. Box 80203,
    %Jeddah 21589, Saudi Arabia}}\\
    %%{\edmonton}{\it{University of Alberta, Department of Physics, 4-181 CCIS, Edmonton AB T6G 2E1 Canada}} \\ 
    %{\yale}{\it{Department of Astronomy, Yale University, P.O. Box 208101, New Haven, CT 06520-8101 USA}} \\ 
    %%{\puertorico}{\it{Department of Physical Sciences, University of Puerto Rico, P.O. Box 23323, San Juan, PR 00931}}
    %{\mpifr}{\it{Max Planck Institute for Radio Astronomy, auf dem Hugel, Bonn}}
    {\nrao}{\it{National Radio Astronomy Observatory, Socorro, NM 87801 USA}}\\
    %{\oxford}{\it{Oxford}}
    %{\chalmers}{\it{Chalmers}}
    {\radboud}{\it{Department of Astrophysics/IMAPP, Radboud University Nijmegen, PO Box 9010, 6500 GL Nijmegen, the Netherlands}} \\
    {\allegro}{\it{ALLEGRO/Leiden Observatory, Leiden University, PO Box 9513, NL-2300 RA Leiden, the Netherlands}} \\
    {\morelia}{\it{Instituto de Radioastronom{\'i}a y Astrof{\'i}sica, UNAM, A.P. 3-72, Xangari, Morelia, 58089, Mexico}} \\
    {\excellence}{\it{University Observatory/Excellence Cluster `Universe' Scheinerstra{\ss}e 1, 81679 M{\"u}nchen, Germany}} \\
    {\casa}{\it{CASA, University of Colorado, 389-UCB, Boulder, CO 80309}} \\ 
    {\cfa}{\it{Harvard-Smithsonian Center for Astrophysics, 60 Garden
               Street, Cambridge, MA 02138, USA}} \\ 
    {\lasp}{\it{LASP, University of Colorado, 600 UCB, Boulder, CO 80309}}\\
    {\sofia}{\it{SOFIA Science Center, NASA Ames Research Center, M/S 232-12, Moffett Field, CA 94035, USA}}\\
    {\jodrell}{\it{Jodrell Bank Centre for Astrophysics, School of Physics and Astronomy, University of Manchester, Oxford Road, Manchester M13 9PL, UK}} \\
    {\mpia}{\it{Max-Planck Institut f{\"u}r Astrophysik, Karl-Schwarzschild-Stra{\ss}e 1, 85748 Garching, Germany}} \\
    {\iah}{\it{Astronomisches Rechen-Institut, Zentrum f{\"u}r Astronomie der Universit{\"a}t Heidelberg, M{\"o}nchhofstra{\ss}e 12-14, 69120 Heidelberg, Germany}}
    }


% Christian Henkel <chenkel@mpifr-bonn.mpg.de>,
% Jens Kauffmann <jens.kauffmann@gmail.com>
% Thushara Pillai <tpillai.astro@gmail.com>
% Karl M. Menten <kmenten@mpifr-bonn.mpg.de>,
% Katharina Immer <kimmer@mpifr-bonn.mpg.de>,
% John Bally <john.bally@colorado.edu>,
% Betsy Mills <millbets@gmail.com>,
% Jeremy Darling <jdarling@origins.colorado.edu>,
% Denise Riquelme <riquelme@mpifr-bonn.mpg.de>,
% Miguel Angel Requena Torres <mrequena@mpifr-bonn.mpg.de>,
% Cara Battersby <cbattersby@cfa.harvard.edu>,
% Leonardo Testi <ltesti@eso.org>,
% Juergen Ott <jott@nrao.edu>,
% Yiping Ao <ypaobb@gmail.com>,
% Susanne Aalto <susanne.aalto@chalmers.se>,
% Thomas Stanke <tstanke@eso.org>,
% Sarah Kendrew <sarahaskendrew@gmail.com>
% Rolf Guesten <rguesten@mpifr-bonn.mpg.de>
% Arnaud Belloche <belloche@mpifr-bonn.mpg.de>


\date{Date: \today ~~ Time: \currenttime}

\abstract
{}
{}
{}
{}
{
(1) \ortho \twotwo emission traces forming very massive (proto-O) stars
(2) There is a spatially distributed population of $\sim$mJy continuum sources,
including probable hypercompact HII regions and O-stars with ionized winds,
around the W51 proto-clusters
(3) There are two clearly detected clusters, W51e and W51 IRS 2, but the
majority of the luminosity in W51 most likely comes from a third population of
OB stars between these clusters.
}

\maketitle

\todo{To-do items are coded in red.}

\section{Introduction}
\todo{Expand introduction - that is always last...}
\footnote{
This paper and all related analysis code are available on the web at
https://github.com/adamginsburg/paper\_w51\_evla.
}


The protoclusters within W51 contain many forming massive stars
\citep{Zhang1997a,Keto2008b,Zapata2008a,Zapata2009a,Zapata2010a,Shi2010b,Shi2010a,Goddi2015a}
and a few that have already reached the main sequence and are visible in the
infrared \citep{Goldader1994a,Okumura2000a,Kumar2004a,Barbosa2008a,Figueredo2008a}.
It is at a parallax-measured distance of 5.41 kpc \citep{Sato2010a,Xu2009a}.

The total luminosity of the W51 protocluster complex has been estimated a few
times using IRAS and KAO to measure the peak of the SED in the far infrared.
The measurements converge on $\sim8.3\ee{6} (D/5.1\mathrm{kpc})$ \lsun
\citep{Harvey1986a,Sievers1991a}.  It is therefore among the most luminous
star-forming regions in the Galaxy.

We present observations of the inner few parsecs of the W51 protocluster
forming region.  We refer to the whole region, including W51 Main (which
contains the W51e sources) and W51 IRS 2 as `W51A'.  Other names are explained
when they are used.  An overview of the names used, including labels, is in the
Appendix.

\section{Observations}
We used the JVLA in multiple bands and configurations.  In project 12B-365, we
observed in A-array in S and C bands with 2 GHz total bandwidth.  In project
13A-064, we observed in C-Band in C (1h) and A (5h) arrays and in Ku-band in D
(1h) and B (5h) arrays.  Our spectral coverage included \ortho \oneone 4.82966 GHz
and \twotwo at 14.488 GHz with 0.3 \kms resolution and the radio recombination
lines (RRLs) H77$\alpha$ (14.1286 GHz) and H110$\alpha$ (4.8741 GHz) at 1 \kms
resolution.  The H110$\alpha$ line had lower S/N than the H77$\alpha$ line but
was otherwise similar, so it is not discussed any more in this paper.

Data reduction was performed using CASA\footnote{\url{http://casa.nrao.edu}}
\citep{McMullin2007a}.  The pipeline-calibrated products were
used, then imaging was performed using CLEAN.  For most images discussed here, we
used uniform weighting.  The reduction scripts are included in a repository
\url{https://github.com/adamginsburg/w51evlareductionscripts}.

The observations and resulting images are summarized in Table
\ref{tab:observations}.  The observation metadata, including times of observation
and program IDs, is in Table \ref{tab:obs_meta}.

\begin{table*}[htp]
\caption{Observations}
\begin{tabular}{llllllll}
\label{tab:observations}
Epoch & Frequency & BMAJ & BMIN & BPA & Noise Estimate & Dynamic Range & Jy-Kelvin \\
 & $\mathrm{GHz}$ & $\mathrm{{}^{\prime\prime}}$ & $\mathrm{{}^{\prime\prime}}$ & $\mathrm{{}^{\circ}}$ & $\mathrm{mJy}$ &  &  \\
\hline
2 & 2.5 & 0.54 & 0.53 & -175.79 & 0.20 & 54 & 6.8\ee{5} \\
2 & 3.5 & 0.41 & 0.39 & 0.66 & 0.06 & 191 & 6.3\ee{5} \\
1 & 4.9 & 0.45 & 0.38 & 88.15 & 0.18 & 154 & 3\ee{5} \\
2 & 4.9 & 0.32 & 0.29 & -84.09 & 0.04 & 322 & 5.6\ee{5} \\
3 & 4.9 & 0.33 & 0.26 & 68.06 & 0.06 & 205 & 5.9\ee{5} \\
3 & 5.0 & 0.29 & 0.22 & 80.33 & 0.03 & 421 & 7.8\ee{5} \\
2 & 5.9 & 0.27 & 0.23 & -81.28 & 0.03 & 464 & 5.7\ee{5} \\
3 & 5.9 & 0.31 & 0.19 & 74.75 & 0.03 & 421 & 6\ee{5} \\
1 & 8.4 & 0.47 & 0.39 & 82.92 & 0.08 & 860 & 9.5\ee{4} \\
2 & 12.6 & 0.38 & 0.35 & 83.67 & 0.08 & 1155 & 5.7\ee{4} \\
2 & 13.0 & 0.34 & 0.33 & 14.83 & 0.05 & 1838 & 6.4\ee{4} \\
2 & 14.1 & 0.34 & 0.33 & 68.87 & 0.09 & 1180 & 5.6\ee{4} \\
1 & 22.5 & 0.32 & 0.29 & -84.80 & 0.57 & 283 & 2.6\ee{4} \\
2 & 25.0 & 0.28 & 0.24 & -4.61 & 0.64 & 299 & 2.8\ee{4} \\
2 & 27.0 & 0.25 & 0.22 & 56.71 & 1.13 & 145 & 3.1\ee{4} \\
2 & 29.0 & 0.23 & 0.21 & 63.14 & 1.32 & 86 & 3\ee{4} \\
2 & 33.0 & 0.21 & 0.18 & 54.25 & 1.11 & 134 & 2.9\ee{4} \\
2 & 36.0 & 0.18 & 0.17 & 74.20 & 1.23 & 72 & 3.1\ee{4} \\
\hline
\end{tabular}
\par
Jy-Kelvin gives the conversion factor from Jy to Kelvin given the synthesized beam size and observation frequency
\end{table*}


While the noise levels in the images are frequently low, we did not achieve a
thermal-noise-limited image in any of the 13A-064 data.  For example, in the
C-band A-array data, the thermal noise is $\sim5$ \microjy, while the achieved
RMS is $\sim30-60$ \microjy.  The images are generally ``artifact-dominated''
at the low-signal end, where inadequately cleaned sidelobes of bright features
provide significant non-gaussian noise.  Some of these artifacts can be seen
in Figure \ref{fig:coverview}.

Despite the deficiency in the image quality, the new observations are
$\sim5\times$ more sensitive than any previous observations in bands that were
previously observed, and the new observations cover a much broader range of
frequencies.

Figures \ref{fig:kuoverview} and \ref{fig:coverview} show the best-quality
images in the Ku-band and C-band combining the short and long array
configurations.  These were made from a combination of the full bandwidth in
their respective bands, so each image includes 2 GHz of bandwidth.  These
images are used in the images in the main text of the paper, and they were used
for source identification, but they were not used for photometry.  Instead, for
the 13A-064 data, individual 1 GHz bands including only the longer-baseline
configuration were used for photometry.


\Figure{figures/diffuse/W51_Ku_overview.pdf}
{The Ku-band image of the W51 region produced with a combination of
EVLA B and D array data using uniform weighting.}
{fig:kuoverview}{0.9}{6.5in}

\Figure{figures/diffuse/W51_C_overview.pdf}
{The C-band image of the W51 region produced with a combination of EVLA
C and A array data using uniform weighting.  Annotated versions of this figure
identifying the regions discussed are available in the Appendix.}
{fig:coverview}{0.9}{6.5in}

\section{Observational Results}
We report six key observational results: 
\begin{enumerate}
    \item Section \ref{sec:pointsources}: The detection of new continuum
        sources, most of which are most likely hypercompact \hii regions. 
    \item Section \ref{sec:variability}: The detection of variability in
        some of the faint point sources over 20 year timescales
    \item Section \ref{sec:associations}: The detection of radio continuum
        sources associated with near-infrared sources, most likely un-embedded
        O-stars
    \item Section \ref{sec:twotwoemission}: The detection of \formaldehyde
        \twotwo emission around sources e2, e8, and W51 North
    \item Section \ref{sec:LOSvelo}: Measurements of line-of-sight velocities
        toward many ultracompact \hii regions using H77$\alpha$ and/or
        \formaldehyde
    \item Section \ref{sec:diffuseemission}: Detection of Orion-bar-like sharp
        edges to the W51 Main HII region, most likely tracing a
        photon-dominated region (PDR)
\end{enumerate}

\subsection{Continuum sources and photometry}
\label{sec:pointsources}
Our data are the most sensitive continuum observations yet performed on the W51
region.  We report new detections of several sources and concrete
identifications of others that were detected in previous data sets but not
reported.

We follow the naming scheme introduced by \citet{Mehringer1994a}.  For the
compact ($r<1\arcsec$) sources within 1 arcminute of W51e2, we use the name
W51e followed by a number.  We identify two new sources, e9 (19:23:43.654
+14:30:26.81) and e10 (19:23:43.956 +14:30:26.95), which were previously
detected but never officially named (as far as we were able to discover).
We additionally split source e8 into a north and south component, plus a more
extended molecular component e8mol.  We also identify a molecular component
between e1, e8, and e10, which we label e10mol.  The source positions
and approximate radii for resolved sources are shown in Figure
\ref{fig:coverview_pointsrcs} and listed in Table \ref{tab:positions}.

\begin{table*}[htp]
\caption{Source Positions}
\begin{tabular}{lllrrll}
\label{tab:positions}
Source Name & RA & Dec & Radius & Phys. Radius & SED Class & Classification \\
 &  &  & $\mathrm{{}^{\prime\prime}}$ & $\mathrm{pc}$ &  &  \\
\hline
Between & 19:23:41.47 & 14:30:51.0 & 4.5 & 0.11 & $E$ & - \\
G49.46-0.36 & 19:23:35.12 & 14:29:55.8 & 34.4 & 0.85 & $E$ & - \\
G49.46-0.38 & 19:23:40.18 & 14:29:39.1 & 19.1 & 0.47 & $E$ & - \\
W51 IRS 2 & 19:23:39.89 & 14:31:08.3 & 1.3 & 0.03 & $E$ & - \\
W51 IRS 2 bubble & 19:23:39.90 & 14:31:08.5 & 21.4 & 0.53 & $E$ & - \\
W51 Main 40 kms RRL Shell & 19:23:42.23 & 14:30:40.2 & 7.8 & 0.19 & $E$ & - \\
W51 Main 65 kms RRL shell & 19:23:42.55 & 14:30:43.1 & 10.8 & 0.27 & $E$ & - \\
W51 Main Shell & 19:23:44.68 & 14:30:23.5 & 41.5 & 1.03 & $E$ & - \\
W51G 8 & 19:23:50.69 & 14:32:50.1 & 7.1 & 0.18 & $E$ & - \\
arc & 19:23:41.00 & 14:30:37.3 & 6.0 & 0.15 & $E$ & - \\
d2 & 19:23:39.82 & 14:31:04.9 & - & - & $b,d,n$ & HCHII \\
d3 & 19:23:35.87 & 14:31:27.9 & 1.9 & 0.05 & $E$ & UCHII \\
d4e & 19:23:39.64 & 14:31:30.6 & - & - & $f,v$ & cCWB \\
d4w & 19:23:39.60 & 14:31:30.4 & - & - & $f$ & cCWB \\
d5 & 19:23:41.78 & 14:31:27.6 & - & - & $w$ & cCWB \\
d6 & 19:23:41.24 & 14:31:11.6 & - & - & $f,n$ & cCWB \\
d7 & 19:23:40.92 & 14:31:06.7 & - & - & $f$ & cCWB \\
e1 & 19:23:43.78 & 14:30:26.1 & 0.7 & 0.02 & $E$ & UCHII \\
e2 & 19:23:43.91 & 14:30:34.6 & - & - & $a,d$ & HCHII \\
e3 & 19:23:43.84 & 14:30:24.7 & - & - & $b,d$ & HCHII \\
e4 & 19:23:43.91 & 14:30:29.5 & - & - & $b,d$ & HCHII \\
e5 & 19:23:41.86 & 14:30:56.8 & - & - & $b,d$ & HCHII \\
e6 & 19:23:41.79 & 14:31:02.7 & 1.0 & 0.03 & $E$ & UCHII \\
e7 & 19:23:44.79 & 14:29:11.3 & 1.8 & 0.04 & $E$ & UCHII \\
e8n & 19:23:43.91 & 14:30:28.2 & - & - & $b/w$ & HCHII \\
e8s & 19:23:43.91 & 14:30:27.9 & - & - & $b$ & HCHII \\
e9 & 19:23:43.65 & 14:30:26.8 & - & - & $c,f$ & HCHII \\
e10 & 19:23:43.96 & 14:30:27.0 & - & - & $f$ & HCHII \\
e11 & 19:23:45.58 & 14:30:26.2 & - & - & $b$ & HCHII \\
e11d & 19:23:45.69 & 14:30:29.1 & 3.6 & 0.09 & $E$ & HII \\
e12 & 19:23:42.86 & 14:30:30.4 & - & - & $E$ & - \\
e13 & 19:23:42.81 & 14:30:36.9 & - & - & $b,c,E$ & cCWB/HII \\
e14 & 19:23:42.61 & 14:30:42.1 & - & - & $b,c,E$ & cCWB/HII \\
e15 & 19:23:38.65 & 14:30:05.7 & - & - & $b,c,E$ & UCHII \\
e16 & 19:23:46.51 & 14:29:50.2 & - & - & $E$ & - \\
e17 & 19:23:45.85 & 14:29:50.2 & - & - & $E$ & - \\
e18 & 19:23:46.18 & 14:29:46.9 & - & - & $E$ & - \\
e18d & 19:23:46.18 & 14:29:44.0 & 3.2 & 0.08 & $E$ & HII \\
e19 & 19:23:44.83 & 14:29:45.0 & - & - & $E$ & - \\
e20 & 19:23:42.86 & 14:30:27.6 & - & - & $E$ & - \\
e21 & 19:23:42.83 & 14:30:27.8 & - & - & $E$ & - \\
e22 & 19:23:42.78 & 14:30:27.6 & - & - & $E$ & - \\
e23 & 19:23:43.06 & 14:30:34.9 & - & - & $E$ & - \\
\hline
\end{tabular}
\par
Objects with name e\#d are the diffuse counterparts to point sources.  The absolute positional accuracy is $\sim0.2\arcsec$.  Sources with no radius are unresolved, with upper limits of 0.3\arcsec (0.007 pc).\\
$a$: $\nu^2$ dependence \\
$b$: $\nu^1$ dependence \\
$c$: $\nu<5$ GHz excess or negative index \\
$d$: $\nu>15$ GHz flat \\
$f$: $\nu$-independent flat \\
$E$: extended (photometry not trustworthy) \\
$n$: near bright, extended emission (may affect photometry) \\
$v$: Likely variable \\
$w$: Too weak for SED classification \\
\newline \\
HCHII: Hypercompact \hii region \\
UCHII: Ultracompact \hii region \\
HII: Extended \hii region or part of a larger \hii region \\
cCWB: candidate colliding-wind binaries \\
unknown: Sources that do not fall under other classifications \\
weak: Too weak to classify based on SED or morphology \\

\end{table*}


We include in this catalog any pointlike sources (at $\sim0.2-0.4\arcsec$
resolution) with emission in two bands, Ku and C (14 and 5 GHz).  We also
include candidate point sources (denoted with a trailing ? in Table
\ref{tab:contsrcs}) that may instead be artifacts from the data reduction
process.  In order to identify point sources, we used
uniformly-weighted maps, which remove much of the extended emission and make
point source detection simpler, but this process make local peaks
of the diffuse emission appear to be pointlike.  However, for many of the
faint and questionable peaks, we have found clear near-infrared associations
(Section \ref{sec:associations}),
suggesting that the treatment of these point sources as real is justified.

The point source data is listed in Table \ref{tab:contsrcs}.  The Epoch column
describes the data source: Epoch 1 comes from \citet{Mehringer1994a}, Epoch 2
comes from 12A-274, 12B-365, or 13A-064 (mid 2012-early
2013), and Epoch 3 (C-band only) comes from
13A-064 in 2014. For the multi-configuration combined images, we list the date
of the highest-resolution observations.

There are two flux density columns.  The first shows the peak flux density
within an aperture centered on the source with a radius equal to the beam major
FWHM; the aperture is therefore about twice as large as the full extent of the
source.  The second column, `Peak - Background', shows the peak flux minus the
minimum flux density in a box $6\times6$ beam FWHMs.  This
background-subtracted flux density is meant to account for negative bowling
that affects parts of the images; there are cases in which a bright point
source is seen on a deeply negative background such that the measured peak flux
is near zero.  When these two values agree, they are reliable, but when they
differ significantly, they are probably affected by image reconstruction
artifacts and the background-subtracted version is preferred.

\begin{table*}[htp]
\caption{Continuum Point Sources (excerpt)}
\begin{tabular}{ccccccc}
\label{tab:contsrcs}
Object & Epoch & Obs. Date & Peak $S_{\nu}$ & Peak - Background & $\sigma$ & Frequency \\
 &  &  & $\mathrm{mJy\,beam^{-1}}$ & $\mathrm{mJy\,beam^{-1}}$ & $\mathrm{mJy\,beam^{-1}}$ & $\mathrm{GHz}$ \\
\hline
d3-diffuse & 2 & 2012-10-16 & 0.38 & 0.24 & 0.21 & 2.5 \\
d3-diffuse & 1 & 2012-08-07 & - & - & 1.4 & 22.5 \\
d4e & 2 & 2012-10-16 & 0.79 & 1.1 & 0.051 & 4.9 \\
d4e & 2 & 2014-04-19 & - & - & 0.08 & 33.0 \\
d4w & 2 & 2013-03-02 & 0.65 & 0.95 & 0.11 & 12.6 \\
d5 & 1 & 1992-10-25 & -0.02 & 0.49 & 0.19 & 4.9 \\
d5 & 2 & 2012-06-21 & 1.4 & 2.7 & 0.85 & 27.0 \\
d6 & 3 & 2014-04-19 & 0.38 & 0.39 & 0.05 & 5.9 \\
e1 & 2 & 2012-10-16 & 11 & 11 & 0.21 & 2.5 \\
e1 & 1 & 2012-08-07 & 12 & 14 & 1.4 & 22.5 \\
e10 & 2 & 2012-10-16 & 1.7 & 2 & 0.051 & 4.9 \\
e10 & 2 & 2014-04-19 & 3.2 & 4.7 & 0.08 & 33.0 \\
e11 & 2 & 2013-03-02 & 0.54 & 0.72 & 0.11 & 12.6 \\
e12? & 2 & 2012-10-16 & 0.38 & 0.42 & 0.051 & 4.9 \\
e12? & 2 & 2012-06-21 & 0.98 & 1.5 & 0.85 & 27.0 \\
e13 & 2 & 2012-10-16 & 0.6 & 0.81 & 0.039 & 5.9 \\
e14 & 2 & 2012-10-16 & 0.92 & 1.4 & 0.21 & 2.5 \\
e14 & 1 & 2012-08-07 & 0.7 & 0.7 & 1.4 & 22.5 \\
e15 & 3 & 2012-10-16 & 0.35 & 0.33 & 0.063 & 4.9 \\
e15 & 2 & 2014-04-19 & - & - & 0.08 & 33.0 \\
e16? & 2 & 2013-03-02 & 0.19 & 0.28 & 0.11 & 12.6 \\
e17? & 2 & 2012-10-16 & 0.069 & 0.22 & 0.051 & 4.9 \\
e17? & 2 & 2012-06-21 & -0.76 & 1.9 & 0.85 & 27.0 \\
e18? & 3 & 2014-04-19 & 0.52 & 0.47 & 0.05 & 5.9 \\
e19? & 2 & 2012-10-16 & 1.9 & 1.8 & 0.21 & 2.5 \\
e19? & 1 & 2012-08-07 & 0.2 & 1 & 1.4 & 22.5 \\
e2 & 3 & 2012-10-16 & 11 & 11 & 0.063 & 4.9 \\
e2 & 2 & 2014-04-19 & 1.5\ee{2} & 1.5\ee{2} & 0.08 & 33.0 \\
e20? & 2 & 2013-03-02 & 0.81 & 0.52 & 0.11 & 12.6 \\
e21? & 2 & 2012-10-16 & 0.16 & 0.36 & 0.051 & 4.9 \\
e21? & 2 & 2012-06-21 & 0 & 1.7 & 0.85 & 27.0 \\
e22? & 2 & 2012-10-16 & 0.12 & 0.25 & 0.039 & 5.9 \\
e23? & 2 & 2012-10-16 & -0.05 & 0.96 & 0.21 & 2.5 \\
e23? & 1 & 2012-08-07 & -0.6 & 0.8 & 1.4 & 22.5 \\
e3 & 3 & 2012-10-16 & 6.6 & 6.7 & 0.063 & 4.9 \\
e3 & 2 & 2014-04-19 & 4.5 & 8.1 & 0.08 & 33.0 \\
e4 & 2 & 2013-03-02 & 9.1 & 9.4 & 0.11 & 12.6 \\
e5 & 2 & 2012-10-16 & 7.8 & 8 & 0.051 & 4.9 \\
e5 & 2 & 2012-06-21 & 24 & 25 & 0.85 & 27.0 \\
e6 & 2 & 2012-10-16 & 1.5 & 1.3 & 0.039 & 5.9 \\
e7 & 2 & 2012-10-16 & 0.37 & 0.24 & 0.21 & 2.5 \\
e7 & 1 & 2012-08-07 & -0.2 & 1.3 & 1.4 & 22.5 \\
e8n & 2 & 2012-10-16 & 0.81 & 1.1 & 0.051 & 4.9 \\
e8n & 2 & 2014-04-19 & 4.2 & 5.8 & 0.08 & 33.0 \\
e8s & 2 & 2013-03-02 & 2.3 & 2.6 & 0.11 & 12.6 \\
e9 & 2 & 2012-10-16 & 1.9 & 2.1 & 0.051 & 4.9 \\
e9 & 2 & 2012-06-21 & 2.5 & 2.9 & 0.85 & 27.0 \\
\hline
\end{tabular}
\par
An excerpt from the point source catalog.  For the full catalog, see Table \ref{tbl:contsrcs_full}
\end{table*}


Figure \ref{fig:d4sed} shows the cutout images with apertures for the source
d4e.  \todo{Additional figures for all candidates are included in the Appendix.}

\Figure{figures/pointsource_seds/d4e_SED.png}
{Cutouts of the source W51 d4e and its SED.
The red circle shows the source aperture used to measure the peak and total
flux.  The red line shows a 1\arcsec scalebar.  The Epochs are indicated as E1,
E2, and E3.
In the SED plot, the dark and light red show the 1-$\sigma$ and 3-$\sigma$
noise levels.  The SED plot is shown twice, first linear-linear, and second
linear-log.
Green points are from Epoch 1, blue points are Epoch 2, and red points are
Epoch 3.  Squares show the peak flux density, circles show the
`background-subtracted' flux density - i.e., the peak minus the minimum value
in the map.
The dotted curve shows a linear SED ($S_\nu \propto \nu$) and the dashed shows
a power-law curve with $\alpha=2$, i.e. $S_\nu \propto \nu^2$.  Both curves are
normalized to go through the 12.1 GHz Epoch 2 data point.  In this case, all of
the Epoch 1 (green) points are within the noise, while all of the Epoch 2 and 3
(blue and red) points are detected at $>5$-$\sigma$.
}
{fig:d4sed}{0.5}{6.5in}

\subsubsection{Variability}
\label{sec:variability}
The multi-epoch data demonstrate that some of these sources are variable.  The
most convincing case for variability is in the (double) source d4 (Figure
\ref{fig:d4sed}).  In the \citet{Mehringer1994a} data, there is no hint of
emission at this location, with a 3-$\sigma$ upper limit of 0.6 mJy.  At the
same position and frequency in 2014, there is a $1.0 \pm 0.06$ mJy source at
the position of d4e.  The case for variability in other sources is weaker but
nonetheless suggestive.

% \todo{Move this somewhere relevant.}
% We found a small offset between the 1994 and 2014 images using a
% cross-correlation method (\url{image-registration.rtfd.org}); the 2014 data
% were shifted by 0.16\arcsec, 0.20\arcsec to the southeast from the 1994 data.
% \todo{There also appears to be some "squeezing" mode, which is inexplicable.
% It probably makes more sense to centroid on e2 and match that way.}


\subsubsection{Source Associations}
\label{sec:associations}
We compare our extracted source catalog with the MOXC Chandra X-ray catalog
\citep{Townsley2014a}.  The associations are listed in Table
\ref{tab:associations}.  We discuss some individual source associations
in more detail in this section.

Both lobes of d4 are
0.3\arcsec from MOXC 192339.62+143130.3 and d6 is 0.3\arcsec from MOXC
192341.23+143111.8.    These
separations may indicate that either the HII regions or the MOXC point sources
are associated with diffuse material, e.g. outflows,
near stars rather than the stars themselves.  
Candidate sources e20 and e22 are $<0.1$\arcsec from
MOXC 192342.86+143027.5 and 192342.77+143027.5, respectively.
There are a handful of X-ray
sources within W51 IRS2 that could be associated with faint HCHII regions that
we do not detect due to confusion, but these could also be associated with
outflowing material (e.g., the Lacy jet; see Section \ref{sec:lacyjet}).  The
lack of correlation between X-ray and HCHII regions clearly indicates that the
embedded OB stars within these HCHII regions are weak X-ray emitters.  

We compared our point source catalog to the UKIDSS K-band and UWISH2 \hh images
to search for infrared associations with our detected radio sources
\citep{Lucas2008a,Froebrich2011a}.  Most of the W51e sources have no
association, as is expected given the high extinction in this region.  The
diffuse region e7 to the south is an exception, associated with UGPS
J192344.79+142911.2 \citep{Lucas2008a}.  Three candidate sources, e20, e21, and
e22, have clear K-band counterparts, though the UGPS cataloged sources do not
align with the evident K-band sources.    e5 exhibits some K-band emission,
though because of the diffuse surroundings, this emission is not cataloged in
the UGPS.

The source d7 is clearly associated with a luminous infrared source, which is
seen both in UKIDSS \citep{Lucas2008a} and NACO \citep{Figueredo2008a} images.
However, the reported position of \citet{Goldader1994a} source RS15 is
0.7\arcsec away, even though in the UKIDSS images there is no detected flux at
this position.  More curiously, the associated MOXC Chandra source
\citep{Townsley2014a} is more closely aligned with the offset RS15 position
than the correct location.  Despite this confusion, we regard the NIR, X-ray,
and radio features to have a common origin.



%\todo{Is this a worthwhile observation?  The flux distribution of detected
%sources at C-band (4.9 GHz) is essentially flat from 1\ee{-4} to 1\ee{-2} Jy.
%There is a hint of a power-law or bimodal distribution from 1\ee{-4} to
%1\ee{-1} Jy at 14.1 GHz.  The distribution probably reflects varying optical
%depth.}

\subsection{\formaldehyde \twotwo emission}
\label{sec:twotwoemission}
We detect \formaldehyde \twotwo emission around W51e2, W51e8, in a region
between e1, e8, and e10, and in W51 North.  We report the tentative detection
of an extended structure between e2 and e1, though this structure is weak
and could be an artifact from the image reconstruction process.
The fitted emission line parameters are listed in Table \ref{tab:emission22}.

\begin{table*}[htp]
\caption{\formaldehyde \twotwo emission line parameters}
\begin{tabular}{ccccccccc}
\label{tab:emission22}
Object Name & Amplitude & $E$(Amplitude) & $V_{LSR}$ & $E(V_{LSR})$ & $\sigma_V$ & $E(\sigma_V)$ & $\Omega_{ap}$ & Detection Status \\
 & $\mathrm{mJy}$ &  & $\mathrm{km\,s^{-1}}$ &  &  &  & $\mathrm{sr}$ &  \\
\hline
NorthCore & 0.652 & 0.018 & 58.877 & 0.074 & 2.375 & 0.074 & 2.3\ee{-10} & - \\
e2-e8 bridge & 0.369 & 0.022 & 56.554 & 0.07 & 1.0 & 0.07 & 1.5\ee{-10} & - \\
e2\_a & 0.641 & 0.026 & 57.412 & 0.1 & 2.167 & 0.1 & 5.7\ee{-11} & - \\
e2\_b & 1.085 & 0.035 & 56.083 & 0.097 & 2.605 & 0.097 & 5.7\ee{-11} & - \\
e2\_c & 0.612 & 0.026 & 56.01 & 0.19 & 3.95 & 0.19 & 5.7\ee{-11} & - \\
e8mol & 1.551 & 0.061 & 60.55 & 0.13 & 2.85 & 0.13 & 2\ee{-11} & - \\
e8mol\_ext & 1.045 & 0.024 & 60.16 & 0.089 & 3.439 & 0.089 & 7.2\ee{-11} & weak \\
e10mol & 0.999 & 0.039 & 58.01 & 0.21 & 4.69 & 0.21 & 2\ee{-11} & - \\
e10mol\_ext & 0.813 & 0.018 & 58.507 & 0.094 & 3.593 & 0.094 & 8.5\ee{-11} & weak \\
\hline
\end{tabular}
\end{table*}


\subsection{Line-of-sight velocities}
\label{sec:LOSvelo}
We have detected H77$\alpha$ emission from many of the ultracompact and
hypercompact \hii regions within W51.  We report their line-of-sight velocities
as measured from gaussian profile fits to their extracted spectra.

The H77$\alpha$ emission line parameters are listed in Table \ref{tab:h77a}, and
the \para \twotwo absorption line parameters are in Table \ref{tab:absorption22}.
%The emission line parameters are only relevant for source e8mol.

The e5 and e6 sources are close to one another (separation 3.1 \arcsec, or
projected distance 0.08 pc), and both exhibit \formaldehyde absorption at 62-63
\kms.  e6 is detected in H77$\alpha$ at 68 \kms, suggesting that the true
velocity of both e5 and e6 are not the same as the \formaldehyde absorption
lines.  The \formaldehyde absorption is likely from cloud in the foreground.
%    , \emph{but} the line profile of the e6 H77$\alpha$ line suggests that the
%velocity measurement may be affected by image reconstruction artifacts from
%50-63 \kms.

\begin{table*}[htp]
\caption{\formaldehyde \twotwo absorption line parameters}
\begin{tabular}{ccccccccc}
\label{tab:absorption22}
Object Name & Amplitude & $E$(Amplitude) & $V_{LSR}$ & $E(V_{LSR})$ & $\sigma_V$ & $E(\sigma_V)$ & $\Omega_{ap}$ & Detection Status \\
 & $\mathrm{mJy}$ &  & $\mathrm{km\,s^{-1}}$ &  &  &  & $\mathrm{sr}$ &  \\
\hline
e1 & -2.922 & 0.034 & 62.531 & 0.078 & 5.81 & 0.078 & 2.9\ee{-11} & ambig \\
e2 & -21.186 & 0.082 & 56.8728 & 0.009 & 2.0231 & 0.009 & 2.5\ee{-11} & - \\
e3 & -2.57 & 0.1 & 64.26 & 0.11 & 2.35 & 0.11 & 9.1\ee{-12} & - \\
e5 & -1.944 & 0.094 & 62.739 & 0.032 & 0.576 & 0.032 & 2.4\ee{-11} & - \\
e6 & -0.598 & 0.015 & 63.771 & 0.061 & 2.149 & 0.061 & 2.4\ee{-10} & - \\
e9 & -0.477 & 0.078 & 55.4 & 0.22 & 1.15 & 0.22 & 2\ee{-11} & - \\
e10 & -0.63 & 0.1 & 66.74 & 0.21 & 1.1 & 0.21 & 1.6\ee{-11} & - \\
\hline
\end{tabular}
\end{table*}


\begin{table*}[htp]
\caption{H$77\alpha$ emission line parameters}
\begin{tabular}{cccccccc}
\label{tab:h77a}
Object Name & Amplitude & $\sigma$(Amplitude) & $V_{LSR}$ & $\sigma(V_{LSR})$ & $dV (\sigma)$ & $\sigma(dV)$ & Detection Status \\
 & $\mathrm{mJy}$ & $\mathrm{mJy}$ & $\mathrm{km\,s^{-1}}$ & $\mathrm{km\,s^{-1}}$ & $\mathrm{km\,s^{-1}}$ & $\mathrm{km\,s^{-1}}$ &  \\
\hline
e1 & 3.0 & 0.2 & 54.87 & 0.85 & 10.9 & 0.85 & - \\
e2 & 0.6 & 0.13 & 56.0 & 3.8 & 15.3 & 3.8 & - \\
e3 & 1.48 & 0.33 & 59.8 & 2.7 & 10.5 & 2.7 & - \\
e4 & 0.56 & 0.39 & 57.2 & 4.4 & 5.4 & 4.4 & - \\
e5 & 0.25 & 0.18 & 52.6 & 4.7 & 5.8 & 4.7 & weak \\
e6 & 0.183 & 0.051 & 68.6 & 4.9 & 15.4 & 4.9 & - \\
e9 & 0.15 & 0.18 & 66.8 & 7.9 & 5.5 & 7.9 & weak \\
e10 & 0.27 & 0.2 & 51.2 & 9.8 & 11.6 & 9.8 & weak \\
\hline
\end{tabular}
\end{table*}


\subsubsection{Velocity Fields}
Many works have examined the velocity field of the gas in W51e2
\citep{Zhang1997a,Keto2008b,Shi2010b,Shi2010a,Goddi2015b}.  For comparison, we
show the velocity field of a $6\arcsec\times6\arcsec$ region centered on W51e2
in Figure
\ref{fig:w51e2velofield}.

\FigureTwo
{figures/velocity/w51e2zoom_natural_contoured_marked.png}
{figures/velocity/w51e2zoom_briggs_contoured_marked.png}
{Velocity (moment 1) maps of \ortho \twotwo in the W51e2 region using (a) the
naturally weighted map and (b) the Briggs-weighted map.  The X's mark
\citet{Shi2010a} cores.   
The contours show the moment 0 maps integrated from 50 to 63 \kms at
[-0.4,-0.3,-0.2,-0.1, 0.010, 0.020, 0.030, 0.040] Jy \kms (natural) and
[-0.28,-0.21,-0.14,-0.07, 0.007, 0.0105, 0.014] Jy \kms (Briggs).
}
{fig:w51e2velofield}{1}{3.5in}

\subsection{Diffuse continuum emission features}
\label{sec:diffuseemission}
There are a few new notable continuum emission features detected in our data
that were not seen in previous shallower data.

Near source e11, there is a bow-shaped feature (Figure \ref{fig:e11bow}).
There are no known associated sources at shorter wavelengths, though in Spitzer
bands there is some diffuse emission at this location.  The bow-like structure
points away from e11, suggesting that it is the driving source.

\Figure{figures/diffuse/e11_bow.png}
{A bow-shaped feature toward the northeast of the source e11 in the Ku-band
continuum image.  The feature resembles the bow shocks of Herbig-Haro object,
but is more likely to be an HII region given its smooth structure.}
{fig:e11bow}{0.5}{6.5in}

The source d3 is a diffuse HII region with radius $r\sim1.9$\arcsec.  It is
associated with a Spitzer source and the 2MASS source 2MASX J19233591+1431288.
The sources d3, e1, e6, and e7 are compact (0.02-0.07 pc) and fairly round;
they are classed as ultracompact HII regions.

The diffuse emission associated with W51 Main traces a broad arc
($r\sim42\arcsec$) that has been seen in many previous data sets.  The new
deeper, higher-resolution data are dramatically improved.  Where in previous
observations, a relatively smooth and clumpy structure was seen, the new images
reveal a network of wispy, filamentary structures.  Figure
\ref{fig:w51mainpeak} shows the region between the W51e and W51 IRS2 clusters.
While this area contains few clear individual sources, it accounts for the
majority of the radio luminosity of the W51 Main region and about half the
total luminosity of the W51 Main/IRS2 complex.

The filamentary structures in the W51 Main peak are unresolved along the short
axis, with aspect ratios $>25$.   These resemble the ``stringlike ionized
features'' noted by \citet[][see
\url{http://images.nrao.edu/402}]{Yusef-Zadeh1990a}, which are associated
primarily with the Orion Bar PDR and its fainter cousin to the northeast.  The
similarity suggests that these features are PDRs, highlighting the sharp
interaction points between the HII region and the surrounding molecular cloud.

The two most prominent of these sharp features are on opposite sides of the W51
Peak (Figure \ref{fig:w51mainpeak}, blue arrows).  The long vertical filament
at RA = 19:23:42.4 (Filament A) approximately faces W51e, while the S-shaped
filament at RA=19:23:41.6 (Filament B) faces nothing in particular.  The
presence of multiple features raises questions about the ionizing source.  In
Orion, the Trapezium is 0.2 pc in projection from the Bar PDR.  The left
filament is 0.6 pc from the W51e cluster, which contains enough O-stars to
illuminate the filaments, but it is not clear whether there is a clear
line-of-sight from those stars, which appear to be deeply embedded in molecular
material, to the filament.  It is possible that there are additional OB stars
embedded in the W51 Peak driving much of its luminosity that have never been
resolved because they are confused with the ionized nebula they produce.

\FigureTwo
{figures/diffuse/w51main_peak.png}
{figures/diffuse/w51main_peak_diff.png}
{({\it a}) A C-band image of the W51 Main peak intensity region.  This region
accounts for more than half of the radio luminosity of W51.
({\it b}) An unsharp-masked version of the image with a 0.3\arcsec smoothing
kernel. 
There are
two pointlike sources in field marked by red arrows, e14 at center-left and the
cometary e13 toward the lower left.
The filamentary features mentioned in Section \ref{sec:diffuseemission} are
identified by blue arrows.  The cyan arrow points along a possible ionized flow,
traced by elongated features parallel to the arrow and a rounded feature to the
southwest.  A counterpart to the northeast is barely visible.
}
{fig:w51mainpeak}{1}{3.5in}

Additionally, the W51 Main ridge is cleanly detected in H77$\alpha$.  Filament
B peaks in the range 40-50 \kms.  From its sharp start toward the southeast,
there is continuous H77$\alpha$ emission toward the northeast from 40 to 80
\kms.

The H77$\alpha$ lines reveal two bubbles that overlap along the line of sight
but are clearly distinct in velocity (Figure \ref{fig:rrl_chan_main}).  A
smaller bubble at 40 \kms shows a velocity gradient orthogonal to the gradient
seen next to filament A, increasing in velocity from southwest to northeast;
this gradient may trace the edges of an expanding spherical shell.  The large
bubble is centered at about 65 \kms and does not exhibit any clear gradients
(Figure \ref{fig:coverview_diffuse}).

\Figure{figures/rrls/h77a_channelmaps_W51Main.png}
{Channel maps of the H77$\alpha$ line integrated over 5 \kms windows
for the W51 Main region.  There are small ring features notable at low and high
velocities; these are identified in Figure \ref{fig:coverview_diffuse}.}
{fig:rrl_chan_main}{1}{7in}

The HII region between W51 Main and W51 IRS 2 appears connected to the W51 IRS2
bubble in the continuum, but it peaks at around 75 \kms.  Along this region,
there are weak `striations' at PA 126 degrees, vaguely orthogonal to a line
pointing back to IRS 2.  These striations are similar in scale to PSF artifacts
from bright pointlike sources, but no such artifaces are seen at this PA
elsewhere in the image and they are smoother than similar features, so they are
probably real.  They may trace the outer edge of the expanding IRS 2 region.
There is also an arc to the southwest of this between-clusters region.

The IRS2 region peaks in velocity around 62 \kms at the center, but exhibits a
consistent gradient from its center to its surroundings, with a more extended
component peaking around 47 \kms.  There is a clear shell around IRS2
with maximum projected radius 21\arcsec (0.5 pc; Figures \ref{fig:w51irs2} and
\ref{fig:rrl_chan_irs2}).  There is a prominent sharp edge feature extending to
the northeast from the central cluster: it may trace the outflow of hot
material from the cluster.

Parts of the shell close to IRS2 are detected in radio recombination lines
(H77$\alpha$), but the regions marked in Figure \ref{fig:w51irs2} with red
arrows are not.  The shell to the northwest peaks at $v_{lsr}=37.5$ \kms, and a
gradient is observed from that shell to the peak velocity of IRS2 at $v_{lsr} =
62.6$ \kms.  The velocity structure is inconsistent with spherical outflow, but
it is consistent with a conical flow structure like that expected from
disk-driven outflows.  This flow structure likely indicates that the cluster
has evacuated a large cavity approximately along the line of sight, which is
consistent with the low observed extinction toward IRS2.  

\FigureTwo
{figures/diffuse/irs2_C_high.png}
{figures/diffuse/irs2_C_diff.png}
{(\textit{a}) The W51 IRS 2 region in C-band continuum and (\textit{b}) unsharp
masked with a 0.3\arcsec kernel.  Concentric ring features surround the IRS 2
region.  A V-shaped feature to the north (cyan arrow) may highlight the edge of
gas flows out of the cluster, extending to the long linear feature pointing
northeast (blue arrow); the corresponding linear feature to the northeast is
\emph{not} observed at Ku-band and is likely to be an imaging artifact.  An
apparent shell is observed centered on the IRS2 cluster, with edge
features to the south and east (red arrows).
}
{fig:w51irs2}{1}{3.5in}

\Figure{figures/rrls/h77a_channelmaps_IRS2.png}
{Channel maps of the H77$\alpha$ line integrated over 5 \kms windows
for the IRS2 region.  Around 30-35\kms, the edges of the IRS2 region become
visible, illustrating the blueshifted cavity discussed in Section
\ref{sec:diffuseemission}.}
{fig:rrl_chan_irs2}{1}{7in}

% Cometary clouds in the W51 region:
% 19:23:42.397 +14:30:07.70 ?

\section{Analysis \& Discussion}
\subsection{The stellar mass}
\label{sec:stellarmass}

% this work is done mostly in the w51_singledish_maps radial_profile code
We have re-measured the infrared luminosity of the W51 protoclusters using Herschel
Hi-Gal data \citep{Molinari2010a,Traficante2011a}, fitting an SED from the 70
to 500 \um with a single blackbody component.
While this is not a very good
measurement of the dust temperature - multiple temperature components are
evident \citep{Sievers1991a} - it provides a good approximation to the total
infrared luminosity, which is dominated by a single warm ($\sim60$ K)
component.  The infrared luminosity is about $L\sim2\ee{7}$ \lsun within a 2 pc
radius\footnote{In many bands, W51 Main and W51 IRS2 are
saturated, so we have interpolated from neighboring pixels to estimate the flux.
Our luminosity is therefore a lower limit.}, which includes
both the W51 IRS2 and W51 Main protoclusters.  It does not include the
mid-infrared luminosity, which may provide an additional $\sim25-50\%$ based on
the IRAS 12 and 25 \um measurements.  Our measured total luminosity is a little
larger than the previous estimates based on IRAS and KAO measurements
\citep{Harvey1986a,Sievers1991a}.  A luminosity $L=2\pm0.5\ee{7}$ \lsun implies
a stellar mass $M_{cl} = 6700 \pm 2300$ \msun, with a corresponding number of
O-stars (greater than 20 \msun) $N_O = 19 \pm 6$, assuming a
\citet{Kroupa2001a} IMF at the zero-age main sequence using \citet{Vacca1996a}
stellar parameters.  This is a lower limit on the present day O-star population,
since much of the short-wavelength radiation is able to escape and is not
reprocessed into the far infrared we have used to infer the luminosity.

Of these expected $\sim20$, we know of 6 as infrared-detected, spectrally
confirmed O-stars.  \citet{Figueredo2008a} found 4 exposed O-type stars and
\citet{Barbosa2008a} found an additional two with strong infrared excess.  None
of these coincide with ultracompact or hypercompact \hii regions, but all are
in the bright and diffuse IRS2 region.

\citet{Mehringer1994a} found an additional 8 ultracompact and hypercompact \hii
regions, all of which appear to be B0 or earlier based on their radio-derived
ionizing photon luminosity.  In this work, we report a handful more.
The other 9-20 O-stars expected to have \emph{already} formed given the
observed total luminosity may be these most luminous hypercompact \hii regions.
However, it is difficult to explain the total luminosity of the region from
hypercompact HII regions alone, since their emission is necessarily confined to
a small region and they contribute minimally to the overall radio luminosity.
The thermal radiation from the OB-stars in \hii regions may illuminate a large
portion of the cloud and contribute substantially to the infrared luminosity.

Another possibility is that the O-stars providing most of the observed
luminosity are near or within the ``shell'' structure of the W51 Main region
(Figure \ref{fig:coverview_diffuse}).  This region dominates the optical and
radio luminosity of W51 Main (though it provides $\lesssim 1/4$ of the FIR
luminosity) and is bright enough in near-infrared and radio continuum emission
to make detection of point sources impossible, explaining why no O-stars have
been previously confirmed.  There are multiple overlapping HII bubbles in the
W51 Main shell at different velocities (Figure \ref{fig:rrl_chan_main}),
suggesting that there are interacting and  expanding bubbles, which in turn
implies that there is a separation between the driving sources and the HII
region of at least the bubble sizes, $\sim0.2-0.3$ pc.  This scenario suggests
the existence of a population of un-embedded O-stars in W51.  

\subsection{The nature of the continuum sources}
% 27 dervied by looking at source_positions table and subtracting extended
% sources
Out of the 27 compact continuum sources reported, most are ultra- or
hyper-compact HII regions.  The ultracompact HII regions are clearly
identified as resolved sources, with $r>0.005$ pc.  The HCHII regions
are more challenging to identify, as they are observed only as unresolved
sources.  Normally, these could be identified by their SEDs, which should
follow a $\alpha\propto2$ Rayleigh-Jeans law up to some turnover point, at which
they would become optically thin and turn over to a $\alpha\propto{-0.1}$ power law
\citep{Wilson2009a}.  However, the SEDs we observed do not exhibit such clean
behavior for any cases except e2.  For the rest, we see $\nu^1$ power laws at
low frequencies (e.g., e4), negative power-laws at low frequencies (e.g.,
e9), or entirely flat SEDs (e.g., e10)\footnote{More complete SED classifications
are given in Table \ref{tab:positions}, and plotted SEDs can be seen in
Appendix \ref{sec:SEDs}.}.

An SED with $\alpha\sim1$ or $\alpha\sim0$ can be explained by density
gradients \citep{Keto2008a,Galvan-Madrid2009a}. However, a minimum at $\sim6$
GHz cannot.  Aditionally, an optically thick HII region that is just unresolved
would have $S_{5 GHz} \sim 70$ mJy, so for the sources discussed in this
section with $S_{5 GHz} < 1$ mJy, the upper limit on the source radius is
$\lesssim160$ au.  If any of these sources are HII regions, then, they are
among the most compact of HCHII regions in the galaxy.




Negative spectral indices at low frequencies are usually assumed to indicate
synchrotron emission \citep{Wilson2009a,Condon2007a}, but young and forming
stars and HII regions are generally not strong synchrotron emitters.  However,
colliding-wind binaries (CWBs) often exhibit nonthermal SEDs, likely caused by
accelerated particles in the wind-wind collision zone \citep{De-Becker2013a}.
Stellar winds around massive stars may also exhibit $\alpha<2$ power-laws, but
they are not expected to deviate far from $\alpha\sim0.6-0.7$
\citep{Wright1975a,Panagia1975b,Reynolds1986a}.

CWBs typically have radio luminosities
$L_{rad}\sim10^{29}-10^{30}$ erg \pers \citep{De-Becker2013a}.   At the
distance of W51, this corresponds to 0.5-5 mJy at 5 GHz, so such binaries
should be detected.  By contrast, single star winds range from $L_{rad} \sim
10^{27}-5\times10^{29}$ erg \pers, with only a few known in the high range
\citep{Bieging1989a}, so these are less likely to be detected.

Both CWBs and radio-bright stellar wind sources should be bright in the
optical and near-infrared, with stellar luminosities $10^5-10^6$ \lsun.
Much of W51 is obscured by infrared extinction from the Galactic plane
and the local cloud, but in the near-IR an unextincted O9 star would have
$M_K\sim10.5$
\citep{Pecaut2013a}\footnote{\url{http://www.pas.rochester.edu/~emamajek/EEM_dwarf_UBVIJHK_colors_Teff.txt}},
or $M_K\sim 13.0$ with (non-local) extinction $A_K=2.6$ \citep{Goldader1994a}.
Some of these stars therefore ought to be detected in the NIR, and we
report such detections in Table \ref{tab:associations}.

Based on these arguments, we have classified each detected compact source in
Table \ref{tab:positions}.  We report 12 candidate CWBs throughout the observed
region.  These sources constitute a large fraction of the un-embedded O-star
population discussed in Section \ref{sec:stellarmass}, and their presence confirms
that un-embedded, main-sequence O-stars reside in the same cloud as the still-forming
clusters.

\subsubsection{The faint sources in W51 Main}

The sources e20, e21, and e22 are likely members of a distributed O-star
population, rather than members of the clusters of compact HII regions (Figure
\ref{fig:w51e20cluster}).  They were detected by \citet{Goldader1994a} in the
K-band at $m_K < 12$, making these the brightest NIR sources outside of W51
IRS2 (Table \ref{tab:associations} gives their associated NIR source names
from large surveys).  Their luminosities suggest they are O-type stars.  Since
they were also detected in X-rays \citep{Townsley2014a}, these are very strong
candidate CWBs.  The high infrared K-band brightness and weak radio
continuum from these sources, in contrast with the bright radio continuum and
absence of infrared emission from e1/e2, supports this hypothesis.  
Assuming they are early O-stars, e20, e21, and e22 are capable of providing the
ionizing and infrared luminosity observed from the W51 Main HII region.

\FigureTwo
{figures/diffuse/peak_cluster_C_high.png}
{figures/diffuse/peak_cluster_C_diff.png}
{({\it a}) A C-band image of the W51 e20/e21/e22 cluster.
({\it b}) An unsharp-masked version of the image with a 0.3\arcsec smoothing
kernel. 
The three point sources toward the bottom of these images are e20, e21, and e22
respectively.  They correspond to two detected infrared K-band sources from
\citet{Goldader1994a} and have K-band luminosities consistent with spectral
type O4 or earlier.  The other point sources seen in this field are not detected
in the infrared.
}
{fig:w51e20cluster}{1}{3.5in}


Questions remain: %\todo{Maybe move this into a further discussion section?}
How close are the e1/e2 and e20/e21/e22 clusters?  They are
separated by 0.36 pc in projection, but we presently have no direct information
about the line-of-sight velocity of the e20 cluster.  There is no 68 \kms
\formaldehyde absorption toward the e20 sources, but morphologically this is
expected since they just managed to avoid the Dark Lane seen in the NIR K-band.  It is
likely that the other sources in this general area, e12, e13, and e23, are also
OB stars that are coincidentally behind infrared dark clouds that obscure them
in the infrared.  It is possible that all of the W51 Main compact sources are
within a $\sim 0.5$ pc sphere.

Are this new cluster, e20-e22, the distributed sources, e12, e13, e14, and e23,
and the e1/e2 cluster going to merge?  If so, there are enough O-stars present
already to imply a $>5000 \msun$ cluster.  If not, though, it seems that the
W51 e1/e2 cluster is a massive cluster forming within an even more massive OB
association.  This latter scenario hints that the OB association stars may be
responsible for shutting off inflow into the forming cluster.  While the gas
already within $\sim 1$ pc of that cluster will free-fall into the cluster in
$\lesssim 1$ Myr, any additional gas will be ionized before it can fall in,
freely streaming out of the cloud and away from the forming cluster.  In this
way, the cluster's formation would be externally regulated rather than
self-regulated.

%Why is the W51 Main HII region brightest precisely where it is?  The molecular
%gas peaks at e1/e2 \citep{Parsons2012a}, but there is little sign of ionization
%there outside the ultracompact HII regions.  The HII region seems to peak in an
%area containing the least molecular gas, which suggests that whatever mass is
%there is predominantly ionized.  The ionized gas seems to form an ionized outer
%shell around the dense molecular gas, suggesting that any further infall into
%this system must happen along thick, shielded flows.

\Figure
{figures/diffuse/continuum_on_c18o.png}
{Contours of the C-band (5 GHz) continuum overlaid on an integrated intensity
map of \ceighteeno 3-2 from 45 to 65 \kms \citep{Parsons2012a}.  The contours
go from 0.1 to 10 mJy in 5 logarithmic steps.  The \ceighteeno peaks on the
e1/e2 region, and has a clear minimum corresponding to the peak of the radio
continuum emission.
}
{fig:contonco}{1}{6.5in}

\subsubsection{The W51 IRS2 faint sources}

Of the compact radio sources identified near W51 IRS2, all but d2 have infrared
associations (Table \ref{tab:associations}).  Four of them also have X-ray
associations.  Except for d3, which is clearly associated with a moderately
extended HII region, these are strong candidate CWBs.

An alternate explanation for the SED, location, and variability of d4 is
discussed in Section \ref{sec:d4}.

\begin{table*}[htp]
\caption{Source Associations}
\begin{tabular}{llll}
\label{tab:associations}
Source Name & X-ray & NIR & Goldader 1994 \\
 &  &  &  \\
\hline
d3 & - & 2MASS19233587+1431286 & - \\
d4e & CXOJ192339.6+143130 & UGPSJ192339.65+143130.9 & - \\
d4w & CXOJ192339.6+143130 & UGPSJ192339.65+143130.9 & - \\
d5 & - & UGPSJ192341.77+143127.6 & - \\
d6 & CXOJ192341.1+143110 & UGPSJ192341.29+143111.8 & - \\
d7 & CXOU192340.96+143106.7 & UGPSJ192340.91+143106.7 & RS15 \\
e7 & - & UGPSJ192344.79+142911.2 & - \\
e14 & - & UGPSJ192342.60+143042.2 & - \\
e15 & - & UGPSJ192338.65+143005.8 & - \\
e20? & CXOU192342.86+143027.5 & UGPSJ192342.85+143027.7 & RS7 \\
e21? & (same as e20) & (same as e20) & (same as e20) \\
e22? & CXOU192342.77+143027.5 & UGPSJ192342.84+143027.5 & RS8 \\
\hline
\end{tabular}

\end{table*}


% \subsection{The nature of the continuum sources}
% Most of the detected point sources are likely to be hypercompact HII regions.
% All that are detected at low frequencies (2-6 GHz) are also detected at 15 GHz.
% Many of these sources have peculiar spectral energy distributions, with maxima
% at 2 and 15 GHz and a minimum at 6 GHz.  These SEDs are not easily explained by
% either simple HII region models or as synchrotron emission sources.  However,
% the association of all of these sources with the W51 region - all are within
% $\sim2\arcmin$ of the central clusters - means they are unlikely to be
% background galaxies.  
% 
% Given the faintness of these sources ($S_{5 GHz} \sim 1$ mJy), they must be
% very small or optically thin if they are free-free emission sources.  A 1 mJy
% source has a brightness temperature of $\sim$1000 K in our uniformly-weighted
% C-band observations, which implies either that such an HII region is optically
% thin, with $\tau < 0.1$, or very small, with $ff<0.1$.  Given a beam FWHM of
% 0.3\arcsec, the implied upper limit size for an optically thick 1 mJy source is
% 1 mpc (200 AU).
% % see HIIregion_size.py
% 
% 
% It is possible that some of the compact sources are stars with no surrounding
% dense medium.  The radio continuum emission would then come from either a
% stellar wind \citep{Gaume1993a} or from nearby compact ionized globules, e.g.
% disks around nearby stars.  Particularly for the sources with near-infrared and
% X-ray associations (e20 and e22), the emission cannot come from a
% density-bounded hypercompact HII region and must instead come from some other
% mechanism.  However, because these sources do not exhibit the typical
% $S\propto\nu^{0.6}$ spectral energy distribution shape, having instead a
% significant excess in the 2-3 GHz band, they cannot be confirmed as
% spherical stellar wind emission sources from their spectral energy distributions.
% \citet{Reynolds1986a} offer explanations for how a non-spherical wind (a jet)
% could result in different spectral indices, but the shallowest they derive
% are $\alpha=-0.1$ corresponding to optically thin free-free emission, so even
% these models cannot explain the long-wavelength excess.
% 
% The diffuse continuum sources are assumed to be HII regions, as previously
% reported \citep{Gaume1993a,Mehringer1994a}.  There is a set of 4 ultracompact
% HII regions with similar size (0.02-0.07 pc): d3, e1, e6, and e7.  All are
% round and reasonably symmetric.  They exhibit some variations in surface
% brightness but are essentially circular.  In a classical view of an expanding
% spherical HII region, these sources are at early but very similar evolutionary
% stages.  Assuming that these all went through a hypercompact HII region phase,
% and assuming that all of the hypercompact and ultracompact HII regions within
% W51 go through the same phases at the same rate, the ultracompact HII phase
% must be 10-30\% as long as the hypercompact HII phase.  However, both of these
% assumptions are questionable.

%\subsection{Sources of illumination in the extended HII regions}
%There are many extended HII regions associated with the W51 cluster area.
%While some have obvious driving sources, the W51 Main region seems not to - the
%hypercompact HII regions are too small and not leaky enough.  If the
%hypercompact HII regions were illuminating the W51 Main, we would expect
%to see some traces of ionized features between e1/e2 and the lobe, or we would
%expect not to see the O-stars as hypercompact HII regions at all.
%Additionally, at the W51 Peak position, we have discovered multiple overlapping
%bubbles at different line-of-sight velocities with centers far from both the
%W51 IRS2 and W51e clusters.
%
%We therefore suggest that a significant fraction of the luminosity in the W51A
%region, and perhaps all from the W51 Main region, comes from a
%partially-embedded population of stars that is extremely confused with the
%luminous HII region it generates.
%
%
%\todo{Fix the naming scheme using \texttt{diffuse\_hii\_regions.reg}.  Create
%figures showing the different velocity structures.}


%However, given the shell-like structure of this loop around
%W51 Main, it seems likely that at least some of the shell-illuminating sources
%are near the W51 e1/e2 cluster.


%The O3 or O4 star W51d spectrally typed by \citet{Barbosa2008a} is probably
%illuminating the majority of the ionized gas in the IRS2 region.
% It is
% particularly impressive, though, since the star is shining through a layer of
% molecular gas that extincts the star.

% Can W51d be illuminating the W51 Main HII region?

\subsection{\formaldehyde emission features}


% \todo{Explain further.  Probably requires a figure.}
% \todo{MERGE THIS: (this is also a `TODO: explain more') from Roberto
% The W51e emission sources show signs of self-absorption around $v\sim55$ \kms,
% close in velocity to the nadir of the W51e2 spectrum.  A diffuse cloud is
% visible in absorption against the HII region toward the south at this same
% velocity, so it seems that the W51e cluster is either embedded within the
% $\sim55$ \kms cloud or behind it.  The cloud wraps around the W51e cluster at
% lower velocities, so it seems more likely they are embedded within it.  All of
% the W51e sources show absorption against the 68 \kms cloud, so they lie behind
% it.}

\formaldehyde \oneone and \twotwo are commonly observed in absorption
but rarely in emission \citep[e.g.][]{Mangum1993a,Araya2007b}.  The \twotwo
line has been observed in emission in the starburst M82 \citep{Mangum2008a},
behind the Orion nebula
\citep{Evans1975a,Kutner1976a,Batrla1983a,Johnston1983a,Bastien1985a,Wilson1989a},
in $\rho$ Ophiucus B
\citep{Loren1980a,Loren1983a,Martin-Pintado1983a,Wadiak1985a}, and in DR 21
\citep{Wilson1982a,Johnston1984a}.  However, the \twotwo line is more commonly
observed in absorption at large distances within the Galaxy.

We have detected 3 regions of \twotwo emission in the W51 region, all
corresponding to previously detected hot molecular cores
\citep{Zhang1997a,Shi2010a,Shi2010b,Goddi2015a}.  The bright
maser source W51e2 is partially surrounded by a `halo' of \formaldehyde \twotwo
emission to its northeast; the \hchii region itself shows only \twotwo
absorption because the continuum source is bright
(Figure \ref{fig:w51bridge22emispec} and Figure \ref{fig:w51mainemicontours}).
The \hchii region W51e8 exhibits extended \twotwo emission, including a
somewhat diffuse region stretching between W51e4 and W51e1.  Finally, in W51
IRS 2, there is extended \twotwo
emission stretching between W51d1 and W51d2, adjacent to the ammonia masers
observed by \citet{Zhang1995a} and more recently \citet{Goddi2015a}, and
aligned with the \citet{Zapata2010a} W51 North Disk (Figure
\ref{fig:w51northcore}, Section \ref{sec:northcore}).

\FigureTwo
{figures/irs2outflow/IRS2_core_on_cont22.png}
{figures/irs2outflow/IRS2_core_on_NACO_K.png}
{The W51 North core shown in \formaldehyde \twotwo emission in contours
overlaid on (a) the 14 GHz continuum and (b) the NACO K-band continuum image.
The contours are at 3 mJy/beam at velocities 56 (blue) to 60 (red) \kms at 0.5
\kms intervals.
}
{fig:w51northcore}{1}{3.5in}

In all cases, the emission is extended and spread smoothly over multiple
velocity channels (Figures 
\ref{fig:w51bridge22emispec}, \ref{fig:w51e2emispec}, \ref{fig:w51e1emispec}).
It is therefore not maser emission.

%There is not a 1-1 detection.  But my judgement on this has varied.
None of these detections have corresponding \oneone emission.  This
nondetection is likely because our brightness sensitivity at C-band is very
poor.  \todo{The upper limit is XXX, ruling out strong masers.}  Additionally,
the foreground cloud is a much stronger absorber in the \oneone line, so any
emission is likely to be obscured by foreground absorption.

%\todo{
%The physical conditions required to produce these extremely bright emission
%regions, with $T_B \gtrsim 500$ K (40 mJy) in the FWHM$\approx0.4$\arcsec (2000
%au) beams, can be explained either as thermalized hot gas or radiatively pumped
%emission.
%}

%\todo{
%The observed high brightness temperatures are similar to the extremely high
%temperatures reported by \citet{Zapata2010a}.  The excitation temperature must
%be $T\sim900$ K at the \formaldehyde \twotwo $\tau=1$ surface.  However, such a
%high temperature is very near the regime in which \formaldehyde would be
%collisionally dissociated.
%}

%\todo{
%A second possibility is that the \formaldehyde is radiatively pumped, resulting
%in a large non-collisionally-driven emission.  Vibrationally excited
%\formaldehyde has infrared
%transitions around 3.5 and 6 \um \citep{Al-Refaie2015a}, so radiative pumping
%requires a very strong radiation field with $T\gtrsim1000$ K.  Such a radiation
%field is consistent with the presence of high-mass young stars.  To avoid
%radiative dissociation, and especially in W51 North, to avoid creating a large
%\hii region, the radiation field must be cooler than $T < 10000$ K,
%implying that the proto-O-star \citep[as inferred from its luminosity and its
%kinematically-derived mass][]{Zapata2008a,Zapata2009a} is presently at a later
%spectral type.
%}

%The required column density for \twotwo
%to become optically thick depends on the velocity gradient and abundance; assuming
%$dV/dR = 1$ \kms \perpc and $X=10^{-9}$, the .

%these are the same...
% Our continuum measurements tighten the limits presented by \citet{Zapata2010a},
% with 5-sigma upper limits of 1 mJy at both C and Ku-band.

The W51e2, W51e8, and North cores both exhibit peak brightness temperatures
$T\sim30-60$ K.  W51e8 seems an excellent analog to the W51 North core, at
least in terms of its brightness temperature and size (see Section
\ref{sec:northcore}).  

% We have detected a weak
% extended continuum source in e8, with a peak $S_{15 GHZ} \approx 2$ mJy and
% $S_{5 GHz} \approx 1$ mJy (Figure \ref{fig:w51mainemicontours}c), both
% consistent with the limits on W51North, where confusion from IRS 2 may prevent
% a detection there.

%bridge is not real
% \subsubsection{An extended molecular structure (a `filament')}
% \label{sec:bridgefilament}
% There is a `bridge' structure connecting the e8 and e2 cores.  This structure
% was seen in \citet{Zhang1997a}, but was poorly resolved and could have been
% dismissed as an artifact of the dirty beam.  It was also detected by
% \citet{Tang2009a} with the BIMA interferometer, but the coarse beam meant that
% it was barely distinguishable from the molecular cores.  This bridge is fainter
% than either of the cores but clearly detected (Figures
% \ref{fig:w51bridge22emispec}, \ref{fig:w51mainemilabels}, and
% \ref{fig:w51mainemicontours}).

%It is also hot, $T_B\gtrsim300$ K at peak,
%and presumably very dense.

%The gas temperatures must either be driven by internal star formation, possibly
%a very early stage massive star like W51 North with no HII region yet formed,
%or the bridge is heated by the cluster of surrounding massive stars.  The
%thermal Jeans mass in this bridge is $M_J = 10.0 \left(\frac{T}{300
%K}\right)^{3/2} \left(\frac{n}{10^7 \percc}\right)^{-1/2}$ \msun, implying that
%any fragments will be very large or, more likely, fragmentation is prevented.

%This is not plausible: e2 is not a star, it is a minicluster
% The bridge could in principle have been created by the ejection of e2 from the
% e1 cluster.  However, the maser proper motions of \citep{Sato2014a} show that
% e2 and e1 have almost no motion relative to one another.  To achieve their
% current separation at a motion of 1 mas \peryr (25 \kms), which is an upper
% limit on their relative motion, they would have had to separate 5-10 kyr ago.
% The current data do not rule out this scenario, but neither do they strongly
% favor it.  It is more likely that this bridge feature traces the parent filament
% from which both the e1 and e2 clusters have formed.

% This bridge filament has dimensions $\sim0.2\times0.05$ pc.  Its implied mass
% is $M\approx100 (n/10^6 \percc) \msun$ assuming it is 0.05 pc along the line of
% sight, where a density $n\gtrsim10^6$ is implied by the detection of
% \formaldehyde \twotwo in emission.  It is about twice as large as the `compact
% ridge' in orion, though fainter \citep{Mangum1990a,Mangum1993b}.

\Figure
{{figures/spectra/emission/NorthCore_h2co22emisson_baselined}.png}
{Spectrum of the W51 North core in \ortho \twotwo.  The spectrum
is the average over a $2\arcsec\times1.5\arcsec$ elliptical aperture extracted
from the Briggs-weighted cube (about 14$\times$ the beam area).  The grey
shading shows the $1\sigma$
errorbars.
}
{fig:w51bridge22emispec}{0.5}{3.5in}

\Figure
{figures/contour_movie/e1e2_h2co22_emission_on_cont22_natural_v55to62peak_labeled.png}
{ Peak intensity contours (red) of the naturally weighted \formaldehyde \twotwo
emission from 55 to 62 \kms in the e1/e2 region superposed on the Ku-band
continuum image.  The contours are at 2,4,6 mJy/beam.  The green circles show
the apertures used for the spectra shown in Figures
\ref{fig:w51e2emispec} and
\ref{fig:w51e1emispec}.  The larger apertures around e8mol and e10mol are
e8mol\_ext and e10mol\_ext.
}{fig:w51mainemilabels}{1}{6.5in}

\FigureThreeAA
{figures/contour_movie/e1e2_h2co22_emission_on_cont22_natural_v56.0.png}
{figures/contour_movie/e1e2_h2co22_emission_on_cont22_natural_v57.0.png}
{figures/contour_movie/e1e2_h2co22_emission_on_cont22_natural_v59.5.png}
{Contours of the \formaldehyde \twotwo emission in the W51 Main region at 3
velocities superposed on the 15 GHz continuum map.  (a) shows the peak of the e2
core, where the center of the core is missed because it is in absorption
against the very bright continuum peak, (b) shows the overlap velocity (the
connecting region between the clumps is an imaging artifact),
and (c) shows the e8 core}
{fig:w51mainemicontours}{1}{2.1in}

\FigureThreeAA
{figures/spectra/emission/W51Ku_BD_h2co_v30to90_briggs0_contsub.image.fits_e2_a_K.png}
{figures/spectra/emission/W51Ku_BD_h2co_v30to90_briggs0_contsub.image.fits_e2_b_K.png}
{figures/spectra/emission/W51Ku_BD_h2co_v30to90_briggs0_contsub.image.fits_e2_c_K.png}
{Spectra of the \twotwo emission around W51e2 at three positions (a)
north/northwest (b) northeast (c) southeast of the e2 HCHII region.  See Figure
\ref{fig:w51mainemilabels} to see where these were extracted.}
{fig:w51e2emispec}{1}{2.25in}

\FigureTwoAA
{figures/spectra/emission/W51Ku_BD_h2co_v30to90_briggs0_contsub.image.fits_e10mol_K.png}
{figures/spectra/emission/W51Ku_BD_h2co_v30to90_briggs0_contsub.image.fits_e8mol_K.png}
{Spectra of the \twotwo emission around W51e1.  See Figure
\ref{fig:w51mainemilabels} to see where these were extracted.
}
{fig:w51e1emispec}{1}{3.5in}


\subsubsection{The W51 North core}
\label{sec:northcore}
\todo{Consider removing this subsection}
The W51 North core (Figure \ref{fig:w51northcore}) has been recognized as one
of the best candidates for a proto-O-star that has not yet reached the main
sequence \citep{Zapata2009a,Zapata2010a,Goddi2015a}.  

The core exhibits a velocity gradient, shifting its centroid position from
southeast to northwest then back to southeast from red to blue.  The direction
is consistent
with the position angle reported by \citet{Zapata2010a}, but the reversal is
not.  We suspect that the foreground cloud, which exhibits strong absorption
toward IRS 2 at 64 \kms, is absorbing some of the core's emission at that
velocity, causing the centroid to shift away from IRS 2.
%Our data also differ
%from \citet{Zapata2010a} in that each channel is centrally peaked, with no
%suggestion of an inner cavity.

% see northcore and centroid_north_core
We can estimate the mass of this core in a few ways.  A reasonable lower limit
on the mass is given by assuming it has a volume density $n\gtrsim10^6$ \percc,
which is required to see it in emission in \ortho \twotwo.  Its radius, as
meaured by fitting a 2D gaussian to its integrated intensity emission map, is
$\sigma=0.9\arcsec$ or 0.025 pc (FWHM=2.1\arcsec, 0.056 pc).  If we assume it
is spherically symmetric and gaussian in
all dimensions, the resulting mass is $M\gtrsim14\msun$.
We compute the \ortho \twotwo column density using \citet{Mangum2015a} equation
100, with a measured integrated intensity of 54.5 mJy \kms or 65.5 K \kms given
the $2\times1.5$\arcsec source area.  The inferred column of \ortho,
assuming LTE, is $\sim5\ee{15} - 5\ee{16}$ \persc, depending on the assumed
kinetic temperature and excitation temperature.  This \formaldehyde column
implies an absurdly large mass ($M\sim5\ee{3}$ \msun) for typical assumed abundances
($X_{\formaldehyde}\sim10^{-9}$), so we conclude that the \formaldehyde abundance
must be 2-3 orders of magnitude greater than in the molecular cloud, with
$X_{\formaldehyde}\sim3\ee{-7}$ providing a mass consistent with the density
lower-limit based mass.  For a reasonable range of `core' masses, $10
\msun<M<100 \msun$, the implied abundance is $3\ee{-7} > X_{\formaldehyde} >
5\ee{-8}$.

The W51e8 core is similar to W51 North in many respects, including size,
linewidth, and brightness.  However, W51e8 contains a hypercompact HII region,
while W51 North does not.  In W51 North, though, detection of a hypercompact
HII region may be prevented by imaging artifacts from the nearby bright IRS 2.
It is therefore plausible that W51 North contains a similar continuum source to
W51e8 ($S_\nu \sim 1$ mJy).

% \todo{This paragraph applies to the 56 to 61 \kms figure, which might
% not be representative.}
% There is a hint of rotation in our data.  Figure \ref{fig:w51northcentroids}
% shows the emission centroids determined by 2D gaussian fitting to the naturally
% weighted cube as a function of velocity.  Excepting the two velocity extrema,
% there is a clear red and blue side, with the redshifted side to the northwest
% and the blueshifted toward the southeast.  This gradient agrees with the
% 22\arcdeg position angle noted by \citet{Zapata2010a}.  However, our data are
% centrally peaked, with no suggestion of an inner cavity.  We also see the gradient
% over a smaller velocity range, from 56 to 61 \kms, while \citet{Zapata2010a}
% observed the gradient from 53 to 68 \kms.   At the higher velocities ($v>61
% \kms$), the image of W51 North appears to be affected by absorption from the
% foreground cloud that is seen in absorption against the IRS2 region.

% \Figure{figures/w51north_core_gaussfits.png}
% {The centroids of the W51 North core \ortho \twotwo emission as a function of 
% velocity from 56 to 63 \kms.  The positions are shown as RA/Dec offsets from
% 19h23m39.999s +14d31m05.41s, the centroid position of the mean over all
% velocities.}
% {fig:w51northcentroids}{1}{5in}

% \subsection{Explanation of the hot gas}
% The high observed temperatures could be an indication either of genuinely hot
% molecular gas or of radiative pumping of the \formaldehyde molecules.  In
% either case, an intense source of radiation must be present; the primary
% difference is whether the excitation is driven by photons or collisions with
% the local dust.
% 
% Collisionally excited gas at these high temperatures implies a high density as
% well, which in turn implies frequent high-energy collisions between molecules.
% It is likely that the high-end tail of these collisions will result in rapid
% dissociation of the molecules.  Since \formaldehyde and \ammonia are both
% relatively complex, with \ammonia forming slowly in cold gas in the absence of
% CO \todo{[citation needed]}, re-formation of the molecules would not provide a
% sufficient population for our observations.  We therefore favor radiative
% (infrared) pumping as the explanation for the high observed brightness.  The gas
% must still be warm, $T\gtrsim200$ K \citep{Henkel2013a}, but not quite as hot
% as we might naively infer.  \citet{Mangum1993a} determined that infrared
% pumping begins to be significant for \formaldehyde when temperatures approach
% $T\gtrsim150$ K (see their Appendix C).
% 
% Normally \ammonia metastable transitions are cited as good tracers of the gas
% temperature because their relative populations can only be set by collisions;
% there are no radiative transitions connecting the K-ladders.  However, high
% radiation temperatures can easily excite the higher vibrational levels of
% \ammonia, which can then decay into different K-ladders \citep[][gives an
% overview of the selection pseudo-rules]{Henkel2013a}.  Therefore, temperatures
% inferred from \ammonia may be inaccurate in these hot cores.


\subsection{The Lacy jet}
\label{sec:lacyjet}
\citet{Lacy2007a} reported the detection of very high velocity ionized gas
in the mid-infrared [Ne II] 12.8\um and S IV 10.5\um lines.  They observe the
gas at a velocity blueshifted about 100 \kms from the IRS2 ionized and molecular
gas velocity.  We have detected the same feature in the H77$\alpha$ RRL.
The RRL shows the same position-velocity structure as the infrared ionized
features.  No redshifted counterflow is detected (Figure \ref{fig:lacyjetslice}).

One minor but notable feature of the H77$\alpha$ flow is that it is very close
to the rest velocity of the He77$\alpha$ line.  If \citet{Lacy2007a} had not
already unambiguously identified the high velocity of this feature in two
lines, we would have assumed it to be slightly redshifted Helium.

\FigureTwo
{figures/jetpv/H77a_cutout_1.png}
{figures/jetpv/w51.neii.square.png}
{Position-velocity slices through (a) the H77$\alpha$ cube and (b) the [Ne II]
cube from \citet{Lacy2007a} tracking the approximate path of the Lacy jet.  Emission is black,
absorption (which in both images is due to reduction artifacts rather than true
physical absorption) is white.  The vertical streaks are continuum sources.
The
abscissa shows the offset position along the slice from J2000 19:23:40.464
+14:31:07.9536 to 19:23:39.4176 +14:31:04.7172, position angle 258\arcdeg.
The blueshifted lobe is evident at -50 \kms in both cubes, and neither show a
redshifted counterpart.  The H77$\alpha$ data reveal that there is truly no
counterpart, ruling out the possibility that it was simply too extincted to be
observed in the infrared.}
{fig:lacyjetslice}{1}{3.5in}

This outflow is close to W51 North, which is a likely driver of the outflow based
on the highly excited \ammonia around it \citep{Henkel2013a,Goddi2015a}.  However,
the accreting O-star W51 IRS 2E is also very close to the outflow and could plausibly
be the driver \citep{Barbosa2008a}.  IRS 2E is the brightest X-ray source in
W51 with a harder X-ray spectrum than other sources in the region, and it emits
most of its energy in iron fluorescence lines, indicating that there is a very
strong interaction with dense material around it \citep{Townsley2014a}.
Independent of its source, though, this flow is unique in exhibiting a high
velocity ($v\sim100 \kms$) over a 0.1 pc extent with no obvious gradient.
Indeed, this lack of a velocity gradient suggests that the flow is launched
at a high velocity and not decelerated by interaction with its environment.


\FigureTwo
{figures/irs2outflow/IRS2_h77a_on_cont22.png} % outflow_overlay.py
{figures/irs2outflow/IRS2_h77a_on_NACO_K.png} % irs_nir.py
{The `Lacy Jet' as seen in H77$\alpha$ overlaid on (a) the 14 GHz continuum
image and (b) the NACO K-band image \citep{Barbosa2008a}.  Contours are at 
0.1, 0.2, 0.3, 0.4, 0.5, 0.6, and 0.7 Jy \kms, integrated over the velocity
range -60 to -16 \kms.}
{fig:lacyjetoverlay}{1}{3.5in}

\subsection{The d4 variable source \& \hh region}
\label{sec:d4}
The d4 pair (d4e and d4w) exhibit $\sim$20-year timescale variability.  d4e has
brightened from $<0.6$ mJy ($3-\sigma$ upper limit) to $>1.0\pm0.06$ mJy.
There is a region of \hh emission, MHO 2419, coincident with their position
\citep[Figure \ref{fig:d4h2}][]{Hodapp2002a,Froebrich2011a}.

Similar variability in radio sources has been observed in nearby star forming
regions \citep{Liu2014c,Forbrich2013a} and associated with variability in
accretion or jet related processes.  It is possible that the variability
observed here is similarly related to jet formation, and it is even plausible
that the d4 pair represents two opposing jets forming the base of a larger
outflow structure.

However, the \hh and [Fe II] emission are clearly associated with bow shocks
based on morphological and velocity structure in the \citet{Hodapp2002a}
spectra, and these bow shocks point back to an origin in IRS 2.  The high
velocity [Fe II] features ($\pm150$ \kms) are a clear indication of a bow
shock.  It is therefore possible that the d4 sources are just emission features
within the bow shock.  Since such features are not observed in the nearby Orion
KL outflow, where the bow shocks are orders of magnitude fainter (Forbrich et
al, in prep), such an association implies that this outflow is much more
powerful than any seen in nearby star forming regions.

\Figure{figures/d4_h2_overlay.png}
{A UWISH2 \hh image with Ku-band contours of sources d4e and d4w overlaid at
0.15, 0.30, 0.45, 0.60, 0.75, and 0.90 mJy/beam.  The box shows the slit
position used in \citet{Hodapp2002a}.  The X's mark positions of MOXC X-ray
sources \citep{Townsley2014a}.
}
{fig:d4h2}{1}{6.5in}

\subsection{The velocity dispersion in the W51e cluster}
\label{sec:vdisp}
We detect H77$\alpha$ toward six of the eight hyper/ultra compact \hii regions
in the W51 Main (W51e) cluster; four of these six are firm detections and two
are weak.  We also measure a velocity from the \formaldehyde emission toward
e8.  Although the molecular and ionized emission trace dramatically different
processes, we treat the line-of-sight velocities of both tracers as though they
are representative of the stellar velocity for this analysis.  With such
compact \hii regions, the ionized gas is unlikely to be significantly shifted
from the stellar velocity.  The core emission is likely to be dominated by the
densest component nearest the central accreting source.

The resulting 1D velocity dispersion is $\sigma=2.0$ \kms if we exclude
the uncertain sources e9
and e10; with e9 and e10 included the dispersion increases to 4.2 \kms, which
suggests that the uncertainty on those two sources renders their velocity
measurements unreliable.  This velocity dispersion is confined to
$r<5.4\arcsec$ or $r<0.13$ pc.  Assuming the stars are virialized (which is
probably not a good assumption), the implied mass is 350-1600 \msun, or an \hh
number density $6\ee{5} < n(\hh) < 2\ee{7}$ \percc.

The e1 subcluster is more compact, with $r=2.9$ \arcsec (0.07 pc), and its
velocity dispersion is also $\sigma=2.0$ \kms.  It is more symmetric and a virial
assumption may be more accurate here.  Its implied density if it is gas-dominated
is $n(\hh)\sim2\ee7$ \percc, but assuming all 7 HCHII regions represent OB
stars and the IMF is populated by a standard \citet{Kroupa2001a} mass function,
the cluster should be star-dominated with a mass $M_{cluster}=700\pm250$ \msun,
which implies a stellar density $n_* \sim 4 n_{gas}$.  Given that there are 
hypercompact HII regions, implying gas densities near the stars $n>>10^6$ \percc,
this is a region where stellar and gas dynamics are closely intertwined.

Given the mass of W51 Main and its $\sim10\kms$ escape velocity, all of these
(proto)stars are clearly bound to the overall cluster gas. 

\FigureTwo
{figures/spectra/h77/e1_h77a_fit.png}
{figures/spectra/h77/e2_h77a_fit.png}
{Fitted H77$\alpha$ spectra of W51e1 and W51e2.  These velocities are used in
conjunction with the \formaldehyde emission fits to determine the velocity
dispersion in Section \ref{sec:vdisp}.
The red curve shows the fit.  The lower spectrum shows the fit residual, with
the dashed line indicating the zero level.  The baselines are imperfect, but we
have made no effort to subtract them during the spectral fitting, instead
assuming that the image-based continuum subtraction is accurate.
}
{fig:h77afits}{1}{3in}

% \subsubsection{The velocity dispersion in the gas cores}
% The molecular gas line width is $\sigma>2.1$ \kms in all of the emission cores
% (Table \ref{tab:emission22}).
% Since the cores
% each have higher internal velocity dispersion than the surrounding gas, they
% must have significant internal motions, either rotation, turbulence, or
% gravitational collapse.
% Because the e2 core is resolved and has velocity
% dispersion greater than the observed velocity gradient, internal turbulence or
% gravitational infall are the only viable explanations for the observed velocity
% dispersion.



% \section{The distributed population of HCHII regions}
% We detect an additional XX point sources in the field.

\subsection{The future evolution of the W51 clusters and molecular cloud}
The W51 clusters are likely to reach a very high star formation efficiency,
since the escape velocity is greater than the sound speed in ionized gas
\citep{Ginsburg2012a,Bressert2012a}.  If they really are forming one or more
young massive cluster, though, the gas within $\sim 1$ pc must be evacuated or
exhausted to achieve a final appearance like that of NGC 3603 or Trumpler 14.

Because IRS 2 is already seen in the near-infrared, it is clear that
the cluster leaks photons.  Since long-wavelength photons are not trapped,
radiation pressure is probably not going to be efficient; the FIR optical depth
term of \citet{Murray2010a} is very small.  Radiative trapping is only
relevant within the $<2\arcsec$ ultracompact HII regions, in which the FIR optical
depth exceeds unity.

The ongoing accretion onto the source IRS2E and probable ongoing accretion onto
the hot cores and HCHII regions throughout these clusters implies that ionizing
radiation is, at present, not halting star formation.  The presence of
accreting stars within large HII regions further confirms that ionizing
radiation and ionized gas pressure is inadequate to stop star formation in the
clusters.

We therefore anticipate that the clusters will continue to consume the dense
gas they presently contain until it is either exhausted or some other feedback
mechanism takes over \citep{Ginsburg2012a,Bressert2012a,Ginsburg2015a}.  The
exhaustion of \emph{local} gas, out to $\sim1-2$ pc, is a necessary
precondition for the formation of a bound cluster \citep{Kruijssen2012a}.
Other feedback mechanisms may be stellar winds, direct radiation pressure
(assuming a single scattering event per photon), or supernovae.

The present stellar mass as inferred from the infrared luminosity (Section
\ref{sec:stellarmass}) is a lower limit on the final mass of the cluster.  The
final mass of the clusters will be at least $\sim7\ee{3} + 4\ee{4} \eta$ \msun,
where $4\ee{4}$ \msun is the approximate dense gas mass within $\sim2$ pc and
$\eta$ is a star formation efficiency, likely to be at least 10\% in the dense
gas, so the stellar $M_{final} \gtrsim 1.1\ee{4}$ \msun.

Since the free-fall time is very short within these clusters ($t_{ff} < 0.5$
Myr out to 2.5 pc containing 5\ee{4} \msun of gas), supernovae will not affect
their formation.  However, once the clusters have exhausted gas locally, there
is both plenty of time and a relatively free path from the clusters to the rest
of the molecular cloud.  The binding energy of a $10^6$ \msun, $r=50$ pc 
molecular cloud is only about $10^{51}$ ergs, so a single supernova perfectly
coupled to the gas could unbind it all.  The $\sim20$ O-stars presently formed,
plus at least that many expected to form, will provide more than enough energy
to unbind the whole GMC.

% In order to match star formation relations, the overall star formation
% efficiency of the cloud must be limited to \todo{10\% ?? [citation needed]}.
% If the cluster progresses to 100\% efficiency, as we predict, the rest of the
% cloud must be limited to XX.

\subsection{The distributed and clustered O-star populations}
The IRS2 and e1/e2 clusters each contain at least 5-10 OB stars, but within 2-3
parsecs of these clusters, there appear to be 10-20 additional OB stars.  Some
of these, like e20-e22, are in their own subclusters, but together they form a
distributed population in the same proto-cluster region.

Among these stars, some are almost certainly still accreting - the unresolved
hypercompact HII regions, at least, must be surrounded by high-density neutral
material, since otherwise they would have expanded into diffuse HII regions.
Others, e.g., e20, may have completed accreting and transitioned into an
exposed phase during which the ionized material in their immediate vicinity is
mass lost from the stars themselves.  The coexistence of both of these phases
implies that there must be some age spread within the distributed population.

%If the age spread is large compared to the age of an O-star, it is plausible
%that the distributed population may produce one or more supernovae before the
%natal clusters have completed their formation.  While this mechanism would
%provide a way to remove the gas and end the star and cluster formation process
%in the cloud, it is intrinsically unlikely since even the shortest stellar
%lifetimes are an order of magnitude longer than the free fall collapse time in
%gas at cluster densities.

If any of the observed subclusters merge, e.g., if the e20 and e1/e2 clusters
merge, the resulting cluster will have an age spread corresponding to the present
age spread.  As a lower limit to the age of the e20 cluster, if we assume it is
responsible for driving the nearest part of the W51 Main shell at a separation
of 0.35 parsecs with a D-type ionization front at $v\sim2$ \kms, we get
$t_{e20} > 0.17$ Myr.  This lower limit is comparable to the gas free fall
time in the cluster and is still consistent with the most accurate stellar age
spread estimates available from photometry of young clusters
\citep[e.g.,][]{Kudryavtseva2012a}.

The coexistence of distributed and clustered populations over a fairly compact
region carries important implications for stellar feedback.  Because these
populations are nearly coeval and spatially coincident, the feedback from the
OB star population onto the gas should not be treated as point-symmetric as if
it were injected from a single cluster.  The distributed population will render
the parent cloud more holy (swiss-cheese-like) than would be implied by a
smooth or even a very turbulent molecular cloud.  These holes would likely
provide networks of low-density gaps through which radiation can escape,
reducing the overall impact of radiative feedback on the cloud.

\section{Conclusions}
We present deep, high-resolution radio continuum images observed in many
low-frequency bands from the Very Large Array.  These images have yielded the
first detection of the stellar cluster that is responsible for most of the
observed radio and infrared luminosity of W51A.  The mass and mass distribution
of these stars is not yet characterized because many of the stars are hidden
from near-infrared view by thick absorbing filaments.

Using these continuum data and \formaldehyde emission and absorption line
information, we have measured the velocity dispersion of the W51 e1/e2
protocluster $\sigma_v=2.0$ \kms.  Although this cluster consists of
hypercompact and ultracompact HII regions, indicating that is is extremely
gas-rich, the velocity dispersion implies a stellar mass that is $\sim4\times$
greater than the gas mass, assuming that the stars are virialized.

The SEDs of the point sources detected throughout the W51 A region have
peculiar shapes.  Most of them exhibit an excess at low frequencies ($<5$ GHz)
that renders their overall SED shape inconsistent with either an optically
thick or thin free-free spectrum.  W51e2 is an exception, with an SED
consistent with optically thick free-free emission from a constant-size source
from 2 to 15 GHz.  We argue that the sources are mostly hypercompact HII
regions, but we cannot explain the widespread low-frequency excesses.

A few sources cannot be hypercompact HII regions because they have infrared
K-band detections.  These sources may be O-stars emitting in the radio from
ionized winds.  These exposed O-stars are proposed as the primary drivers of
the overall infrared and radio luminosity observed from the W51 Main emission
peak, as the hypercompact HII regions are unlikely to allow ionizing continuum
photons to escape.

We detected radio recombination line (H77$\alpha$) emission throughout W51.
The \citet{Lacy2007a} high-velocity ionized jet is clearly detected, but
remains mysterious as there is still no direct information about what drives
it.  The RRLs trace a number of overlapping bubbles toward the W51 Main
emission peak and show that the IRS 2 region has excavated a cavity along our
line of sight.

We have detected the three massive cores from \citet{Zhang1997a} in
\formaldehyde \twotwo emission.  For W51 North, we evaluated the mass and
derived a lower limit $M>15 \msun$, though it is likely to be significantly
higher.  The \formaldehyde abundance in these cores is enhanced compared to the
surrounding molecular cloud by at least 100$\times$, with $X_{\formaldehyde}
\sim 10^{-7}$, similar to the Orion BN/KL hot core.  In the W51e8 molecular
core, we have found a compact continuum source, indicating unambiguously that
these cores are forming massive stars.

The \formaldehyde \twotwo 14.488 GHz transition is a good tracer of early-stage
very massive star formation.  It is apparently enhanced in abundance by 2-3
orders of magnitude over ISM values in high-mass protostellar cores, allowing
it to be detected in emission at long wavelengths.  It will be a powerful tool
for studying the earliest stagest of high-mass star formation throughout the
Galaxy with the Next Generation VLA or a Square Kilometer Array if the
high-frequency end includes this transition.  Ongoing JVLA surveys (e.g.,
KuGARS, PI Thompson, \url{http://library.nrao.edu/proposals/catalog/10267}) may
detect a significant additional population of proto-O-stars.




\textbf{Acknowledgements}:

\textbf{Code Packages Used}:

\begin{itemize}
    \item pyradex \url{https://github.com/adamginsburg/pyradex}
    \item myRadex \url{https://github.com/fjdu/myRadex}
    \item pyspeckit \url{http://pyspeckit.bitbucket.org}
    \item aplpy \url{https://aplpy.github.io/}
    \item wcsaxes \url{http://wcsaxes.readthedocs.org}
    \item spectral cube \url{http://spectral-cube.readthedocs.org}
    \item pvextractor \url{http://pvextractor.readthedocs.org/}
\end{itemize}

\ifstandalone
\bibliographystyle{apj_w_etal}  % or "siam", or "alpha", or "abbrv"
%\bibliography{thesis}      % bib database file refs.bib
\bibliography{bibdesk}      % bib database file refs.bib
\fi


\appendix
\section{Figures labeling point sources and diffuse regions}
We include additional versions of Figure \ref{fig:coverview} with labels for
the compact (Figure \ref{fig:coverview_pointsrcs}) and extended (Figure
\ref{fig:coverview_diffuse}) HII regions.

\Figure{figures/diffuse/W51_C_overview_pointsourcelabels.pdf}
{The C-band image of the W51A region with point sources labeled.  The blue
sources are previously known; the green are new additions in this paper.}
{fig:coverview_pointsrcs}{0.9}{6.5in}

\Figure{figures/diffuse/W51_C_overview_diffusehiiregionlabels.pdf}
{The C-band image of the W51A region with diffuse HII regions labeled.  The labels
are above the circles except where arrows are used.}
{fig:coverview_diffuse}{0.9}{6.5in}

%\onecolumn
%\begin{table*}[htp]
\caption{Continuum Point Sources}
\begin{tabular}{ccccccc}
\label{tab:contsrcs_full}
Object & Epoch & Obs. Date & Peak $S_{\nu}$ & Peak - Background & RMS & Frequency \\
 &  &  & $\mathrm{mJy\,beam^{-1}}$ & $\mathrm{mJy\,beam^{-1}}$ & $\mathrm{mJy\,beam^{-1}}$ & $\mathrm{GHz}$ \\
\hline
d2 & 2 & 2012-10-16 & 4 & 6 & 2 & 2.5 \\
d2 & 2 & 2012-10-16 & 8.5 & 9.1 & 0.8 & 3.5 \\
d2 & 1 & 1992-10-25 & 9.1 & 9.4 & 0.1 & 4.9 \\
d2 & 2 & 2012-10-16 & 6.9 & 8 & 0.5 & 4.9 \\
d2 & 3 & 2014-04-19 & 3.7 & 4 & 0.05 & 4.9 \\
d2 & 2 & 2012-10-16 & 4.3 & 4.6 & 0.05 & 5.9 \\
d2 & 3 & 2014-04-19 & 8 & 10 & 1 & 5.9 \\
d2 & 1 & 1992-10-25 & 3.6 & 3.6 & 0.2 & 8.4 \\
d2 & 2 & 2013-03-02 & 4.2 & 4.4 & 0.04 & 12.6 \\
d2 & 2 & 2013-03-02 & 3 & 6 & 1 & 14.1 \\
d2 & 1 & 1993-03-16 & 3.3 & 3.6 & 0.08 & 22.5 \\
d2 & 2 & 2012-05-31 & 8.8 & 9.2 & 0.09 & 25.0 \\
d2 & 2 & 2012-06-21 & 2.8 & 2.5 & 0.06 & 27.0 \\
d2 & 2 & 2012-08-07 & 1.6 & 1.7 & 0.2 & 29.0 \\
d2 & 2 & 2012-06-21 & 9 & 12 & 1 & 33.0 \\
d2 & 2 & 2012-08-07 & 2.3 & 2.6 & 0.06 & 36.0 \\
d3-diffuse & 2 & 2012-10-16 & 1 & 5 & 2 & 2.5 \\
d3-diffuse & 2 & 2012-10-16 & 0.5 & 1.9 & 0.8 & 3.5 \\
d3-diffuse & 3 & 2014-04-19 & 0.79 & 1.1 & 0.05 & 4.9 \\
d3-diffuse & 2 & 2012-10-16 & -0.4 & 0.7 & 0.5 & 4.9 \\
d3-diffuse & 1 & 1992-10-25 & 1 & 1.3 & 0.1 & 4.9 \\
d3-diffuse & 2 & 2012-10-16 & 0.62 & 0.75 & 0.05 & 5.9 \\
d3-diffuse & 3 & 2014-04-19 & 1 & 3 & 1 & 5.9 \\
d3-diffuse & 1 & 1992-10-25 & -0.2 & 0.2 & 0.2 & 8.4 \\
d3-diffuse & 2 & 2013-03-02 & 0.7 & 0.91 & 0.04 & 12.6 \\
d3-diffuse & 2 & 2013-03-02 & - & - & 1 & 14.1 \\
d3-diffuse & 1 & 1993-03-16 & 0.85 & 1 & 0.08 & 22.5 \\
d3-diffuse & 2 & 2012-05-31 & 0.95 & 1.2 & 0.09 & 25.0 \\
d3-diffuse & 2 & 2012-06-21 & 0.19 & 0.16 & 0.06 & 27.0 \\
d3-diffuse & 2 & 2012-08-07 & 1.1 & 1.3 & 0.2 & 29.0 \\
d3-diffuse & 2 & 2012-06-21 & - & - & 1 & 33.0 \\
d3-diffuse & 2 & 2012-08-07 & 0.91 & 1.2 & 0.06 & 36.0 \\
d4e & 2 & 2012-10-16 & 2 & 2 & 2 & 2.5 \\
d4e & 2 & 2012-10-16 & 0.1 & 1.1 & 0.8 & 3.5 \\
d4e & 3 & 2014-04-19 & 0.14 & 0.3 & 0.05 & 4.9 \\
d4e & 1 & 1992-10-25 & 0.3 & 0.6 & 0.1 & 4.9 \\
d4e & 2 & 2012-10-16 & -0 & 1 & 0.5 & 4.9 \\
d4e & 3 & 2014-04-19 & -0 & 1 & 1 & 5.9 \\
d4e & 2 & 2012-10-16 & 0.21 & 0.29 & 0.05 & 5.9 \\
d4e & 1 & 1992-10-25 & 0.7 & 1 & 0.2 & 8.4 \\
d4e & 2 & 2013-03-02 & 0.16 & 0.27 & 0.04 & 12.6 \\
d4e & 2 & 2013-03-02 & 1 & 1 & 1 & 14.1 \\
d4e & 1 & 1993-03-16 & 0.24 & 0.37 & 0.08 & 22.5 \\
d4e & 2 & 2012-05-31 & 0.46 & 0.61 & 0.09 & 25.0 \\
d4e & 2 & 2012-06-21 & 0.71 & 0.5 & 0.06 & 27.0 \\
d4e & 2 & 2012-08-07 & 0.4 & 0.9 & 0.2 & 29.0 \\
d4e & 2 & 2012-06-21 & 1 & 1 & 1 & 33.0 \\
d4e & 2 & 2012-08-07 & 0.22 & 0.7 & 0.06 & 36.0 \\
d4w & 2 & 2012-10-16 & 1 & 2 & 2 & 2.5 \\
d4w & 2 & 2012-10-16 & 1.8 & 1.2 & 0.8 & 3.5 \\
d4w & 2 & 2012-10-16 & 0.7 & 0.7 & 0.5 & 4.9 \\
d4w & 1 & 1992-10-25 & 1.6 & 1.1 & 0.1 & 4.9 \\
d4w & 3 & 2014-04-19 & 0.43 & 0.83 & 0.05 & 4.9 \\
d4w & 2 & 2012-10-16 & 0.38 & 0.52 & 0.05 & 5.9 \\
d4w & 3 & 2014-04-19 & 2 & 2 & 1 & 5.9 \\
d4w & 1 & 1992-10-25 & 2.6 & 1.3 & 0.2 & 8.4 \\
d4w & 2 & 2013-03-02 & 0.33 & 0.58 & 0.04 & 12.6 \\
d4w & 2 & 2013-03-02 & 0 & 1 & 1 & 14.1 \\
d4w & 1 & 1993-03-16 & 0.59 & 0.74 & 0.08 & 22.5 \\
d4w & 2 & 2012-05-31 & 1 & 0.89 & 0.09 & 25.0 \\
d4w & 2 & 2012-06-21 & 1.8 & 0.45 & 0.06 & 27.0 \\
d4w & 2 & 2012-08-07 & 0.9 & 1.4 & 0.2 & 29.0 \\
d4w & 2 & 2012-06-21 & 1 & 1 & 1 & 33.0 \\
d4w & 2 & 2012-08-07 & 0.77 & 0.81 & 0.06 & 36.0 \\
d5 & 2 & 2012-10-16 & - & - & 2 & 2.5 \\
d5 & 2 & 2012-10-16 & - & - & 0.8 & 3.5 \\
d5 & 1 & 1992-10-25 & 0.4 & 0.3 & 0.1 & 4.9 \\
d5 & 3 & 2014-04-19 & 0.03 & 0.07 & 0.05 & 4.9 \\
d5 & 2 & 2012-10-16 & - & - & 0.5 & 4.9 \\
d5 & 3 & 2014-04-19 & - & - & 1 & 5.9 \\
d5 & 2 & 2012-10-16 & 0.03 & 0.08 & 0.05 & 5.9 \\
d5 & 1 & 1992-10-25 & 0.2 & 0.2 & 0.2 & 8.4 \\
d5 & 2 & 2013-03-02 & 0.04 & 0.06 & 0.04 & 12.6 \\
d5 & 2 & 2013-03-02 & - & - & 1 & 14.1 \\
d5 & 1 & 1993-03-16 & 0.18 & 0.12 & 0.08 & 22.5 \\
d5 & 2 & 2012-05-31 & 0.12 & 0.16 & 0.09 & 25.0 \\
d5 & 2 & 2012-06-21 & 0.43 & 0.24 & 0.06 & 27.0 \\
d5 & 2 & 2012-08-07 & 0.4 & 0.2 & 0.2 & 29.0 \\
d5 & 2 & 2012-06-21 & - & - & 1 & 33.0 \\
d5 & 2 & 2012-08-07 & 0.07 & 0.17 & 0.06 & 36.0 \\
d6 & 2 & 2012-10-16 & 0 & 0 & 2 & 2.5 \\
d6 & 2 & 2012-10-16 & 0 & 0 & 0.8 & 3.5 \\
d6 & 3 & 2014-04-19 & 0.43 & 0.48 & 0.05 & 4.9 \\
d6 & 1 & 1992-10-25 & 1.1 & 1 & 0.1 & 4.9 \\
d6 & 2 & 2012-10-16 & 1.2 & 1.2 & 0.5 & 4.9 \\
d6 & 2 & 2012-10-16 & 0.52 & 0.47 & 0.05 & 5.9 \\
d6 & 3 & 2014-04-19 & 0 & 0 & 1 & 5.9 \\
d6 & 1 & 1992-10-25 & 1.2 & 0.6 & 0.2 & 8.4 \\
d6 & 2 & 2013-03-02 & 0.37 & 0.5 & 0.04 & 12.6 \\
d6 & 2 & 2013-03-02 & 2 & 2 & 1 & 14.1 \\
d6 & 1 & 1993-03-16 & 0.51 & 0.34 & 0.08 & 22.5 \\
d6 & 2 & 2012-05-31 & 1.1 & 1.1 & 0.09 & 25.0 \\
d6 & 2 & 2012-06-21 & 0.81 & 0.3 & 0.06 & 27.0 \\
d6 & 2 & 2012-08-07 & 0.8 & 0.7 & 0.2 & 29.0 \\
d6 & 2 & 2012-06-21 & -0 & 1 & 1 & 33.0 \\
d6 & 2 & 2012-08-07 & 0.47 & 0.49 & 0.06 & 36.0 \\
d7 & 2 & 2012-10-16 & - & - & 2 & 2.5 \\
d7 & 2 & 2012-10-16 & - & - & 0.8 & 3.5 \\
d7 & 1 & 1992-10-25 & 0.1 & 0.3 & 0.1 & 4.9 \\
d7 & 2 & 2012-10-16 & - & - & 0.5 & 4.9 \\
d7 & 3 & 2014-04-19 & 0.07 & 0.22 & 0.05 & 4.9 \\
d7 & 2 & 2012-10-16 & 0.13 & 0.14 & 0.05 & 5.9 \\
d7 & 3 & 2014-04-19 & - & - & 1 & 5.9 \\
d7 & 1 & 1992-10-25 & 0.8 & 0.6 & 0.2 & 8.4 \\
d7 & 2 & 2013-03-02 & 0.11 & 0.17 & 0.04 & 12.6 \\
d7 & 2 & 2013-03-02 & -1 & 2 & 1 & 14.1 \\
d7 & 1 & 1993-03-16 & 0.19 & 0.19 & 0.08 & 22.5 \\
d7 & 2 & 2012-05-31 & 0.2 & 0.26 & 0.09 & 25.0 \\
d7 & 2 & 2012-06-21 & 0.5 & 0.26 & 0.06 & 27.0 \\
d7 & 2 & 2012-08-07 & 0.1 & 0.5 & 0.2 & 29.0 \\
d7 & 2 & 2012-06-21 & -0 & 1 & 1 & 33.0 \\
d7 & 2 & 2012-08-07 & 0.15 & 0.29 & 0.06 & 36.0 \\
e1 & 2 & 2012-10-16 & -2 & 2 & 2 & 2.5 \\
e1 & 2 & 2012-10-16 & 0.1 & 0.8 & 0.8 & 3.5 \\
e1 & 2 & 2012-10-16 & -0 & 0.7 & 0.5 & 4.9 \\
e1 & 3 & 2014-04-19 & 0.4 & 0.6 & 0.05 & 4.9 \\
e1 & 1 & 1992-10-25 & 0.8 & 0.5 & 0.1 & 4.9 \\
e1 & 3 & 2014-04-19 & -0 & 2 & 1 & 5.9 \\
e1 & 2 & 2012-10-16 & 0.26 & 0.38 & 0.05 & 5.9 \\
e1 & 1 & 1992-10-25 & 1.9 & 1 & 0.2 & 8.4 \\
e1 & 2 & 2013-03-02 & 0.29 & 0.45 & 0.04 & 12.6 \\
e1 & 2 & 2013-03-02 & 1 & 2 & 1 & 14.1 \\
e1 & 1 & 1993-03-16 & -0.09 & 0.48 & 0.08 & 22.5 \\
e1 & 2 & 2012-05-31 & 0.46 & 0.38 & 0.09 & 25.0 \\
e1 & 2 & 2012-06-21 & 1.3 & 0.31 & 0.06 & 27.0 \\
e1 & 2 & 2012-08-07 & 0.5 & 1.2 & 0.2 & 29.0 \\
e1 & 2 & 2012-06-21 & -1 & 1 & 1 & 33.0 \\
e1 & 2 & 2012-08-07 & 0.52 & 0.74 & 0.06 & 36.0 \\
e10 & 2 & 2012-10-16 & -0 & 2 & 2 & 2.5 \\
e10 & 2 & 2012-10-16 & 1.4 & 1.6 & 0.8 & 3.5 \\
e10 & 2 & 2012-10-16 & 0.7 & 1.5 & 0.5 & 4.9 \\
e10 & 1 & 1992-10-25 & 0.1 & 0.5 & 0.1 & 4.9 \\
e10 & 3 & 2014-04-19 & 0.01 & 0.08 & 0.05 & 4.9 \\
e10 & 3 & 2014-04-19 & 1 & 2 & 1 & 5.9 \\
e10 & 2 & 2012-10-16 & 0.03 & 0.08 & 0.05 & 5.9 \\
e10 & 1 & 1992-10-25 & -0 & 0.5 & 0.2 & 8.4 \\
e10 & 2 & 2013-03-02 & -0.04 & 0.1 & 0.04 & 12.6 \\
e10 & 2 & 2013-03-02 & 1 & 3 & 1 & 14.1 \\
e10 & 1 & 1993-03-16 & 0.01 & 0.1 & 0.08 & 22.5 \\
e10 & 2 & 2012-05-31 & 0.3 & 0.54 & 0.09 & 25.0 \\
e10 & 2 & 2012-06-21 & 0.1 & 0.13 & 0.06 & 27.0 \\
e10 & 2 & 2012-08-07 & 0.3 & 0.3 & 0.2 & 29.0 \\
e10 & 2 & 2012-06-21 & 1 & 2 & 1 & 33.0 \\
e10 & 2 & 2012-08-07 & 0.04 & 0.21 & 0.06 & 36.0 \\
e11 & 2 & 2012-10-16 & 2 & 5 & 2 & 2.5 \\
e11 & 2 & 2012-10-16 & 0.9 & 1.4 & 0.8 & 3.5 \\
e11 & 1 & 1992-10-25 & 0.4 & 0.5 & 0.1 & 4.9 \\
e11 & 3 & 2014-04-19 & 0.29 & 0.32 & 0.05 & 4.9 \\
e11 & 2 & 2012-10-16 & - & - & 0.5 & 4.9 \\
e11 & 3 & 2014-04-19 & 5 & 6 & 1 & 5.9 \\
e11 & 2 & 2012-10-16 & 0.25 & 0.26 & 0.05 & 5.9 \\
e11 & 1 & 1992-10-25 & 0.7 & 0.8 & 0.2 & 8.4 \\
e11 & 2 & 2013-03-02 & 0.17 & 0.24 & 0.04 & 12.6 \\
e11 & 2 & 2013-03-02 & - & - & 1 & 14.1 \\
e11 & 1 & 1993-03-16 & 0.35 & 0.33 & 0.08 & 22.5 \\
e11 & 2 & 2012-05-31 & 0.31 & 0.39 & 0.09 & 25.0 \\
e11 & 2 & 2012-06-21 & 1 & 0.71 & 0.06 & 27.0 \\
e11 & 2 & 2012-08-07 & 1 & 1.1 & 0.2 & 29.0 \\
e11 & 2 & 2012-06-21 & - & - & 1 & 33.0 \\
e11 & 2 & 2012-08-07 & 0.42 & 0.6 & 0.06 & 36.0 \\
e12 & 2 & 2012-10-16 & 3 & 5 & 2 & 2.5 \\
e12 & 2 & 2012-10-16 & 4.7 & 5.6 & 0.8 & 3.5 \\
e12 & 1 & 1992-10-25 & 2.3 & 2.6 & 0.1 & 4.9 \\
e12 & 2 & 2012-10-16 & 1.2 & 2.3 & 0.5 & 4.9 \\
e12 & 3 & 2014-04-19 & 0.81 & 1.1 & 0.05 & 4.9 \\
e12 & 2 & 2012-10-16 & 0.77 & 0.96 & 0.05 & 5.9 \\
e12 & 3 & 2014-04-19 & 4 & 5 & 1 & 5.9 \\
e12 & 1 & 1992-10-25 & 1.5 & 1.1 & 0.2 & 8.4 \\
e12 & 2 & 2013-03-02 & 0.68 & 0.88 & 0.04 & 12.6 \\
e12 & 2 & 2013-03-02 & 4 & 7 & 1 & 14.1 \\
e12 & 1 & 1993-03-16 & 0.77 & 1.1 & 0.08 & 22.5 \\
e12 & 2 & 2012-05-31 & 2.4 & 2.7 & 0.09 & 25.0 \\
e12 & 2 & 2012-06-21 & 1.6 & 1.1 & 0.06 & 27.0 \\
e12 & 2 & 2012-08-07 & 1.3 & 1.3 & 0.2 & 29.0 \\
e12 & 2 & 2012-06-21 & 4 & 6 & 1 & 33.0 \\
e12 & 2 & 2012-08-07 & 0.86 & 1.1 & 0.06 & 36.0 \\
e13 & 2 & 2012-10-16 & 0 & 3 & 2 & 2.5 \\
e13 & 2 & 2012-10-16 & 3.7 & 4.8 & 0.8 & 3.5 \\
e13 & 3 & 2014-04-19 & 0.89 & 1.2 & 0.05 & 4.9 \\
e13 & 1 & 1992-10-25 & 2.3 & 2.6 & 0.1 & 4.9 \\
e13 & 2 & 2012-10-16 & 0.8 & 2.2 & 0.5 & 4.9 \\
e13 & 3 & 2014-04-19 & 3 & 4 & 1 & 5.9 \\
e13 & 2 & 2012-10-16 & 1.1 & 1.3 & 0.05 & 5.9 \\
e13 & 1 & 1992-10-25 & 1.5 & 1.1 & 0.2 & 8.4 \\
e13 & 2 & 2013-03-02 & 1.2 & 1.4 & 0.04 & 12.6 \\
e13 & 2 & 2013-03-02 & 1 & 5 & 1 & 14.1 \\
e13 & 1 & 1993-03-16 & 0.94 & 1.2 & 0.08 & 22.5 \\
e13 & 2 & 2012-05-31 & 2.1 & 2.4 & 0.09 & 25.0 \\
e13 & 2 & 2012-06-21 & 1.5 & 0.96 & 0.06 & 27.0 \\
e13 & 2 & 2012-08-07 & 1.3 & 1.3 & 0.2 & 29.0 \\
e13 & 2 & 2012-06-21 & 1 & 2 & 1 & 33.0 \\
e13 & 2 & 2012-08-07 & 0.86 & 1.1 & 0.06 & 36.0 \\
e14 & 2 & 2012-10-16 & 1 & 3 & 2 & 2.5 \\
e14 & 2 & 2012-10-16 & 1.7 & 2.8 & 0.8 & 3.5 \\
e14 & 1 & 1992-10-25 & 2.1 & 2.4 & 0.1 & 4.9 \\
e14 & 3 & 2014-04-19 & 1.7 & 2 & 0.05 & 4.9 \\
e14 & 2 & 2012-10-16 & 0.6 & 2.1 & 0.5 & 4.9 \\
e14 & 3 & 2014-04-19 & 2 & 5 & 1 & 5.9 \\
e14 & 2 & 2012-10-16 & 1.3 & 1.5 & 0.05 & 5.9 \\
e14 & 1 & 1992-10-25 & 2.2 & 2.3 & 0.2 & 8.4 \\
e14 & 2 & 2013-03-02 & 1.5 & 1.7 & 0.04 & 12.6 \\
e14 & 2 & 2013-03-02 & 0 & 4 & 1 & 14.1 \\
e14 & 1 & 1993-03-16 & 1.5 & 1.7 & 0.08 & 22.5 \\
e14 & 2 & 2012-05-31 & 1.7 & 2 & 0.09 & 25.0 \\
e14 & 2 & 2012-06-21 & 1.4 & 0.96 & 0.06 & 27.0 \\
e14 & 2 & 2012-08-07 & 1.8 & 1.8 & 0.2 & 29.0 \\
e14 & 2 & 2012-06-21 & 3 & 5 & 1 & 33.0 \\
e14 & 2 & 2012-08-07 & 1.6 & 1.9 & 0.06 & 36.0 \\
e15 & 2 & 2012-10-16 & 0 & 0 & 2 & 2.5 \\
e15 & 2 & 2012-10-16 & -2.3 & 1.8 & 0.8 & 3.5 \\
e15 & 3 & 2014-04-19 & 0.33 & 0.51 & 0.05 & 4.9 \\
e15 & 1 & 1992-10-25 & 1.3 & 0.9 & 0.1 & 4.9 \\
e15 & 2 & 2012-10-16 & 0.2 & 1 & 0.5 & 4.9 \\
e15 & 2 & 2012-10-16 & 0.51 & 0.34 & 0.05 & 5.9 \\
e15 & 3 & 2014-04-19 & -1 & 5 & 1 & 5.9 \\
e15 & 1 & 1992-10-25 & 1.9 & 0.9 & 0.2 & 8.4 \\
e15 & 2 & 2013-03-02 & 0.41 & 0.44 & 0.04 & 12.6 \\
e15 & 2 & 2013-03-02 & 2 & 3 & 1 & 14.1 \\
e15 & 1 & 1993-03-16 & 0.82 & 0.49 & 0.08 & 22.5 \\
e15 & 2 & 2012-05-31 & 0.88 & 0.67 & 0.09 & 25.0 \\
e15 & 2 & 2012-06-21 & 1.5 & 0.7 & 0.06 & 27.0 \\
e15 & 2 & 2012-08-07 & 1.9 & 1.8 & 0.2 & 29.0 \\
e15 & 2 & 2012-06-21 & 1 & 1 & 1 & 33.0 \\
e15 & 2 & 2012-08-07 & 0.56 & 0.73 & 0.06 & 36.0 \\
e16 & 2 & 2012-10-16 & 1.1\ee{2} & 1.2\ee{2} & 2 & 2.5 \\
e16 & 2 & 2012-10-16 & 1.9\ee{2} & 1.9\ee{2} & 0.8 & 3.5 \\
e16 & 1 & 1992-10-25 & 98 & 98 & 0.1 & 4.9 \\
e16 & 3 & 2014-04-19 & 13 & 13 & 0.05 & 4.9 \\
e16 & 2 & 2012-10-16 & 1.6\ee{2} & 1.6\ee{2} & 0.5 & 4.9 \\
e16 & 3 & 2014-04-19 & 1.6\ee{2} & 1.7\ee{2} & 1 & 5.9 \\
e16 & 2 & 2012-10-16 & 14 & 15 & 0.05 & 5.9 \\
e16 & 1 & 1992-10-25 & 14 & 14 & 0.2 & 8.4 \\
e16 & 2 & 2013-03-02 & 14 & 15 & 0.04 & 12.6 \\
e16 & 2 & 2013-03-02 & 89 & 92 & 1 & 14.1 \\
e16 & 1 & 1993-03-16 & 11 & 11 & 0.08 & 22.5 \\
e16 & 2 & 2012-05-31 & 1\ee{2} & 1\ee{2} & 0.09 & 25.0 \\
e16 & 2 & 2012-06-21 & 27 & 27 & 0.06 & 27.0 \\
e16 & 2 & 2012-08-07 & 6 & 6.4 & 0.2 & 29.0 \\
e16 & 2 & 2012-06-21 & 1.5\ee{2} & 1.5\ee{2} & 1 & 33.0 \\
e16 & 2 & 2012-08-07 & 8.5 & 9.2 & 0.06 & 36.0 \\
e17 & 2 & 2012-10-16 & 5 & 7 & 2 & 2.5 \\
e17 & 2 & 2012-10-16 & 11 & 12 & 0.8 & 3.5 \\
e17 & 3 & 2014-04-19 & 9.9 & 9.9 & 0.05 & 4.9 \\
e17 & 1 & 1992-10-25 & 22 & 22 & 0.1 & 4.9 \\
e17 & 2 & 2012-10-16 & 12 & 14 & 0.5 & 4.9 \\
e17 & 2 & 2012-10-16 & 7 & 7 & 0.05 & 5.9 \\
e17 & 3 & 2014-04-19 & 7 & 7 & 1 & 5.9 \\
e17 & 1 & 1992-10-25 & 16 & 15 & 0.2 & 8.4 \\
e17 & 2 & 2013-03-02 & 8.2 & 8 & 0.04 & 12.6 \\
e17 & 2 & 2013-03-02 & 4 & 6 & 1 & 14.1 \\
e17 & 1 & 1993-03-16 & 7.9 & 8.2 & 0.08 & 22.5 \\
e17 & 2 & 2012-05-31 & 18 & 18 & 0.09 & 25.0 \\
e17 & 2 & 2012-06-21 & 20 & 19 & 0.06 & 27.0 \\
e17 & 2 & 2012-08-07 & 11 & 11 & 0.2 & 29.0 \\
e17 & 2 & 2012-06-21 & 6 & 8 & 1 & 33.0 \\
e17 & 2 & 2012-08-07 & 11 & 11 & 0.06 & 36.0 \\
e18 & 2 & 2012-10-16 & 0 & 3 & 2 & 2.5 \\
e18 & 2 & 2012-10-16 & 0.2 & 1.6 & 0.8 & 3.5 \\
e18 & 1 & 1992-10-25 & 0.5 & 0.3 & 0.1 & 4.9 \\
e18 & 2 & 2012-10-16 & 0.1 & 0.7 & 0.5 & 4.9 \\
e18 & 3 & 2014-04-19 & 0.12 & 0.33 & 0.05 & 4.9 \\
e18 & 2 & 2012-10-16 & 0.16 & 0.23 & 0.05 & 5.9 \\
e18 & 3 & 2014-04-19 & -0 & 1 & 1 & 5.9 \\
e18 & 1 & 1992-10-25 & 2.1 & 1.2 & 0.2 & 8.4 \\
e18 & 2 & 2013-03-02 & 0.12 & 0.25 & 0.04 & 12.6 \\
e18 & 2 & 2013-03-02 & 0 & 1 & 1 & 14.1 \\
e18 & 1 & 1993-03-16 & -0.27 & 0.26 & 0.08 & 22.5 \\
e18 & 2 & 2012-05-31 & 0.32 & 0.24 & 0.09 & 25.0 \\
e18 & 2 & 2012-06-21 & 1.5 & 0.36 & 0.06 & 27.0 \\
e18 & 2 & 2012-08-07 & 0.2 & 0.9 & 0.2 & 29.0 \\
e18 & 2 & 2012-06-21 & 0 & 2 & 1 & 33.0 \\
e18 & 2 & 2012-08-07 & 0.1 & 0.33 & 0.06 & 36.0 \\
e19 & 2 & 2012-10-16 & 1 & 2 & 2 & 2.5 \\
e19 & 2 & 2012-10-16 & 1 & 1.3 & 0.8 & 3.5 \\
e19 & 2 & 2012-10-16 & 0.5 & 1.2 & 0.5 & 4.9 \\
e19 & 3 & 2014-04-19 & 0.76 & 0.95 & 0.05 & 4.9 \\
e19 & 1 & 1992-10-25 & 1.9 & 1.6 & 0.1 & 4.9 \\
e19 & 3 & 2014-04-19 & 0 & 2 & 1 & 5.9 \\
e19 & 2 & 2012-10-16 & 0.57 & 0.71 & 0.05 & 5.9 \\
e19 & 1 & 1992-10-25 & 2.6 & 1.6 & 0.2 & 8.4 \\
e19 & 2 & 2013-03-02 & 0.6 & 0.81 & 0.04 & 12.6 \\
e19 & 2 & 2013-03-02 & 1 & 1 & 1 & 14.1 \\
e19 & 1 & 1993-03-16 & 0.57 & 0.9 & 0.08 & 22.5 \\
e19 & 2 & 2012-05-31 & 1.5 & 1.4 & 0.09 & 25.0 \\
e19 & 2 & 2012-06-21 & 2 & 1.2 & 0.06 & 27.0 \\
e19 & 2 & 2012-08-07 & 0.8 & 2.3 & 0.2 & 29.0 \\
e19 & 2 & 2012-06-21 & 1 & 2 & 1 & 33.0 \\
e19 & 2 & 2012-08-07 & 1.1 & 1.5 & 0.06 & 36.0 \\
e2 & 2 & 2012-10-16 & 18 & 20 & 2 & 2.5 \\
e2 & 2 & 2012-10-16 & 30 & 30 & 0.8 & 3.5 \\
e2 & 1 & 1992-10-25 & 24 & 25 & 0.1 & 4.9 \\
e2 & 2 & 2012-10-16 & 15 & 17 & 0.5 & 4.9 \\
e2 & 3 & 2014-04-19 & 7.8 & 8 & 0.05 & 4.9 \\
e2 & 2 & 2012-10-16 & 8.1 & 8.9 & 0.05 & 5.9 \\
e2 & 3 & 2014-04-19 & 25 & 26 & 1 & 5.9 \\
e2 & 1 & 1992-10-25 & 4.8 & 5.7 & 0.2 & 8.4 \\
e2 & 2 & 2013-03-02 & 8 & 8.5 & 0.04 & 12.6 \\
e2 & 2 & 2013-03-02 & 15 & 16 & 1 & 14.1 \\
e2 & 1 & 1993-03-16 & 6.5 & 6.8 & 0.08 & 22.5 \\
e2 & 2 & 2012-05-31 & 24 & 25 & 0.09 & 25.0 \\
e2 & 2 & 2012-06-21 & 6.9 & 6.8 & 0.06 & 27.0 \\
e2 & 2 & 2012-08-07 & 0.1 & 4.1 & 0.2 & 29.0 \\
e2 & 2 & 2012-06-21 & 20 & 22 & 1 & 33.0 \\
e2 & 2 & 2012-08-07 & 4.9 & 6.3 & 0.06 & 36.0 \\
e20 & 2 & 2012-10-16 & 1 & 3 & 2 & 2.5 \\
e20 & 2 & 2012-10-16 & 1.9 & 2.6 & 0.8 & 3.5 \\
e20 & 2 & 2012-10-16 & 0.2 & 1.8 & 0.5 & 4.9 \\
e20 & 1 & 1992-10-25 & 2.8 & 2.8 & 0.1 & 4.9 \\
e20 & 3 & 2014-04-19 & 1.9 & 2.1 & 0.05 & 4.9 \\
e20 & 2 & 2012-10-16 & 1.3 & 1.5 & 0.05 & 5.9 \\
e20 & 3 & 2014-04-19 & 3 & 3 & 1 & 5.9 \\
e20 & 1 & 1992-10-25 & 2.4 & 2 & 0.2 & 8.4 \\
e20 & 2 & 2013-03-02 & 1.5 & 1.6 & 0.04 & 12.6 \\
e20 & 2 & 2013-03-02 & 3 & 5 & 1 & 14.1 \\
e20 & 1 & 1993-03-16 & 1.9 & 1.9 & 0.08 & 22.5 \\
e20 & 2 & 2012-05-31 & 2.2 & 2.3 & 0.09 & 25.0 \\
e20 & 2 & 2012-06-21 & 1.8 & 1.1 & 0.06 & 27.0 \\
e20 & 2 & 2012-08-07 & 2.6 & 2.8 & 0.2 & 29.0 \\
e20 & 2 & 2012-06-21 & 2 & 3 & 1 & 33.0 \\
e20 & 2 & 2012-08-07 & 2.2 & 2.5 & 0.06 & 36.0 \\
e21 & 2 & 2012-10-16 & -0 & 4 & 2 & 2.5 \\
e21 & 2 & 2012-10-16 & 0.7 & 1.5 & 0.8 & 3.5 \\
e21 & 1 & 1992-10-25 & 0.5 & 0.7 & 0.1 & 4.9 \\
e21 & 2 & 2012-10-16 & 0.3 & 0.9 & 0.5 & 4.9 \\
e21 & 3 & 2014-04-19 & 0.25 & 0.35 & 0.05 & 4.9 \\
e21 & 2 & 2012-10-16 & 0.28 & 0.35 & 0.05 & 5.9 \\
e21 & 3 & 2014-04-19 & 3 & 3 & 1 & 5.9 \\
e21 & 1 & 1992-10-25 & 0.5 & 0.7 & 0.2 & 8.4 \\
e21 & 2 & 2013-03-02 & 0.29 & 0.38 & 0.04 & 12.6 \\
e21 & 2 & 2013-03-02 & 1 & 2 & 1 & 14.1 \\
e21 & 1 & 1993-03-16 & 0.24 & 0.27 & 0.08 & 22.5 \\
e21 & 2 & 2012-05-31 & 0.55 & 0.62 & 0.09 & 25.0 \\
e21 & 2 & 2012-06-21 & 0.37 & 0.3 & 0.06 & 27.0 \\
e21 & 2 & 2012-08-07 & 0.1 & 0.4 & 0.2 & 29.0 \\
e21 & 2 & 2012-06-21 & 0 & 2 & 1 & 33.0 \\
e21 & 2 & 2012-08-07 & 0.17 & 0.22 & 0.06 & 36.0 \\
e22 & 2 & 2012-10-16 & 22 & 24 & 2 & 2.5 \\
e22 & 2 & 2012-10-16 & 28 & 30 & 0.8 & 3.5 \\
e22 & 3 & 2014-04-19 & 5.8 & 6.1 & 0.05 & 4.9 \\
e22 & 1 & 1992-10-25 & 18 & 17 & 0.1 & 4.9 \\
e22 & 2 & 2012-10-16 & 10 & 10 & 0.5 & 4.9 \\
e22 & 3 & 2014-04-19 & 26 & 28 & 1 & 5.9 \\
e22 & 2 & 2012-10-16 & 4.7 & 4.5 & 0.05 & 5.9 \\
e22 & 1 & 1992-10-25 & 10 & 8.1 & 0.2 & 8.4 \\
e22 & 2 & 2013-03-02 & 3.5 & 4.3 & 0.04 & 12.6 \\
e22 & 2 & 2013-03-02 & 29 & 31 & 1 & 14.1 \\
e22 & 1 & 1993-03-16 & 6.3 & 5.7 & 0.08 & 22.5 \\
e22 & 2 & 2012-05-31 & 18 & 17 & 0.09 & 25.0 \\
e22 & 2 & 2012-06-21 & 14 & 12 & 0.06 & 27.0 \\
e22 & 2 & 2012-08-07 & 5.4 & 7.5 & 0.2 & 29.0 \\
e22 & 2 & 2012-06-21 & 26 & 27 & 1 & 33.0 \\
e22 & 2 & 2012-08-07 & 4.4 & 6.3 & 0.06 & 36.0 \\
e23 & 2 & 2012-10-16 & -2 & 2 & 2 & 2.5 \\
e23 & 2 & 2012-10-16 & 0 & 0.7 & 0.8 & 3.5 \\
e23 & 1 & 1992-10-25 & 0.8 & 0.5 & 0.1 & 4.9 \\
e23 & 2 & 2012-10-16 & 0.4 & 1.1 & 0.5 & 4.9 \\
e23 & 3 & 2014-04-19 & 0.16 & 0.36 & 0.05 & 4.9 \\
e23 & 2 & 2012-10-16 & 0.12 & 0.24 & 0.05 & 5.9 \\
e23 & 3 & 2014-04-19 & -0 & 1 & 1 & 5.9 \\
e23 & 1 & 1992-10-25 & 1.9 & 0.8 & 0.2 & 8.4 \\
e23 & 2 & 2013-03-02 & 0.08 & 0.25 & 0.04 & 12.6 \\
e23 & 2 & 2013-03-02 & 2 & 2 & 1 & 14.1 \\
e23 & 1 & 1993-03-16 & -0.32 & 0.25 & 0.08 & 22.5 \\
e23 & 2 & 2012-05-31 & 0.35 & 0.28 & 0.09 & 25.0 \\
e23 & 2 & 2012-06-21 & 1.4 & 0.36 & 0.06 & 27.0 \\
e23 & 2 & 2012-08-07 & 0.4 & 1.2 & 0.2 & 29.0 \\
e23 & 2 & 2012-06-21 & -0 & 2 & 1 & 33.0 \\
e23 & 2 & 2012-08-07 & 0.39 & 0.61 & 0.06 & 36.0 \\
e3 & 2 & 2012-10-16 & 1 & 2 & 2 & 2.5 \\
e3 & 2 & 2012-10-16 & 1.2 & 2 & 0.8 & 3.5 \\
e3 & 1 & 1992-10-25 & 3.2 & 3.1 & 0.1 & 4.9 \\
e3 & 2 & 2012-10-16 & 0.8 & 1.8 & 0.5 & 4.9 \\
e3 & 3 & 2014-04-19 & 2.5 & 1.9 & 0.05 & 4.9 \\
e3 & 2 & 2012-10-16 & 0.64 & 1.2 & 0.05 & 5.9 \\
e3 & 3 & 2014-04-19 & 1 & 2 & 1 & 5.9 \\
e3 & 1 & 1992-10-25 & 3.6 & 4.3 & 0.2 & 8.4 \\
e3 & 2 & 2013-03-02 & 1.5 & 1.3 & 0.04 & 12.6 \\
e3 & 2 & 2013-03-02 & 1 & 1 & 1 & 14.1 \\
e3 & 1 & 1993-03-16 & 2.5 & 1.9 & 0.08 & 22.5 \\
e3 & 2 & 2012-05-31 & 2.7 & 2.6 & 0.09 & 25.0 \\
e3 & 2 & 2012-06-21 & 4.1 & 3.9 & 0.06 & 27.0 \\
e3 & 2 & 2012-08-07 & 3.6 & 5.2 & 0.2 & 29.0 \\
e3 & 2 & 2012-06-21 & 1 & 2 & 1 & 33.0 \\
e3 & 2 & 2012-08-07 & 4.1 & 4 & 0.06 & 36.0 \\
e4 & 2 & 2012-10-16 & -0 & 2 & 2 & 2.5 \\
e4 & 2 & 2012-10-16 & 0.2 & 0.8 & 0.8 & 3.5 \\
e4 & 2 & 2012-10-16 & 0.5 & 0.8 & 0.5 & 4.9 \\
e4 & 1 & 1992-10-25 & 0.3 & 0.7 & 0.1 & 4.9 \\
e4 & 3 & 2014-04-19 & 0.34 & 0.54 & 0.05 & 4.9 \\
e4 & 2 & 2012-10-16 & 0.38 & 0.39 & 0.05 & 5.9 \\
e4 & 3 & 2014-04-19 & -0 & 1 & 1 & 5.9 \\
e4 & 1 & 1992-10-25 & 0.3 & 0.5 & 0.2 & 8.4 \\
e4 & 2 & 2013-03-02 & 0.33 & 0.47 & 0.04 & 12.6 \\
e4 & 2 & 2013-03-02 & 0 & 1 & 1 & 14.1 \\
e4 & 1 & 1993-03-16 & 0.42 & 0.47 & 0.08 & 22.5 \\
e4 & 2 & 2012-05-31 & 0.34 & 0.59 & 0.09 & 25.0 \\
e4 & 2 & 2012-06-21 & 0.73 & 0.37 & 0.06 & 27.0 \\
e4 & 2 & 2012-08-07 & 1 & 1.4 & 0.2 & 29.0 \\
e4 & 2 & 2012-06-21 & -1 & 1 & 1 & 33.0 \\
e4 & 2 & 2012-08-07 & 0.2 & 0.6 & 0.06 & 36.0 \\
e5 & 2 & 2012-10-16 & 2 & 3 & 2 & 2.5 \\
e5 & 2 & 2012-10-16 & 0.5 & 1 & 0.8 & 3.5 \\
e5 & 3 & 2014-04-19 & 0 & 0.18 & 0.05 & 4.9 \\
e5 & 1 & 1992-10-25 & 0.5 & 0.4 & 0.1 & 4.9 \\
e5 & 2 & 2012-10-16 & -0.6 & 0.8 & 0.5 & 4.9 \\
e5 & 2 & 2012-10-16 & 0.04 & 0.12 & 0.05 & 5.9 \\
e5 & 3 & 2014-04-19 & 1 & 3 & 1 & 5.9 \\
e5 & 1 & 1992-10-25 & 1.3 & 0.8 & 0.2 & 8.4 \\
e5 & 2 & 2013-03-02 & 0.08 & 0.18 & 0.04 & 12.6 \\
e5 & 2 & 2013-03-02 & 1 & 2 & 1 & 14.1 \\
e5 & 1 & 1993-03-16 & 0.09 & 0.19 & 0.08 & 22.5 \\
e5 & 2 & 2012-05-31 & 0.23 & 0.21 & 0.09 & 25.0 \\
e5 & 2 & 2012-06-21 & 0.93 & 0.17 & 0.06 & 27.0 \\
e5 & 2 & 2012-08-07 & -0.1 & 1 & 0.2 & 29.0 \\
e5 & 2 & 2012-06-21 & -0 & 1 & 1 & 33.0 \\
e5 & 2 & 2012-08-07 & 0.25 & 0.29 & 0.06 & 36.0 \\
e6 & 2 & 2012-10-16 & 5 & 9 & 2 & 2.5 \\
e6 & 2 & 2012-10-16 & 10 & 12 & 0.8 & 3.5 \\
e6 & 3 & 2014-04-19 & 7.3 & 7.6 & 0.05 & 4.9 \\
e6 & 1 & 1992-10-25 & 15 & 15 & 0.1 & 4.9 \\
e6 & 2 & 2012-10-16 & 9.7 & 12 & 0.5 & 4.9 \\
e6 & 3 & 2014-04-19 & 8 & 11 & 1 & 5.9 \\
e6 & 2 & 2012-10-16 & 6.6 & 6.9 & 0.05 & 5.9 \\
e6 & 1 & 1992-10-25 & 8 & 7.9 & 0.2 & 8.4 \\
e6 & 2 & 2013-03-02 & 7.1 & 7.3 & 0.04 & 12.6 \\
e6 & 2 & 2013-03-02 & 2 & 4 & 1 & 14.1 \\
e6 & 1 & 1993-03-16 & 6.6 & 6.7 & 0.08 & 22.5 \\
e6 & 2 & 2012-05-31 & 14 & 14 & 0.09 & 25.0 \\
e6 & 2 & 2012-06-21 & 6.9 & 6.2 & 0.06 & 27.0 \\
e6 & 2 & 2012-08-07 & 4.8 & 5.1 & 0.2 & 29.0 \\
e6 & 2 & 2012-06-21 & 5 & 8 & 1 & 33.0 \\
e6 & 2 & 2012-08-07 & 6 & 6.3 & 0.06 & 36.0 \\
e7 & 2 & 2012-10-16 & -0 & 3 & 2 & 2.5 \\
e7 & 2 & 2012-10-16 & 0.3 & 0.9 & 0.8 & 3.5 \\
e7 & 2 & 2012-10-16 & 0.2 & 1 & 0.5 & 4.9 \\
e7 & 3 & 2014-04-19 & 0.38 & 0.42 & 0.05 & 4.9 \\
e7 & 1 & 1992-10-25 & 1 & 0.6 & 0.1 & 4.9 \\
e7 & 2 & 2012-10-16 & 0.42 & 0.39 & 0.05 & 5.9 \\
e7 & 3 & 2014-04-19 & 1 & 1 & 1 & 5.9 \\
e7 & 1 & 1992-10-25 & 2.1 & 1.3 & 0.2 & 8.4 \\
e7 & 2 & 2013-03-02 & 0.25 & 0.41 & 0.04 & 12.6 \\
e7 & 2 & 2013-03-02 & 0 & 2 & 1 & 14.1 \\
e7 & 1 & 1993-03-16 & 0.03 & 0.44 & 0.08 & 22.5 \\
e7 & 2 & 2012-05-31 & 0.6 & 0.52 & 0.09 & 25.0 \\
e7 & 2 & 2012-06-21 & 1.7 & 0.65 & 0.06 & 27.0 \\
e7 & 2 & 2012-08-07 & -0 & 0.9 & 0.2 & 29.0 \\
e7 & 2 & 2012-06-21 & 1 & 2 & 1 & 33.0 \\
e7 & 2 & 2012-08-07 & 0.6 & 0.66 & 0.06 & 36.0 \\
e8n & 2 & 2012-10-16 & - & - & 2 & 2.5 \\
e8n & 2 & 2012-10-16 & - & - & 0.8 & 3.5 \\
e8n & 1 & 1992-10-25 & 0.2 & 0.3 & 0.1 & 4.9 \\
e8n & 2 & 2012-10-16 & 0.4 & 0.8 & 0.5 & 4.9 \\
e8n & 3 & 2014-04-19 & 0.05 & 0.13 & 0.05 & 4.9 \\
e8n & 2 & 2012-10-16 & 0.08 & 0.13 & 0.05 & 5.9 \\
e8n & 3 & 2014-04-19 & - & - & 1 & 5.9 \\
e8n & 1 & 1992-10-25 & 0.4 & 0.4 & 0.2 & 8.4 \\
e8n & 2 & 2013-03-02 & 0.05 & 0.14 & 0.04 & 12.6 \\
e8n & 2 & 2013-03-02 & -0 & 1 & 1 & 14.1 \\
e8n & 1 & 1993-03-16 & 0.13 & 0.12 & 0.08 & 22.5 \\
e8n & 2 & 2012-05-31 & 0.31 & 0.36 & 0.09 & 25.0 \\
e8n & 2 & 2012-06-21 & 0.35 & 0.15 & 0.06 & 27.0 \\
e8n & 2 & 2012-08-07 & 0.3 & 0.4 & 0.2 & 29.0 \\
e8n & 2 & 2012-06-21 & -1 & 1 & 1 & 33.0 \\
e8n & 2 & 2012-08-07 & 0.16 & 0.2 & 0.06 & 36.0 \\
e8s & 2 & 2012-10-16 & -0 & 3 & 2 & 2.5 \\
e8s & 2 & 2012-10-16 & 0 & 1.5 & 0.8 & 3.5 \\
e8s & 2 & 2012-10-16 & -0.4 & 0.7 & 0.5 & 4.9 \\
e8s & 3 & 2014-04-19 & 0.42 & 0.71 & 0.05 & 4.9 \\
e8s & 1 & 1992-10-25 & 0.6 & 0.9 & 0.1 & 4.9 \\
e8s & 2 & 2012-10-16 & 0.4 & 0.53 & 0.05 & 5.9 \\
e8s & 3 & 2014-04-19 & 1 & 2 & 1 & 5.9 \\
e8s & 1 & 1992-10-25 & 0.1 & 0.5 & 0.2 & 8.4 \\
e8s & 2 & 2013-03-02 & 0.45 & 0.66 & 0.04 & 12.6 \\
e8s & 2 & 2013-03-02 & - & - & 1 & 14.1 \\
e8s & 1 & 1993-03-16 & 0.48 & 0.65 & 0.08 & 22.5 \\
e8s & 2 & 2012-05-31 & 0.53 & 0.77 & 0.09 & 25.0 \\
e8s & 2 & 2012-06-21 & 0.16 & 0.13 & 0.06 & 27.0 \\
e8s & 2 & 2012-08-07 & 1.1 & 1.3 & 0.2 & 29.0 \\
e8s & 2 & 2012-06-21 & - & - & 1 & 33.0 \\
e8s & 2 & 2012-08-07 & 0.69 & 0.98 & 0.06 & 36.0 \\
e9 & 2 & 2012-10-16 & 2 & 5 & 2 & 2.5 \\
e9 & 2 & 2012-10-16 & 0.8 & 0.7 & 0.8 & 3.5 \\
e9 & 3 & 2014-04-19 & 0.08 & 0.1 & 0.05 & 4.9 \\
e9 & 2 & 2012-10-16 & -0.2 & 1.3 & 0.5 & 4.9 \\
e9 & 1 & 1992-10-25 & -0 & 0.1 & 0.1 & 4.9 \\
e9 & 2 & 2012-10-16 & 0.14 & 0.1 & 0.05 & 5.9 \\
e9 & 3 & 2014-04-19 & 1 & 2 & 1 & 5.9 \\
e9 & 1 & 1992-10-25 & 0.4 & 0.3 & 0.2 & 8.4 \\
e9 & 2 & 2013-03-02 & 0.05 & 0.07 & 0.04 & 12.6 \\
e9 & 2 & 2013-03-02 & - & - & 1 & 14.1 \\
e9 & 1 & 1993-03-16 & 0.29 & 0.14 & 0.08 & 22.5 \\
e9 & 2 & 2012-05-31 & 0.02 & 0.17 & 0.09 & 25.0 \\
e9 & 2 & 2012-06-21 & 0.37 & 0.18 & 0.06 & 27.0 \\
e9 & 2 & 2012-08-07 & 0.4 & 0.2 & 0.2 & 29.0 \\
e9 & 2 & 2012-06-21 & - & - & 1 & 33.0 \\
e9 & 2 & 2012-08-07 & -0 & 0.11 & 0.06 & 36.0 \\
\hline
\end{tabular}

\end{table*}

%\twocolumn

\clearpage
\section{Point Source Photometry Catalog}
\label{sec:SEDs}
\todo{The full version of the point source photometry catalog is available in
digital form...}

\todo{d3 is not a point source}

\input{preface_aa}

\begin{document}

\title{Toward gas exhaustion in the W51 high-mass protoclusters: Supplemental SED document}
\titlerunning{JVLA SEDs}
\authorrunning{Ginsburg et al}

% B.1

\Figure{figures/pointsource_seds/d3_diffuse_SED.png}
{Point source photometry of d3.  See Figure B.1 for
details}
{fig:d3sed}{1}{7in}
\clearpage

\Figure{figures/pointsource_seds/d4e_SED.png}
{Point source photometry of d4e.  See Figure B.1 for
details}
{fig:d4esed}{1}{7in}
\clearpage

\Figure{figures/pointsource_seds/d4w_SED.png}
{Point source photometry of d4w.  See Figure B.1 for
details}
{fig:d4wsed}{1}{7in}
\clearpage

\Figure{figures/pointsource_seds/d5_SED.png}
{Point source photometry of d5.  See Figure B.1 for
details}
{fig:d5sed}{1}{7in}
\clearpage

\Figure{figures/pointsource_seds/d6_SED.png}
{Point source photometry of d6.  See Figure B.1 for
details}
{fig:d6sed}{1}{7in}
\clearpage

\Figure{figures/pointsource_seds/e10_SED.png}
{Point source photometry of e10.  See Figure B.1 for
details}
{fig:e10sed}{1}{7in}
\clearpage

\Figure{figures/pointsource_seds/e11_SED.png}
{Point source photometry of e11.  See Figure B.1 for
details}
{fig:e11sed}{1}{7in}
\clearpage

\Figure{figures/pointsource_seds/e12?_SED.png}
{Point source photometry of e12?.  See Figure B.1 for
details}
{fig:e12?sed}{1}{7in}
\clearpage

\Figure{figures/pointsource_seds/e13_SED.png}
{Point source photometry of e13.  See Figure B.1 for
details}
{fig:e13sed}{1}{7in}
\clearpage

\Figure{figures/pointsource_seds/e14_SED.png}
{Point source photometry of e14.  See Figure B.1 for
details}
{fig:e14sed}{1}{7in}
\clearpage

\Figure{figures/pointsource_seds/e15_SED.png}
{Point source photometry of e15.  See Figure B.1 for
details}
{fig:e15sed}{1}{7in}
\clearpage

\Figure{figures/pointsource_seds/e16?_SED.png}
{Point source photometry of e16?.  See Figure B.1 for
details}
{fig:e16?sed}{1}{7in}
\clearpage

\Figure{figures/pointsource_seds/e17?_SED.png}
{Point source photometry of e17?.  See Figure B.1 for
details}
{fig:e17?sed}{1}{7in}
\clearpage

\Figure{figures/pointsource_seds/e18?_SED.png}
{Point source photometry of e18?.  See Figure B.1 for
details}
{fig:e18?sed}{1}{7in}
\clearpage

\Figure{figures/pointsource_seds/e19?_SED.png}
{Point source photometry of e19?.  See Figure B.1 for
details}
{fig:e19?sed}{1}{7in}
\clearpage

\Figure{figures/pointsource_seds/e1_SED.png}
{Point source photometry of e1.  See Figure B.1 for
details}
{fig:e1sed}{1}{7in}
\clearpage

\Figure{figures/pointsource_seds/e20?_SED.png}
{Point source photometry of e20?.  See Figure B.1 for
details}
{fig:e20?sed}{1}{7in}
\clearpage

\Figure{figures/pointsource_seds/e21?_SED.png}
{Point source photometry of e21?.  See Figure B.1 for
details}
{fig:e21?sed}{1}{7in}
\clearpage

\Figure{figures/pointsource_seds/e22?_SED.png}
{Point source photometry of e22?.  See Figure B.1 for
details}
{fig:e22?sed}{1}{7in}
\clearpage

\Figure{figures/pointsource_seds/e23?_SED.png}
{Point source photometry of e23?.  See Figure B.1 for
details}
{fig:e23?sed}{1}{7in}
\clearpage

\Figure{figures/pointsource_seds/e2_SED.png}
{Point source photometry of e2.  See Figure B.1 for
details}
{fig:e2sed}{1}{7in}
\clearpage

\Figure{figures/pointsource_seds/e3_SED.png}
{Point source photometry of e3.  See Figure B.1 for
details}
{fig:e3sed}{1}{7in}
\clearpage

\Figure{figures/pointsource_seds/e4_SED.png}
{Point source photometry of e4.  See Figure B.1 for
details}
{fig:e4sed}{1}{7in}
\clearpage

\Figure{figures/pointsource_seds/e5_SED.png}
{Point source photometry of e5.  See Figure B.1 for
details}
{fig:e5sed}{1}{7in}
\clearpage

\Figure{figures/pointsource_seds/e6_SED.png}
{Point source photometry of e6.  See Figure B.1 for
details}
{fig:e6sed}{1}{7in}
\clearpage

\Figure{figures/pointsource_seds/e7_SED.png}
{Point source photometry of e7.  See Figure B.1 for
details}
{fig:e7sed}{1}{7in}
\clearpage

\Figure{figures/pointsource_seds/e8n_SED.png}
{Point source photometry of e8n.  See Figure B.1 for
details}
{fig:e8nsed}{1}{7in}
\clearpage

\Figure{figures/pointsource_seds/e8s_SED.png}
{Point source photometry of e8s.  See Figure B.1 for
details}
{fig:e8ssed}{1}{7in}
\clearpage

\Figure{figures/pointsource_seds/e9_SED.png}
{Point source photometry of e9.  See Figure B.1 for
details}
{fig:e9sed}{1}{7in}
\clearpage

\end{document}


\section{Observation metadata}
The observation metadata is summarized in Table \ref{tab:obs_meta}.
\begin{table*}[htp]
\caption{Observation Metadata}
\begin{tabular}{llllllll}
\label{tab:obs_meta}
Band & Program & TOS & Config & Reference & BW & Date & Notes \\
 &  & $\mathrm{s}$ &  &  & $\mathrm{MHz}$ &  &  \\
\hline
4.9 GHz Epoch 1 & AD0128 & 6830 & B & Mehringer1994a & 25 & 1984-Feb-06 & - \\
4.9 GHz Epoch 1 & AM0367 & 10410 & C & Mehringer1994a & 100 & 1992-Apr-23 & - \\
4.9 GHz Epoch 1 & AM0367 & 6240 & D & Mehringer1994a & 100 & 1992-Aug-04 & - \\
4.9 GHz Epoch 1 & AM0367 & 4800 & D & Mehringer1994a & 100 & 1992-Aug-04 & - \\
8.4 GHz Epoch 1 & AP0242 & 2640 & B & Gaume1993a & 50 & 1992-Oct-24 & Uncertain provenance \\
4.9 GHz Epoch 1 & AM0374 & 6860 & A & Mehringer1994a & 50 & 1992-Oct-25 & - \\
4.9 GHz Epoch 1 & AM0374 & 3350 & A & Mehringer1994a & 50 & 1992-Oct-26 & - \\
22.5 GHz Epoch 1 & AM0374 & 6360 & B & Gaume1993a & 100 & 1993-Mar-16 & - \\
29.0 GHz Epoch 2 & 12A-274 & 5481 & B & Goddi2015a,Goddi2016a & 64 & 2012-Aug-07 & - \\
33.0 GHz Epoch 2 & 12A-274 & 5481 & B & Goddi2015a,Goddi2016a & 64 & 2012-Aug-07 & - \\
36.0 GHz Epoch 2 & 12A-274 & 5481 & B & Goddi2015a,Goddi2016a & 64 & 2012-Aug-07 & - \\
2.5 GHz Epoch 2 & 12B-365 & 778 & A & - & 1024 & 2012-Dec-09 & - \\
3.5 GHz Epoch 2 & 12B-365 & 778 & A & - & 1024 & 2012-Dec-09 & - \\
2.5 GHz Epoch 2 & 12B-365 & 658 & A & - & 1024 & 2012-Dec-24 & - \\
3.5 GHz Epoch 2 & 12B-365 & 658 & A & - & 1024 & 2012-Dec-24 & - \\
29.0 GHz Epoch 2 & 12A-274 & 5585 & B & Goddi2015a,Goddi2016a & 64 & 2012-Jun-21 & - \\
33.0 GHz Epoch 2 & 12A-274 & 5585 & B & Goddi2015a,Goddi2016a & 64 & 2012-Jun-21 & - \\
36.0 GHz Epoch 2 & 12A-274 & 5585 & B & Goddi2015a,Goddi2016a & 64 & 2012-Jun-21 & - \\
25.0 GHz Epoch 2 & 12A-274 & 3829 & B & Goddi2015a,Goddi2016a & 64 & 2012-May-31 & - \\
27.0 GHz Epoch 2 & 12A-274 & 3829 & B & Goddi2015a,Goddi2016a & 64 & 2012-May-31 & - \\
5.9 GHz Epoch 2 & 12B-365 & 777 & A & - & 1024 & 2012-Nov-17 & - \\
4.9 GHz Epoch 2 & 12B-365 & 777 & A & - & 1024 & 2012-Nov-17 & - \\
5.9 GHz Epoch 2 & 12B-365 & 658 & A & - & 1024 & 2012-Nov-24 & - \\
4.9 GHz Epoch 2 & 12B-365 & 658 & A & - & 1024 & 2012-Nov-24 & - \\
5.9 GHz Epoch 2 & 12B-365 & 2154 & A & - & 1024 & 2012-Oct-16 & - \\
4.9 GHz Epoch 2 & 12B-365 & 2154 & A & - & 1024 & 2012-Oct-16 & - \\
5.9 GHz Epoch 2 & 12B-365 & 777 & A & - & 1024 & 2012-Oct-29 & - \\
4.9 GHz Epoch 2 & 12B-365 & 777 & A & - & 1024 & 2012-Oct-29 & - \\
4.9 GHz Epoch 3 & 13A-064 & 4069 & C & - & 1024 & 2013-Jun-04 & - \\
5.9 GHz Epoch 3 & 13A-064 & 4069 & C & - & 1024 & 2013-Jun-04 & - \\
14.1 GHz Epoch 2 & 13A-064 & 4149 & D & - & 1024 & 2013-Mar-02 & - \\
12.6 GHz Epoch 2 & 13A-064 & 4149 & D & - & 1024 & 2013-Mar-02 & - \\
14.1 GHz Epoch 2 & 13A-064 & 14704 & B & - & 1024 & 2013-Oct-01 & - \\
12.6 GHz Epoch 2 & 13A-064 & 14704 & B & - & 1024 & 2013-Oct-01 & - \\
4.9 GHz Epoch 3 & 13A-064 & 15952 & A & - & 1024 & 2014-Apr-19 & - \\
5.9 GHz Epoch 3 & 13A-064 & 15952 & A & - & 1024 & 2014-Apr-19 & - \\
\hline
\end{tabular}

\end{table*}


\end{document}
