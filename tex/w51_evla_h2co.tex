%\documentclass[defaultstyle,11pt]{thesis}
%\documentclass[]{report}
%\documentclass[]{article}
%\usepackage{aastex_hack}
%\usepackage{deluxetable}
%\documentclass[preprint]{aastex}
%\documentclass{aa}
\newcommand\arcdeg{\mbox{$^\circ$}\xspace} 

\pdfminorversion=4


%%%%%%%%%%%%%%%%%%%%%%%%%%%%%%%%%%%%%%%%%%%%%%%%%%%%%%%%%%%%%%%%
%%%%%%%%%%%  see documentation for information about  %%%%%%%%%%
%%%%%%%%%%%  the options (11pt, defaultstyle, etc.)   %%%%%%%%%%
%%%%%%%  http://www.colorado.edu/its/docs/latex/thesis/  %%%%%%%
%%%%%%%%%%%%%%%%%%%%%%%%%%%%%%%%%%%%%%%%%%%%%%%%%%%%%%%%%%%%%%%%
%		\documentclass[typewriterstyle]{thesis}
% 		\documentclass[modernstyle]{thesis}
% 		\documentclass[modernstyle,11pt]{thesis}
%	 	\documentclass[modernstyle,12pt]{thesis}

%%%%%%%%%%%%%%%%%%%%%%%%%%%%%%%%%%%%%%%%%%%%%%%%%%%%%%%%%%%%%%%%
%%%%%%%%%%%    load any packages which are needed    %%%%%%%%%%%
%%%%%%%%%%%%%%%%%%%%%%%%%%%%%%%%%%%%%%%%%%%%%%%%%%%%%%%%%%%%%%%%
\usepackage{latexsym}		% to get LASY symbols
\usepackage{graphicx}		% to insert PostScript figures
%\usepackage{deluxetable}
\usepackage{rotating}		% for sideways tables/figures
\usepackage{natbib}  % Requires natbib.sty, available from http://ads.harvard.edu/pubs/bibtex/astronat/
\usepackage{savesym}
\usepackage{pdflscape}
\usepackage{amssymb}
\usepackage{morefloats}
%\savesymbol{singlespace}
\savesymbol{doublespace}
%\usepackage{wrapfig}
%\usepackage{setspace}
\usepackage{xspace}
\usepackage{color}
\usepackage{multicol}
\usepackage{mdframed}
\usepackage{url}
\usepackage{subfigure}
%\usepackage{emulateapj}
\usepackage{lscape}
\usepackage{grffile}
\usepackage{standalone}
\standalonetrue
\usepackage{import}
\usepackage[utf8]{inputenc}
\usepackage{longtable}
\usepackage{booktabs}
\usepackage[yyyymmdd,hhmmss]{datetime}
\usepackage{fancyhdr}







%\renewcommand\ion[2]{#1$\;${%
%\ifx\@currsize\normalsize\small \else
%\ifx\@currsize\small\footnotesize \else
%\ifx\@currsize\footnotesize\scriptsize \else
%\ifx\@currsize\scriptsize\tiny \else
%\ifx\@currsize\large\normalsize \else
%\ifx\@currsize\Large\large
%\fi\fi\fi\fi\fi\fi
%\rmfamily\@Roman{#2}}\relax}% 

\newcommand{\paa}{Pa\ensuremath{\alpha}}
\newcommand{\brg}{Br\ensuremath{\gamma}}
\newcommand{\msun}{\ensuremath{M_{\odot}}\xspace}			%  Msun
\newcommand{\mdot}{\ensuremath{\dot{M}}\xspace}
\newcommand{\lsun}{\ensuremath{L_{\odot}}}			%  Lsun
\newcommand{\lbol}{\ensuremath{L_{\mathrm{bol}}}}	%  Lbol
\newcommand{\ks}{K\ensuremath{_{\mathrm{s}}}}		%  Ks
\newcommand{\hh}{\ensuremath{\textrm{H}_{2}}\xspace}			%  H2
\newcommand{\dens}{\ensuremath{n(\hh) [\percc]}\xspace}
\newcommand{\formaldehyde}{\ensuremath{\textrm{H}_2\textrm{CO}}\xspace}
\newcommand{\formaldehydeIso}{\ensuremath{\textrm{H}_2~^{13}\textrm{CO}}\xspace}
\newcommand{\methanol}{\ensuremath{\textrm{CH}_3\textrm{OH}}\xspace}
\newcommand{\ortho}{\ensuremath{\textrm{o-H}_2\textrm{CO}}\xspace}
\newcommand{\para}{\ensuremath{\textrm{p-H}_2\textrm{CO}}\xspace}
\newcommand{\oneone}{\ensuremath{1_{1,0}-1_{1,1}}\xspace}
\newcommand{\twotwo}{\ensuremath{2_{1,1}-2_{1,2}}\xspace}
\newcommand{\threethree}{\ensuremath{3_{1,2}-3_{1,3}}\xspace}
\newcommand{\threeohthree}{\ensuremath{3_{0,3}-2_{0,2}}\xspace}
\newcommand{\threetwotwo}{\ensuremath{3_{2,2}-2_{2,1}}\xspace}
\newcommand{\threetwoone}{\ensuremath{3_{2,1}-2_{2,0}}\xspace}
\newcommand{\fourtwotwo}{\ensuremath{4_{2,2}-3_{1,2}}\xspace} % CH3OH 218.4 GHz
\newcommand{\methylcyanide}{\ensuremath{\textrm{CH}_{3}\textrm{CN}}\xspace}
\newcommand{\Rone}{\ensuremath{\para~S_{\nu}(\threetwoone) / S_{\nu}(\threeohthree)}\xspace}
\newcommand{\Rtwo}{\ensuremath{\para~S_{\nu}(\threetwotwo) / S_{\nu}(\threetwoone)}\xspace}
\newcommand{\JKaKc}{\ensuremath{J_{K_a K_c}}}
\newcommand{\water}{H$_{2}$O}		%  H2O
\newcommand{\feii}{\ion{Fe}{2}}		%  FeII
\newcommand{\uchii}{UC\ion{H}{ii}\xspace}
\newcommand{\UCHII}{UC\ion{H}{ii}\xspace}
\newcommand{\hchii}{HC\ion{H}{ii}\xspace}
\newcommand{\HCHII}{HC\ion{H}{ii}\xspace}
\newcommand{\hii}{H~{\sc ii}\xspace}
\newcommand{\hi}{H~{\sc i}\xspace}
\newcommand{\Hii}{H~{\sc ii}\xspace}
\newcommand{\HII}{H~{\sc ii}\xspace}
\newcommand{\Xform}{\ensuremath{X_{\formaldehyde}}}
\newcommand{\kms}{\textrm{km~s}\ensuremath{^{-1}}\xspace}	%  km s-1
\newcommand{\nsample}{456\xspace}
\newcommand{\CFR}{5\xspace} % nMPC / 0.25 / 2 (6 for W51 once, 8 for W51 twice) REFEDIT: With f_observed=0.3, becomes 3/2./0.3 = 5
\newcommand{\permyr}{\ensuremath{\mathrm{Myr}^{-1}}\xspace}
\newcommand{\pers}{\ensuremath{\mathrm{s}^{-1}}\xspace}
\newcommand{\tsuplim}{0.5\xspace} % upper limit on starless timescale
\newcommand{\ncandidates}{18\xspace}
\newcommand{\mindist}{8.7\xspace}
\newcommand{\rcluster}{2.5\xspace}
\newcommand{\ncomplete}{13\xspace}
\newcommand{\middistcut}{13.0\xspace}
\newcommand{\nMPC}{3\xspace} % only count W51 once.  W51, W49, G010
\newcommand{\obsfrac}{30}
\newcommand{\nMPCtot}{10\xspace} % = nmpc / obsfrac
\newcommand{\nMPCtoterr}{6\xspace} % = sqrt(nmpc) / obsfrac
\newcommand{\plaw}{2.1\xspace}
\newcommand{\plawerr}{0.3\xspace}
\newcommand{\mmin}{\ensuremath{10^4~\msun}\xspace}
%\newcommand{\perkmspc}{\textrm{per~km~s}\ensuremath{^{-1}}\textrm{pc}\ensuremath{^{-1}}\xspace}	%  km s-1 pc-1
\newcommand{\kmspc}{\textrm{km~s}\ensuremath{^{-1}}\textrm{pc}\ensuremath{^{-1}}\xspace}	%  km s-1 pc-1
\newcommand{\sqcm}{cm$^{2}$\xspace}		%  cm^2
\newcommand{\percc}{\ensuremath{\textrm{cm}^{-3}}\xspace}
\newcommand{\perpc}{\ensuremath{\textrm{pc}^{-1}}\xspace}
\newcommand{\persc}{\ensuremath{\textrm{cm}^{-2}}\xspace}
\newcommand{\persr}{\ensuremath{\textrm{sr}^{-1}}\xspace}
\newcommand{\peryr}{\ensuremath{\textrm{yr}^{-1}}\xspace}
\newcommand{\perkmspc}{\textrm{km~s}\ensuremath{^{-1}}\textrm{pc}\ensuremath{^{-1}}\xspace}	%  km s-1 pc-1
\newcommand{\perkms}{\textrm{per~km~s}\ensuremath{^{-1}}\xspace}	%  km s-1 
\newcommand{\um}{\ensuremath{\mu \textrm{m}}\xspace}    % micron
\newcommand{\mum}{\um}
\newcommand{\htwo}{\ensuremath{\textrm{H}_2}}    % micron
\newcommand{\Htwo}{\ensuremath{\textrm{H}_2}}    % micron
\newcommand{\HtwoO}{\ensuremath{\textrm{H}_2\textrm{O}}}    % micron
\newcommand{\htwoo}{\ensuremath{\textrm{H}_2\textrm{O}}}    % micron
\newcommand{\ha}{\ensuremath{\textrm{H}\alpha}}
\newcommand{\hb}{\ensuremath{\textrm{H}\beta}}
\newcommand{\so}{SO~\ensuremath{5_6-4_5}\xspace}
\newcommand{\SO}{SO~\ensuremath{1_2-1_1}\xspace}
\newcommand{\ammonia}{NH\ensuremath{_3}\xspace}
\newcommand{\twelveco}{\ensuremath{^{12}\textrm{CO}}\xspace}
\newcommand{\thirteenco}{\ensuremath{^{13}\textrm{CO}}\xspace}
\newcommand{\ceighteeno}{\ensuremath{\textrm{C}^{18}\textrm{O}}\xspace}
\def\ee#1{\ensuremath{\times10^{#1}}}
\newcommand{\degrees}{\ensuremath{^{\circ}}}
% can't have \degree because I'm getting a degree...
\newcommand{\lowirac}{800}
\newcommand{\highirac}{8000}
\newcommand{\lowmips}{600}
\newcommand{\highmips}{5000}
\newcommand{\perbeam}{\ensuremath{\textrm{beam}^{-1}}}
\newcommand{\ds}{\ensuremath{\textrm{d}s}}
\newcommand{\dnu}{\ensuremath{\textrm{d}\nu}}
\newcommand{\dv}{\ensuremath{\textrm{d}v}}
\def\secref#1{Section \ref{#1}}
\def\eqref#1{Equation \ref{#1}}
\def\facility#1{#1}
%\newcommand{\arcmin}{'}

\newcommand{\necluster}{Sh~2-233IR~NE}
\newcommand{\swcluster}{Sh~2-233IR~SW}
\newcommand{\region}{IRAS 05358}

\newcommand{\nwfive}{40}
\newcommand{\nouter}{15}

\newcommand{\vone}{{\rm v}1.0\xspace}
\newcommand{\vtwo}{{\rm v}2.0\xspace}
\newcommand\mjysr{\ensuremath{{\rm MJy~sr}^{-1}}}
\newcommand\jybm{\ensuremath{{\rm Jy~bm}^{-1}}}
\newcommand\nbolocat{8552\xspace}
\newcommand\nbolocatnew{548\xspace}
\newcommand\nbolocatnonew{8004\xspace} % = nbolocat-nbolocatnew
\renewcommand\arcdeg{\mbox{$^\circ$}\xspace} 
\renewcommand\arcmin{\mbox{$^\prime$}\xspace} 
\renewcommand\arcsec{\mbox{$^{\prime\prime}$}\xspace} 

\newcommand{\todo}[1]{\textcolor{red}{#1}}
\newcommand{\okinfinal}[1]{{#1}}
%% only needed if not aastex
%\newcommand{\keywords}[1]{}
%\newcommand{\email}[1]{}
%\newcommand{\affil}[1]{}


%aastex hack
%\newcommand\arcdeg{\mbox{$^\circ$}}%
%\newcommand\arcmin{\mbox{$^\prime$}\xspace}%
%\newcommand\arcsec{\mbox{$^{\prime\prime}$}\xspace}%

%\newcommand\epsscale[1]{\gdef\eps@scaling{#1}}
%
%\newcommand\plotone[1]{%
% \typeout{Plotone included the file #1}
% \centering
% \leavevmode
% \includegraphics[width={\eps@scaling\columnwidth}]{#1}%
%}%
%\newcommand\plottwo[2]{{%
% \typeout{Plottwo included the files #1 #2}
% \centering
% \leavevmode
% \columnwidth=.45\columnwidth
% \includegraphics[width={\eps@scaling\columnwidth}]{#1}%
% \hfil
% \includegraphics[width={\eps@scaling\columnwidth}]{#2}%
%}}%


%\newcommand\farcm{\mbox{$.\mkern-4mu^\prime$}}%
%\let\farcm\farcm
%\newcommand\farcs{\mbox{$.\!\!^{\prime\prime}$}}%
%\let\farcs\farcs
%\newcommand\fp{\mbox{$.\!\!^{\scriptscriptstyle\mathrm p}$}}%
%\newcommand\micron{\mbox{$\mu$m}}%
%\def\farcm{%
% \mbox{.\kern -0.7ex\raisebox{.9ex}{\scriptsize$\prime$}}%
%}%
%\def\farcs{%
% \mbox{%
%  \kern  0.13ex.%
%  \kern -0.95ex\raisebox{.9ex}{\scriptsize$\prime\prime$}%
%  \kern -0.1ex%
% }%
%}%

\def\Figure#1#2#3#4#5{
\begin{figure*}[htp]
\includegraphics[scale=#4,angle=#5]{#1}
\caption{#2}
\label{#3}
\end{figure*}
}


\def\RotFigure#1#2#3#4#5{
\begin{sidewaysfigure*}[htp]
\includegraphics[scale=#4,width=#5]{#1}
\caption{#2}
\label{#3}
\end{sidewaysfigure*}
}

\def\FigureSVG#1#2#3#4{
\begin{figure*}[htp]
    \def\svgwidth{#4}
    \input{#1}
    \caption{#2}
    \label{#3}
\end{figure*}
}

% originally intended to be included in a two-column paper
% this is in includegraphics: ,width=3in
% but, not for thesis
\def\OneColFigure#1#2#3#4#5{
\begin{figure}[htpb]
\epsscale{#4}
\includegraphics[scale=#4,angle=#5]{#1}
\caption{#2}
\label{#3}
\end{figure}
}

\def\SubFigure#1#2#3#4#5{
\begin{figure*}[htp]
\addtocounter{figure}{-1}
\epsscale{#4}
\includegraphics[angle=#5]{#1}
\caption{#2}
\label{#3}
\end{figure*}
}

%\def\FigureTwo#1#2#3#4#5{
%\begin{figure*}[htp]
%\epsscale{#5}
%\plottwo{#1}{#2}
%\caption{#3}
%\label{#4}
%\end{figure*}
%}

\def\FigureTwo#1#2#3#4#5#6{
\begin{figure*}[htp]
\subfigure[]{ \includegraphics[scale=#5,width=#6]{#1} }
\subfigure[]{ \includegraphics[scale=#5,width=#6]{#2} }
\caption{#3}
\label{#4}
\end{figure*}
}

\def\FigureTwoAA#1#2#3#4#5#6{
\begin{figure*}[htp]
\subfigure[]{ \includegraphics[scale=#5,width=#6]{#1} }
\\
\subfigure[]{ \includegraphics[scale=#5,width=#6]{#2} }
\caption{#3}
\label{#4}
\end{figure*}
}

\newenvironment{rotatepage}%
{}{}
   %{\pagebreak[4]\afterpage\global\pdfpageattr\expandafter{\the\pdfpageattr/Rotate 90}}%
   %{\pagebreak[4]\afterpage\global\pdfpageattr\expandafter{\the\pdfpageattr/Rotate 0}}%


\def\RotFigureTwoAA#1#2#3#4#5#6{
\begin{rotatepage}
\begin{sidewaysfigure*}[htp]
\subfigure[]{ \includegraphics[scale=#5,width=#6]{#1} }
\\
\subfigure[]{ \includegraphics[scale=#5,width=#6]{#2} }
\caption{#3}
\label{#4}
\end{sidewaysfigure*}
\end{rotatepage}
}

\def\RotFigureThreeAA#1#2#3#4#5#6#7{
\begin{rotatepage}
\begin{sidewaysfigure*}[htp]
\subfigure[]{ \includegraphics[scale=#6,width=#7]{#1} }
\\
\subfigure[]{ \includegraphics[scale=#6,width=#7]{#2} }
\\
\subfigure[]{ \includegraphics[scale=#6,width=#7]{#3} }
\caption{#4}
\label{#5}
\end{sidewaysfigure*}
\end{rotatepage}
\clearpage
}

\def\FigureThreeAA#1#2#3#4#5#6#7{
\begin{figure*}[htp]
\subfigure[]{ \includegraphics[scale=#6,width=#7]{#1} }
\\
\subfigure[]{ \includegraphics[scale=#6,width=#7]{#2} }
\\
\subfigure[]{ \includegraphics[scale=#6,width=#7]{#3} }
\caption{#4}
\label{#5}
\end{figure*}
}


\def\TallFigureTwo#1#2#3#4#5#6{
    \FigureTwo{#1}{#2}{#3}{#4}{#5}
    }

\def\SubFigureTwo#1#2#3#4#5{
\begin{figure*}[htp]
\addtocounter{figure}{-1}
\epsscale{#5}
\plottwo{#1}{#2}
\caption{#3}
\label{#4}
\end{figure*}
}

\def\FigureFour#1#2#3#4#5#6{
\begin{figure*}[htp]
\subfigure[]{ \includegraphics[width=3in,type=png,ext=.png,read=.png]{#1} }
\subfigure[]{ \includegraphics[width=3in,type=png,ext=.png,read=.png]{#2} }
\subfigure[]{ \includegraphics[width=3in,type=png,ext=.png,read=.png]{#3} }
\subfigure[]{ \includegraphics[width=3in,type=png,ext=.png,read=.png]{#4} }
\caption{#5}
\label{#6}
\end{figure*}
}

\def\FigureFourPDF#1#2#3#4#5#6{
\begin{figure*}[htp]
\subfigure[]{ \includegraphics[width=3in,type=pdf,ext=.pdf,read=.pdf]{#1} }
\subfigure[]{ \includegraphics[width=3in,type=pdf,ext=.pdf,read=.pdf]{#2} }
\subfigure[]{ \includegraphics[width=3in,type=pdf,ext=.pdf,read=.pdf]{#3} }
\subfigure[]{ \includegraphics[width=3in,type=pdf,ext=.pdf,read=.pdf]{#4} }
\caption{#5}
\label{#6}
\end{figure*}
}

\def\FigureThreePDF#1#2#3#4#5{
\begin{figure*}[htp]
\subfigure[]{ \includegraphics[width=3in,type=pdf,ext=.pdf,read=.pdf]{#1} }
\subfigure[]{ \includegraphics[width=3in,type=pdf,ext=.pdf,read=.pdf]{#2} }
\subfigure[]{ \includegraphics[width=3in,type=pdf,ext=.pdf,read=.pdf]{#3} }
\caption{#4}
\label{#5}
\end{figure*}
}

\def\Table#1#2#3#4#5{
%\renewcommand{\thefootnote}{\alph{footnote}}
\begin{table}
\caption{#2}
\label{#3}
    \begin{tabular}{#1}
        \hline\hline
        #4
        \hline
        #5
        \hline
    \end{tabular}
\end{table}
%\renewcommand{\thefootnote}{\arabic{footnote}}
}


%\def\Table#1#2#3#4#5#6{
%%\renewcommand{\thefootnote}{\alph{footnote}}
%\begin{deluxetable}{#1}
%\tablewidth{0pt}
%\tabletypesize{\footnotesize}
%\tablecaption{#2}
%\tablehead{#3}
%\startdata
%\label{#4}
%#5
%\enddata
%\bigskip
%#6
%\end{deluxetable}
%%\renewcommand{\thefootnote}{\arabic{footnote}}
%}

%\def\tablenotetext#1#2{
%\footnotetext[#1]{#2}
%}

\def\LongTable#1#2#3#4#5#6#7#8{
% required to get tablenotemark to work: http://www2.astro.psu.edu/users/stark/research/psuthesis/longtable.html
\renewcommand{\thefootnote}{\alph{footnote}}
\begin{longtable}{#1}
\caption[#2]{#2}
\label{#4} \\

 \\
\hline 
#3 \\
\hline
\endfirsthead

\hline
#3 \\
\hline
\endhead

\hline
\multicolumn{#8}{r}{{Continued on next page}} \\
\hline
\endfoot

\hline 
\endlastfoot
#7 \\

#5
\hline
#6 \\

\end{longtable}
\renewcommand{\thefootnote}{\arabic{footnote}}
}

\def\TallFigureTwo#1#2#3#4#5#6{
\begin{figure*}[htp]
\epsscale{#5}
\subfigure[]{ \includegraphics[width=#6]{#1} }
\subfigure[]{ \includegraphics[width=#6]{#2} }
\caption{#3}
\label{#4}
\end{figure*}
}

		% file containing author's macro definitions


\begin{document}

\title{The proto-O-star population of W51}
\titlerunning{W51 EVLA H2CO}
\authorrunning{Ginsburg et al}
% for future reference, this is probably a better approach:
% https://github.com/dfm/peerless/blob/af483ced97045c213650ed807c430b2f87d2c8c9/document/ms.tex#L104
% assuming it's compatible with A&A
\newcommand{\eso}{$^{1}$}
\newcommand{\nrao}{$^{2}$}
\newcommand{\radboud}{$^{3}$}
\newcommand{\allegro}{$^{4}$}
\newcommand{\morelia}{$^{5}$}
\newcommand{\excellence}{$^{6}$}
\newcommand{\casa}{$^{7}$}
\newcommand{\cfa}{$^{8}$}
\newcommand{\lasp}{$^{9}$}
\newcommand{\jodrell}{$^{11}$}
\newcommand{\sofia}{$^{12}$}
\newcommand{\mpia}{$^{13}$}
\newcommand{\iah}{$^{14}$}


\author{
Adam Ginsburg{\eso},
W.M. Goss{\nrao}, Ciriaco Goddi{\radboud$^{,}$\allegro}, Roberto Galvan-Madrid{\morelia}, Jim Dale\excellence, John Bally{\casa}, Cara
Battersby{\cfa}, Allison Youngblood{\lasp}, Ravi Sankrit{\sofia}, Rowan Smith{\jodrell}, Jeremy Darling\casa, J.~M.~Diederik Kruijssen{\mpia$^{,}$\iah},
Hauyu Baobab Liu{\eso}
        }
%
% Randolf Klein, Jin Koda,  nick scoville,  Allison
% Youngblood, Eric Becklin,  Adam Ginsburg,

%\institute{
%      {$^\casa$}{\it{CASA, University of Colorado, 389-UCB, Boulder, CO 80309}}}
%      {$^\eso$}{\it{European Southern Observatory, Karl-Schwarzschild-Strasse 2, D-85748 Garching bei München, Germany}}}
%      {$^\cfa$}{\it{CfA}}}
%      {$^\mpifr$}{\it{Max Planck Institute for Radio Astronomy, auf dem Hugel, Bonn}}}
%      {$^\nrao$}{\it{National Radio Astronomy Observatory, Socorro}}}
%      {$^\oxford$}{\it{Oxford}}}
%      {$^\chalmers$}{\it{Chalmers}}}
%}
\institute{
    {\eso}{
           \it{
               European Southern Observatory, Karl-Schwarzschild-Stra{\ss}e 2, D-85748 Garching bei München, Germany\\
                      \email{Adam.Ginsburg@eso.org}
               }
           } \\ 
    %{\saudi}{\it{Astron. Dept., King Abdulaziz University, P.O. Box 80203,
    %Jeddah 21589, Saudi Arabia}}\\
    %%{\edmonton}{\it{University of Alberta, Department of Physics, 4-181 CCIS, Edmonton AB T6G 2E1 Canada}} \\ 
    %{\yale}{\it{Department of Astronomy, Yale University, P.O. Box 208101, New Haven, CT 06520-8101 USA}} \\ 
    %%{\puertorico}{\it{Department of Physical Sciences, University of Puerto Rico, P.O. Box 23323, San Juan, PR 00931}}
    %{\mpifr}{\it{Max Planck Institute for Radio Astronomy, auf dem Hugel, Bonn}}
    {\nrao}{\it{National Radio Astronomy Observatory, Socorro, NM 87801 USA}}\\
    %{\oxford}{\it{Oxford}}
    %{\chalmers}{\it{Chalmers}}
    {\radboud}{\it{Department of Astrophysics/IMAPP, Radboud University Nijmegen, PO Box 9010, 6500 GL Nijmegen, the Netherlands}} \\
    {\allegro}{\it{ALLEGRO/Leiden Observatory, Leiden University, PO Box 9513, NL-2300 RA Leiden, the Netherlands}} \\
    {\morelia}{\it{Instituto de Radioastronom{\'i}a y Astrof{\'i}sica, UNAM, A.P. 3-72, Xangari, Morelia, 58089, Mexico}} \\
    {\excellence}{\it{University Observatory/Excellence Cluster `Universe' Scheinerstra{\ss}e 1, 81679 M{\"u}nchen, Germany}} \\
    {\casa}{\it{CASA, University of Colorado, 389-UCB, Boulder, CO 80309}} \\ 
    {\cfa}{\it{Harvard-Smithsonian Center for Astrophysics, 60 Garden
               Street, Cambridge, MA 02138, USA}} \\ 
    {\lasp}{\it{LASP, University of Colorado, 600 UCB, Boulder, CO 80309}}\\
    {\sofia}{\it{SOFIA Science Center, NASA Ames Research Center, M/S 232-12, Moffett Field, CA 94035, USA}}\\
    {\jodrell}{\it{Jodrell Bank Centre for Astrophysics, School of Physics and Astronomy, University of Manchester, Oxford Road, Manchester M13 9PL, UK}} \\
    {\mpia}{\it{Max-Planck Institut f{\"u}r Astrophysik, Karl-Schwarzschild-Stra{\ss}e 1, 85748 Garching, Germany}} \\
    {\iah}{\it{Astronomisches Rechen-Institut, Zentrum f{\"u}r Astronomie der Universit{\"a}t Heidelberg, M{\"o}nchhofstra{\ss}e 12-14, 69120 Heidelberg, Germany}}
    }


% Christian Henkel <chenkel@mpifr-bonn.mpg.de>,
% Jens Kauffmann <jens.kauffmann@gmail.com>
% Thushara Pillai <tpillai.astro@gmail.com>
% Karl M. Menten <kmenten@mpifr-bonn.mpg.de>,
% Katharina Immer <kimmer@mpifr-bonn.mpg.de>,
% John Bally <john.bally@colorado.edu>,
% Betsy Mills <millbets@gmail.com>,
% Jeremy Darling <jdarling@origins.colorado.edu>,
% Denise Riquelme <riquelme@mpifr-bonn.mpg.de>,
% Miguel Angel Requena Torres <mrequena@mpifr-bonn.mpg.de>,
% Cara Battersby <cbattersby@cfa.harvard.edu>,
% Leonardo Testi <ltesti@eso.org>,
% Juergen Ott <jott@nrao.edu>,
% Yiping Ao <ypaobb@gmail.com>,
% Susanne Aalto <susanne.aalto@chalmers.se>,
% Thomas Stanke <tstanke@eso.org>,
% Sarah Kendrew <sarahaskendrew@gmail.com>
% Rolf Guesten <rguesten@mpifr-bonn.mpg.de>
% Arnaud Belloche <belloche@mpifr-bonn.mpg.de>


\date{Date: \today ~~ Time: \currenttime}

\abstract
{}
{}
{}
{}
{
(1) \ortho \twotwo emission traces forming very massive (proto-O) stars
(2) 
}

\maketitle

\todo{To-do items are coded in red.}

\section{Introduction}

The protoclusters within W51 contain many forming massive stars
\citep{Zhang1997a,Keto2008b,Zapata2008a,Zapata2009a,Zapata2010a,Goddi2015a,Shi2010a,Shi2010b}
and a few that have already reached the main sequence and are visible in the
infrared \citep{Barbosa2008a,Figueredo2008a}.

The total luminosity of the W51 protocluster complex has been estimated a few
times using IRAS and KAO to measure the peak of the SED in the far infrared.
The measurements converge on $\sim8.3\ee{6} (D/5.1\mathrm{kpc})$ \lsun
\citep{Harvey1986a,Sievers1991a}.

\section{Observations}
We used the JVLA in multiple bands and configurations.  In project 12B-365, we
observed in A-array in S and C bands with 2 GHz total bandwidth.  In project
13A-064, we observed in C-Band in C (1h) and A (5h) arrays and in Ku-band in D
(1h) and B (5h) arrays.  Our spectral coverage included \ortho \oneone 4.82966 GHz
and \twotwo at 14.488 GHz with 0.3 \kms resolution and H77$\alpha$ and H110$\alpha$
at 1 \kms resolution.

\section{Observational Results}
We report three key observational results: 
\begin{enumerate}
    \item The detection of new continuum sources, which are most likely
        hypercompact \hii regions.
    \item The detection of \formaldehyde \twotwo emission around sources e2,
        e8, and W51 North
    \item Measurements of line-of-sight velocities toward many ultracompact
        \hii regions using H77$\alpha$ and/or \formaldehyde
\end{enumerate}

\subsection{Continuum photometry}
Our data are the most sensitive continuum observations yet performed on the W51
region.  We report new detections of some sources, and concrete identifications
of others that were detected in previous data sets but not reported.

We follow the naming scheme introduced by \citet{Mehringer1994a}.  For the
compact ($r<1\arcsec$) sources within 1 arcminute of W51e2, we use the name
W51e followed by a number.  We identify two new sources, e9 (19:23:43.654
+14:30:26.81) and e10 (19:23:43.956 +14:30:26.95), which were previously
detected but never officially named (as far as we were able to discover).
We additionally split source e8 into a north and south component, plus a more
extended molecular component e8mol.  We also identify a molecular component
between e1, e8, and e10, which we identify as e10mol.

We include in this catalog any pointlike sources (at $\sim0.2-0.4\arcsec$
resolution) with emission in two bands, Ku and C (14 and 5 GHz).  We also
include candidate point sources that may instead be artifacts from the data
reduction process.  In order to identify point sources, we used
uniformly-weighted maps, which remove much of the extended emission and make
pointsource detection simpler, but this process also can result in local peaks
of the diffuse emission appearing to be pointlike.

The point source data is listed in Table \ref{tab:contsrcs}.  The Epoch column
describes the data source: Epoch 1 comes from \citet{Mehringer1994a}, Epoch 2
comes from 12B-365, and Epoch 3 comes from 13A-064.  There are two flux density
columns.  The first shows the peak flux density, which in some
cases is negative because the source sits in a region affected by large angular
scale negative bowls.  The second, `Peak - Background', shows the peak flux
minus the lowest value within that same cutout region.  When these two values
agree, they are reliable, but when they differ significantly they are probably
affected by image reconstruction artifacts.

The multi-epoch data demonstrate that some of these sources are variable.  The
most convincing case for variability is in the (double) source d4.  In the
\citet{Mehringer1994a} data, there is no hint of emission at this location,
with a 5-$\sigma$ upper limit of 1 mJy.  At the same position and frequency in
2014, there is a 1.2 mJy source at the position of d4e.  The case for
variability in other sources is weaker but nonetheless suggestive.

We directly compared the C-band data between the 1994 and 2014 epochs by taking
a difference image.  The 2014 data have higher resolution, so this map
effectively highlights both variable and compact structures.  We found a small
offset between these images using a cross-correlation method
(\url{image-registration.rtfd.org}); the 2014 data were shifted by 0.16, 0.20
\arcsec to the southeast from the 1994 data.

\subsection{\formaldehyde \twotwo emission}
We detect \formaldehyde \twotwo emission around W51e2, W51e8, in a region
between e1, e8, and e10, and in W51 North.  We report the tentative detection
of an extended structure between e2 and e1, though this structure is very weak
and could be an artifact from the image reconstruction process.

The W51e sources all show self-absorption around $v\sim55$ \kms, close in
velocity to the nadir of the W51e2 spectrum.  A diffuse cloud is visible toward
the south at this same velocity, so it seems that the W51e cluster is either
embedded within the $\sim55$ \kms cloud or behind it.  The cloud wraps around
the W51e cluster at lower velocities, so it seems more likely they are embedded
within it.  All of the W51e sources show absorption against the 68 \kms cloud,
so they lie behind that.

The fitted emission line parameters are listed in Table \ref{tab:emission22}.


\begin{table*}[htp]
\caption{\formaldehyde \twotwo emission line parameters}
\begin{tabular}{ccccccccc}
\label{tab:emission22}
Object Name & Amplitude & $E$(Amplitude) & $V_{LSR}$ & $E(V_{LSR})$ & $\sigma_V$ & $E(\sigma_V)$ & $\Omega_{ap}$ & Detection Status \\
 & $\mathrm{mJy}$ &  & $\mathrm{km\,s^{-1}}$ &  &  &  & $\mathrm{sr}$ &  \\
\hline
NorthCore & 0.652 & 0.018 & 58.877 & 0.074 & 2.375 & 0.074 & 2.3\ee{-10} & - \\
e2-e8 bridge & 0.369 & 0.022 & 56.554 & 0.07 & 1.0 & 0.07 & 1.5\ee{-10} & - \\
e2\_a & 0.641 & 0.026 & 57.412 & 0.1 & 2.167 & 0.1 & 5.7\ee{-11} & - \\
e2\_b & 1.085 & 0.035 & 56.083 & 0.097 & 2.605 & 0.097 & 5.7\ee{-11} & - \\
e2\_c & 0.612 & 0.026 & 56.01 & 0.19 & 3.95 & 0.19 & 5.7\ee{-11} & - \\
e8mol & 1.551 & 0.061 & 60.55 & 0.13 & 2.85 & 0.13 & 2\ee{-11} & - \\
e8mol\_ext & 1.045 & 0.024 & 60.16 & 0.089 & 3.439 & 0.089 & 7.2\ee{-11} & weak \\
e10mol & 0.999 & 0.039 & 58.01 & 0.21 & 4.69 & 0.21 & 2\ee{-11} & - \\
e10mol\_ext & 0.813 & 0.018 & 58.507 & 0.094 & 3.593 & 0.094 & 8.5\ee{-11} & weak \\
\hline
\end{tabular}
\end{table*}


\subsection{Line-of-sight velocities}
We have detected H77$\alpha$ emission from many of the ultracompact and
hypercompact \hii regions within W51.  We report their line-of-sight velocities
as measured from gaussian profile fits to their extracted spectra.

The H77$\alpha$ emission line parameters are listed in Table \ref{tab:h77a}, and
the \para \twotwo absorption line parameters are in Table \ref{tab:absorption22}.
%The emission line parameters are only relevant for source e8mol.

\begin{table*}[htp]
\caption{\formaldehyde \twotwo absorption line parameters}
\begin{tabular}{ccccccccc}
\label{tab:absorption22}
Object Name & Amplitude & $E$(Amplitude) & $V_{LSR}$ & $E(V_{LSR})$ & $\sigma_V$ & $E(\sigma_V)$ & $\Omega_{ap}$ & Detection Status \\
 & $\mathrm{mJy}$ &  & $\mathrm{km\,s^{-1}}$ &  &  &  & $\mathrm{sr}$ &  \\
\hline
e1 & -2.922 & 0.034 & 62.531 & 0.078 & 5.81 & 0.078 & 2.9\ee{-11} & ambig \\
e2 & -21.186 & 0.082 & 56.8728 & 0.009 & 2.0231 & 0.009 & 2.5\ee{-11} & - \\
e3 & -2.57 & 0.1 & 64.26 & 0.11 & 2.35 & 0.11 & 9.1\ee{-12} & - \\
e5 & -1.944 & 0.094 & 62.739 & 0.032 & 0.576 & 0.032 & 2.4\ee{-11} & - \\
e6 & -0.598 & 0.015 & 63.771 & 0.061 & 2.149 & 0.061 & 2.4\ee{-10} & - \\
e9 & -0.477 & 0.078 & 55.4 & 0.22 & 1.15 & 0.22 & 2\ee{-11} & - \\
e10 & -0.63 & 0.1 & 66.74 & 0.21 & 1.1 & 0.21 & 1.6\ee{-11} & - \\
\hline
\end{tabular}
\end{table*}


\begin{table*}[htp]
\caption{H$77\alpha$ emission line parameters}
\begin{tabular}{cccccccc}
\label{tab:h77a}
Object Name & Amplitude & $\sigma$(Amplitude) & $V_{LSR}$ & $\sigma(V_{LSR})$ & $dV (\sigma)$ & $\sigma(dV)$ & Detection Status \\
 & $\mathrm{mJy}$ & $\mathrm{mJy}$ & $\mathrm{km\,s^{-1}}$ & $\mathrm{km\,s^{-1}}$ & $\mathrm{km\,s^{-1}}$ & $\mathrm{km\,s^{-1}}$ &  \\
\hline
e1 & 3.0 & 0.2 & 54.87 & 0.85 & 10.9 & 0.85 & - \\
e2 & 0.6 & 0.13 & 56.0 & 3.8 & 15.3 & 3.8 & - \\
e3 & 1.48 & 0.33 & 59.8 & 2.7 & 10.5 & 2.7 & - \\
e4 & 0.56 & 0.39 & 57.2 & 4.4 & 5.4 & 4.4 & - \\
e5 & 0.25 & 0.18 & 52.6 & 4.7 & 5.8 & 4.7 & weak \\
e6 & 0.183 & 0.051 & 68.6 & 4.9 & 15.4 & 4.9 & - \\
e9 & 0.15 & 0.18 & 66.8 & 7.9 & 5.5 & 7.9 & weak \\
e10 & 0.27 & 0.2 & 51.2 & 9.8 & 11.6 & 9.8 & weak \\
\hline
\end{tabular}
\end{table*}


\subsection{Diffuse emission features}
There are a few new notable emission features detected in our data that were
not seen in previous shallower data.

Near source e11, there is a bow-shaped feature (Figure \ref{fig:e11bow}).
There are no known associated sources at shorter wavelengths, though in Spitzer
bands there is some diffuse emission at this location.  The bow-like structure
points away from e11, suggesting that it is the driving source.

\Figure{figures/diffuse/e11_bow.png}
{A bow-shaped feature near the source e11 in the Ku-band continuum image.}
{fig:e11bow}{0.5}{0}

The source d3 is a diffuse HII region with radius $r\sim1.9$ \arcsec.  It is
associated with a Spitzer source and the 2MASS source 2MASX J19233591+1431288.

The diffuse emission associated with W51 Main traces a broad arc that has been
seen in many previous data sets.  The new deeper, higher-resolution data are
dramatically improved.  Where in previous observations, a relatively smooth and
clumpy structure was seen, the new images reveal a network of wispy,
filamentary structures.  Figure \ref{fig:w51mainpeak} shows the region between
the W51e and W51 IRS2 clusters.  While this area contains few clear individual
sources, it accounts for the majority of the luminosity of the W51 Main region
and about half the total luminosity of the W51 Main/IRS2 complex.  

The filamentary structures in the W51 Main peak are unresolved along the short
axis, with aspect ratios $>25$.

\FigureTwo
{figures/diffuse/w51main_peak.png}
{figures/diffuse/w51main_peak_diff.png}
{A C-band image of the W51 Main peak intensity region.  This region accounts
for more than half of the total luminosity of W51.  There are two pointlike
sources in field, e14 at center-left and the cometary e13 toward the lower
left. }
{fig:w51mainpeak}{1}{3.5in}

\section{Analysis}
\subsection{The stellar mass}

We have re-measured the luminosity of the W51 protoclusters using Herschel
Hi-Gal data, fitting an SED from the 70 to 500 \um with a single blackbody
component.  While this is not a very good measurement of the dust temperature -
multiple temperature components are evident \citep{Sievers1991a} - it provides
a good approximation to the total luminosity, which is dominated by a single
warm ($\sim60$ K) component.  The luminosity is about $L\sim2\ee{7}$ \lsun
within a 2 pc radius, which includes both the W51 IRS2 and W51 Main
protoclusters.  It does not include the mid-infrared luminosity, which may
provide an additional $\sim25-50\%$ based on the IRAS measurements. A
luminosity $L=2\pm0.5\ee{7}$ \lsun implies a stellar mass $M_{cl} = 6700 \pm
2300$ \msun, with a corresponding number of O-stars (greater than 20 \msun)
$N_O = 19 \pm 6$.

Of these expected $\sim20$, 4-5 are known O-stars.  \citet{Figuredo2008a} found
4 exposed O-type stars and \citet{Barbosa2008a} found an additional two with
strong infrared excess.  None of these coincide with ultracompact or
hypercompact \hii regions, but all are in the bright and diffuse IRS2 region.
\citet{Mehringer1994a} found an additional 8 ultracompact and hypercompact \hii
regions, all of which appear to be B0 or earlier based on their radio-derived
ionizing photon luminosity.

The other 9-20 O-stars expected to have \emph{already} formed given the observed
luminosity most likely are in fact these most luminous hypercompact \hii regions,
and they simply have their apparent ionizing luminosity suppressed by ongoing
accretion.  Alternatively, all of the so-far detected stars could be unresolved
multiples, though since all of the known stars are in W51 IRS2, this
explanation cannot account for the luminosity of W51 Main.  Another possibility
is that these remaining O-stars are embedded in the ``shell'' structure of the
W51 Main region; this region dominates the luminosity of W51 Main and is bright
enough in near-infrared and radio continuum emission to make detection of point
sources impossible.  However, given the shell-like structure of this loop around
W51 Main, it seems more likely that the shell-illuminating sources are
somewhere near the W51 e1/e2 cluster.

The present stellar mass as inferred from the infrared luminosity is a lower
limit on the final mass of the cluster.  Star formation in W51 is ongoing and
is unlikely to cease until supernovae begin to explode, since even ionized gas
is presently gravitationally bound
\citep{Ginsburg2012a,Bressert2012a,Ginsburg2015a}.

%The O3 or O4 star W51d spectrally typed by \citet{Barbosa2008a} is probably
%illuminating the majority of the ionized gas in the IRS2 region.
% It is
% particularly impressive, though, since the star is shining through a layer of
% molecular gas that extincts the star.

% Can W51d be illuminating the W51 Main HII region?

\subsection{\formaldehyde emission features}
While \formaldehyde \oneone and \twotwo are commonly observed in absorption,
the \oneone line has only been observed in emission in our galaxy as a maser
\citep{Araya2007b}.  The \twotwo line has been observed in emission only  in the
starburst M82, and there very weakly \citep{Mangum2008a}.

We have detected 3 regions of \twotwo emission in the W51 region, all
corresponding to previously detected hot \ammonia cores
\citep{Zhang1997a,Goddi2015a}.  The bright
maser source W51e2 is partially surrounded by a `halo' of \formaldehyde \twotwo
emission to its northeast; the \hchii region itself shows only \twotwo
absorption because the continuum source has a high brightness temperature.  The
\hchii region W51e8 exhibits extended \twotwo emission, including a somewhat
diffuse region stretching between W51e4 and W51e1.  Finally, in W51 IRS 2,
there is extended \twotwo emission stretching between W51d1 and W51d2, adjacent
to the ammonia masers observed by \citet{Zhang1995a} and more recently
\citet{Goddi2015a}, and very close to (but not perfectly correlated with) the
\citet{Zapata2010a} W51 North Disk.

In all cases, the emission is extended and spread smoothly over multiple
velocity channels.  It is therefore not maser emission.

None of these detections have corresponding \oneone emission.  This
nondetection is likely because our brightness sensitivity at C-band is very
poor.  If the
\formaldehyde lines are in local thermal equilibrium and optically thick, the
\oneone brightness temperature should be the sames a the \twotwo,
$T_B\sim500-1000$ K, which is below the noise floor of our C-band observations;
if the dynamic range around the continuum peaks could be improved, we might
hope for weak detections of \oneone emission, but with the current depth, the
lack of a detection is unsurprising.

The physical conditions required to produce these extremely bright emission
regions, with $T_B \gtrsim 500$ K in the FWHM$\approx0.4$\arcsec (2000 au)
beams, can be explained either as thermalized hot gas or radiatively pumped
emission.

The observed high brightness temperatures are similar to the extremely high
temperatures reported by \citet{Zapata2010a}.  The excitation temperature must
be $T\sim900$ K at the \formaldehyde \twotwo $\tau=1$ surface.  However, such a
high temperature is very near the regime in which \formaldehyde would be
collisionally dissociated.

A second possibility is that the \formaldehyde is radiatively pumped, resulting
in a large non-collisionally-driven emission.  Vibrationally excited
\formaldehyde has infrared
transitions around 3.5 and 6 \um \citep{Al-Refaie2015a}, so radiative pumping
requires a very strong radiation field with $T\gtrsim1000$ K.  Such a radiation
field is consistent with the presence of high-mass young stars.  To avoid
radiative dissociation, and especially in W51 North, to avoid creating a large
\hii region, the radiation field must be cooler than $T < 10000$ K,
implying that the proto-O-star \citep[as inferred from its luminosity and its
kinematically-derived mass][]{Zapata2008a,Zapata2009a} is presently at a later
spectral type.

%The required column density for \twotwo
%to become optically thick depends on the velocity gradient and abundance; assuming
%$dV/dR = 1$ \kms \perpc and $X=10^{-9}$, the .

%these are the same...
% Our continuum measurements tighten the limits presented by \citet{Zapata2010a},
% with 5-sigma upper limits of 1 mJy at both C and Ku-band.

The W51e2 and W51e8 cores both exhibit peak brightness temperatures $T>200$ K.
While \citet{Zhang1997a} noted that these cores are ``hot'', the temperatures we
now report are significantly higher.  W51e8 seems an excellent analog to the W51 North
core, at least in terms of its gas density and temperature.  We have detected a
weak extended continuum source in e8, with a peak $S_{15 GHZ} \approx 2$ mJy
and $S_{5 GHz} \approx 1$ mJy, both consistent with the limits on W51North,
where confusion from IRS 2 may prevent a detection.

\subsubsection{Filaments}
There is a `bridge' structure connecting the e8 and e2 cores.  This structure
was seen in \citet{Zhang1997a}, but was poorly resolved and could have been
dismissed as an artifact of the dirty beam.  This bridge is fainter than either
of the cores but clearly detected (Figure \ref{fig:w51bridge22emispec} and
Figure \ref{fig:w51mainemicontours}).  It is also hot, $T\gtrsim300$ K at peak,
and must be extremely dense.

The gas temperatures must either be driven by internal star formation, possibly
a very early stage massive star like W51 North with no HII region yet formed,
or the bridge is heated by the cluster of surrounding massive stars.  The
thermal Jeans mass in this bridge is $M_J = 10.0 \left(\frac{T}{300
K}\right)^{3/2} \left(\frac{n}{10^7 \percc}\right)^{-1/2}$ \msun, implying that
any fragments will be very large or, more likely, fragmentation is prevented.

The bridge could in principle have been created by the ejection of e2 from the
e1 cluster.  However, the maser proper motions of \citep{Saito2014a} show that
e2 and e1 have almost no motion relative to one another.  To achieve their
current separation at a motion of 1 mas \peryr (25 \kms), which is an upper
limit on their relative motion, they would have had to separate 5-10 kyr ago.
The current data do not rule out this scenario, but neither do they strongly
favor it.

\Figure{{figures/spectra/emission/W51Ku_BD_h2co_v30to90_briggs0_contsub.image.fits_W51NorthCore_K}.png}
{Spectrum of the W51 North core in \ortho \twotwo.}
{fig:w51n22emispec}{0.5}{0}

\Figure{{figures/spectra/emission/W51Ku_BD_h2co_v30to90_briggs0_contsub.image.fits_e2-e8 bridge_K}.png}
{Spectrum of the `bridge' connecting cores e2 and e8.}
{fig:w51bridge22emispec}{0.5}{0}

\FigureThreeAA
{figures/contour_movie/e1e2_h2co22_emission_on_cont22_natural_v56.0.png}
{figures/contour_movie/e1e2_h2co22_emission_on_cont22_natural_v57.0.png}
{figures/contour_movie/e1e2_h2co22_emission_on_cont22_natural_v59.5.png}
{Contours of the \formaldehyde \twotwo emission in the W51 Main region at 3
velocities superposed on the 15 GHz continuum map.  (a) shows the peak of the e2
core, where the center of the core is missed because it is in absorption
against the very bright continuum peak, (b) shows the e2-e8 `bridge' feature,
and (c) shows the e8 core}
{fig:w51mainemicontours}{1}{3.5in}

\FigureThreeAA
{figures/spectra/emission/W51Ku_BD_h2co_v30to90_briggs0_contsub.image.fits_W51e2_a_K.png}
{figures/spectra/emission/W51Ku_BD_h2co_v30to90_briggs0_contsub.image.fits_W51e2_b_K.png}
{figures/spectra/emission/W51Ku_BD_h2co_v30to90_briggs0_contsub.image.fits_W51e2_c_K.png}
{Spectra of the \twotwo emission around W51e2.}
{fig:w51e2emispec}{1}{3.5in}

\FigureTwoAA
{figures/spectra/emission/W51Ku_BD_h2co_v30to90_briggs0_contsub.image.fits_W51e1north_K.png}
{figures/spectra/emission/W51Ku_BD_h2co_v30to90_briggs0_contsub.image.fits_W51e8core_K.png}
{Spectra of the \twotwo emission around W51e1.}
{fig:w51e1emispec}{1}{3.5in}

\subsection{Explanation of the hot gas}
The high observed temperatures could be an indication either of genuinely hot
molecular gas or of radiative pumping of the \formaldehyde molecules.  In
either case, an intense source of radiation must be present; the primary
difference is whether the excitation is driven by photons or collisions with
the local dust.

Collisionally excited gas at these high temperatures implies a high density as
well, which in turn implies frequent high-energy collisions between molecules.
It is likely that the high-end tail of these collisions will result in rapid
dissociation of the molecules.  Since \formaldehyde and \ammonia are both
relatively complex, with \ammonia forming slowly in cold gas in the absence of
CO \citep{Caselli?}, re-formation of the molecules would not provide a
sufficient population for our observations.  We therefore favor radiative
(infrared) pumping as the explanation for the high observed brightness.  The gas
must still be warm, $T\gtrsim200$ K \citep{Henkel2013a}, but not quite as hot
as we might naively infer.  \citet{Mangum1993a} determined that infrared
pumping begins to be significant for \formaldehyde when temperatures approach
$T\gtrsim150$ K (see their Appendix C).

Normally \ammonia metastable transitions are cited as good tracers of the gas
temperature because their relative populations can only be set by collisions;
there are no radiative transitions connecting the K-ladders.  However, high
radiation temperatures can easily excite the higher vibrational levels of
\ammonia, which can then decay into different K-ladders \citep[][gives an
overview of the selection pseudo-rules]{Henkel2013a}.  Therefore, temperatures
inferred from \ammonia may be inaccurate in these hot cores.

\subsection{The Lacy jet}
\citet{Lacy2007a} reported the detection of very high velocity ionized gas
in the mid-infrared [Ne II] 12.8\um and S IV 10.5\um lines.  They observe the
gas at a velocity blueshifted about 100 \kms from the IRS2 ionized and molecular
gas velocity.  We have detected the same feature in the H77$\alpha$ RRL.
The RRL shows the same position-velocity structure as the infrared ionized
features.  No redshifted counterflow is detected (Figure \ref{fig:lacyjetslice}).

\FigureTwo
{figures/jetpv/H77a_cutout_1.png}
{figures/jetpv/w51.neii.square.png}
{Position-velocity slices through (a) the H77$\alpha$ cube and (b) the [Ne II]
cube from \citet{Lacy2007a} tracking the approximate path of the Lacy jet.  The
blueshifted lobe is evident at -50 \kms in both cubes, and neither show a
redshifted counterpart.  The H77$\alpha$ data reveal that there is not an
extincted counterpart.}
{fig:lacyjetslice}{1}{3.5in}

\section{The velocity dispersion in the W51e cluster}
We detect H77$\alpha$ toward 6 of the 8 hyper/ultra compact \hii regions in the
W51 Main (W51e) cluster; 4 of these 6 are firm detections and two are weak.  We
also measure a velocity from the \formaldehyde emission toward e8.  The
resulting 1D velocity dispersion is $\sigma=2.0$ \kms if we exclude sources e9
and e10; with e9 and e10 included the dispersion increases to 4.2 \kms, which
suggests that the uncertainty on those two sources renders their velocity
measurements unreliable.  This velocity dispersion is confined to
$r<5.4\arcsec$ or $r<0.13$ pc.  Assuming the stars are virialized (which is
probably not a good assumption), the implied mass is 350-1600 \msun, or an \hh
number density $6\ee{5} < n(\hh) < 2\ee{7}$ \percc.

The e1 subcluster is more compact, with $r=2.9$ \arcsec (0.07 pc), and its
velocity dispersion is $\sigma=2.0$ \kms.  It is more symmetric and a virial
assumption may be more accurate here.  Its implied density
is $n(\hh)\sim2\ee7$ \percc.

Given the mass of W51 Main and its $\sim10\kms$ escape velocity, all of these
(proto)stars are clearly bound to the gas. 

The e5 and e6 sources are close to one another (separation 3.1 \arcsec, or
projected distance 0.08 pc), and both exhibit \formaldehyde absorption at 62-63
\kms.  e6 is detected in H77$\alpha$ at 68 \kms, suggesting that the true
velocity of both e5 and e6 may not be the same as the \formaldehyde absorption
lines, \emph{but} the line profile of the e6 H77$\alpha$ line suggests that the
velocity measurement may be affected by image reconstruction artifacts from
50-63 \kms.

% \section{The distributed population of HCHII regions}
% We detect an additional XX point sources in the field.

\section{Conclusions}

The \formaldehyde \twotwo 14.488 GHz transition is a good tracer of early-stage
very massive star formation.  It will be a powerful tool for studying the earliest
stagest of high-mass star formation throughout the Galaxy with a Square
Kilometer Array if the high-frequency end includes this transition.  Ongoing
JVLA surveys (e.g., KuGARS, PI Thompson,
\url{http://library.nrao.edu/proposals/catalog/10267}) may detect a significant
population of proto-O-stars, though with its limited depth it may only detect
source out to
distance XXX.



\textbf{Acknowledgements}:

\textbf{Code Packages Used}:

\begin{itemize}
    \item aplpy \url{http://aplpy.github.io}
    \item pyradex \url{https://github.com/adamginsburg/pyradex}
    \item myRadex \url{https://github.com/fjdu/myRadex}
    \item pyspeckit \url{http://pyspeckit.bitbucket.org}
    \item aplpy \url{https://aplpy.github.io/}
    \item wcsaxes \url{http://wcsaxes.readthedocs.org}
    \item spectral cube \url{http://spectral-cube.readthedocs.org}
    \item pvextractor \url{http://pvextractor.readthedocs.org/}
\end{itemize}

An excerpt from the point source catalog.  The full version is available in
digital form.
\begin{table*}[htp]
\caption{Continuum Point Sources (excerpt)}
\begin{tabular}{ccccccc}
\label{tab:contsrcs}
Object & Epoch & Obs. Date & Peak $S_{\nu}$ & Peak - Background & $\sigma$ & Frequency \\
 &  &  & $\mathrm{mJy\,beam^{-1}}$ & $\mathrm{mJy\,beam^{-1}}$ & $\mathrm{mJy\,beam^{-1}}$ & $\mathrm{GHz}$ \\
\hline
d3-diffuse & 2 & 2012-10-16 & 0.38 & 0.24 & 0.21 & 2.5 \\
d3-diffuse & 1 & 2012-08-07 & - & - & 1.4 & 22.5 \\
d4e & 2 & 2012-10-16 & 0.79 & 1.1 & 0.051 & 4.9 \\
d4e & 2 & 2014-04-19 & - & - & 0.08 & 33.0 \\
d4w & 2 & 2013-03-02 & 0.65 & 0.95 & 0.11 & 12.6 \\
d5 & 1 & 1992-10-25 & -0.02 & 0.49 & 0.19 & 4.9 \\
d5 & 2 & 2012-06-21 & 1.4 & 2.7 & 0.85 & 27.0 \\
d6 & 3 & 2014-04-19 & 0.38 & 0.39 & 0.05 & 5.9 \\
e1 & 2 & 2012-10-16 & 11 & 11 & 0.21 & 2.5 \\
e1 & 1 & 2012-08-07 & 12 & 14 & 1.4 & 22.5 \\
e10 & 2 & 2012-10-16 & 1.7 & 2 & 0.051 & 4.9 \\
e10 & 2 & 2014-04-19 & 3.2 & 4.7 & 0.08 & 33.0 \\
e11 & 2 & 2013-03-02 & 0.54 & 0.72 & 0.11 & 12.6 \\
e12? & 2 & 2012-10-16 & 0.38 & 0.42 & 0.051 & 4.9 \\
e12? & 2 & 2012-06-21 & 0.98 & 1.5 & 0.85 & 27.0 \\
e13 & 2 & 2012-10-16 & 0.6 & 0.81 & 0.039 & 5.9 \\
e14 & 2 & 2012-10-16 & 0.92 & 1.4 & 0.21 & 2.5 \\
e14 & 1 & 2012-08-07 & 0.7 & 0.7 & 1.4 & 22.5 \\
e15 & 3 & 2012-10-16 & 0.35 & 0.33 & 0.063 & 4.9 \\
e15 & 2 & 2014-04-19 & - & - & 0.08 & 33.0 \\
e16? & 2 & 2013-03-02 & 0.19 & 0.28 & 0.11 & 12.6 \\
e17? & 2 & 2012-10-16 & 0.069 & 0.22 & 0.051 & 4.9 \\
e17? & 2 & 2012-06-21 & -0.76 & 1.9 & 0.85 & 27.0 \\
e18? & 3 & 2014-04-19 & 0.52 & 0.47 & 0.05 & 5.9 \\
e19? & 2 & 2012-10-16 & 1.9 & 1.8 & 0.21 & 2.5 \\
e19? & 1 & 2012-08-07 & 0.2 & 1 & 1.4 & 22.5 \\
e2 & 3 & 2012-10-16 & 11 & 11 & 0.063 & 4.9 \\
e2 & 2 & 2014-04-19 & 1.5\ee{2} & 1.5\ee{2} & 0.08 & 33.0 \\
e20? & 2 & 2013-03-02 & 0.81 & 0.52 & 0.11 & 12.6 \\
e21? & 2 & 2012-10-16 & 0.16 & 0.36 & 0.051 & 4.9 \\
e21? & 2 & 2012-06-21 & 0 & 1.7 & 0.85 & 27.0 \\
e22? & 2 & 2012-10-16 & 0.12 & 0.25 & 0.039 & 5.9 \\
e23? & 2 & 2012-10-16 & -0.05 & 0.96 & 0.21 & 2.5 \\
e23? & 1 & 2012-08-07 & -0.6 & 0.8 & 1.4 & 22.5 \\
e3 & 3 & 2012-10-16 & 6.6 & 6.7 & 0.063 & 4.9 \\
e3 & 2 & 2014-04-19 & 4.5 & 8.1 & 0.08 & 33.0 \\
e4 & 2 & 2013-03-02 & 9.1 & 9.4 & 0.11 & 12.6 \\
e5 & 2 & 2012-10-16 & 7.8 & 8 & 0.051 & 4.9 \\
e5 & 2 & 2012-06-21 & 24 & 25 & 0.85 & 27.0 \\
e6 & 2 & 2012-10-16 & 1.5 & 1.3 & 0.039 & 5.9 \\
e7 & 2 & 2012-10-16 & 0.37 & 0.24 & 0.21 & 2.5 \\
e7 & 1 & 2012-08-07 & -0.2 & 1.3 & 1.4 & 22.5 \\
e8n & 2 & 2012-10-16 & 0.81 & 1.1 & 0.051 & 4.9 \\
e8n & 2 & 2014-04-19 & 4.2 & 5.8 & 0.08 & 33.0 \\
e8s & 2 & 2013-03-02 & 2.3 & 2.6 & 0.11 & 12.6 \\
e9 & 2 & 2012-10-16 & 1.9 & 2.1 & 0.051 & 4.9 \\
e9 & 2 & 2012-06-21 & 2.5 & 2.9 & 0.85 & 27.0 \\
\hline
\end{tabular}
\par
An excerpt from the point source catalog.  For the full catalog, see Table \ref{tbl:contsrcs_full}
\end{table*}


\ifstandalone
\bibliographystyle{apj_w_etal}  % or "siam", or "alpha", or "abbrv"
%\bibliography{thesis}      % bib database file refs.bib
\bibliography{bibdesk}      % bib database file refs.bib
\fi

\end{document}
