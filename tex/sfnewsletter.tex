%%%%%%%%%%%%%%%%%%%%%%%%%%%%%%%%%%%%%%%%%%%%%%%%%%%%%%%%%%%%%%%%%%%%%%%%%%%%%
%%%                                                                       %%%
%%%            LaTeX MACRO FOR THE STAR FORMATION NEWSLETTER              %%%
%%%                                                                       %%%
%%%    Please use for abstracts of papers which have been ACCEPTED in     %%%
%%%    REFEREED JOURNALS (do not send abstracts of reviews for books      %%%
%%%    or conference notes).  Merely fill in the brackets below and       %%%
%%%    mail to reipurth@ifa.hawaii.edu.  If you have problems, let me     %%%
%%%    know in an accompanying note and I will fix them.                  %%%
%%%                                                                       %%%
%%%%%%%%%%%%%%%%%%%%%%%%%%%%%%%%%%%%%%%%%%%%%%%%%%%%%%%%%%%%%%%%%%%%%%%%%%%%%

\documentclass[]{article}
\textwidth 18cm
\textheight 23cm
\oddsidemargin -1cm
\topmargin 0cm
\parskip 0.15cm
\parindent 0pt
\small
\newcommand{\eso}{$^{1}$}
\newcommand{\nrao}{$^{2}$}
\newcommand{\radboud}{$^{3}$}
\newcommand{\allegro}{$^{4}$}
\newcommand{\morelia}{$^{5}$}
\newcommand{\excellence}{$^{6}$}
\newcommand{\casa}{$^{7}$}
\newcommand{\cfa}{$^{8}$}
\newcommand{\lasp}{$^{9}$}
\newcommand{\jodrell}{$^{11}$}
\newcommand{\sofia}{$^{12}$}
\newcommand{\mpia}{$^{13}$}
\newcommand{\iah}{$^{14}$}
\newcommand{\ortho}{\ensuremath{\textrm{o-H}_2\textrm{CO}}\xspace}
\newcommand{\twotwo}{\ensuremath{2_{1,1}-2_{1,2}}\xspace}

\begin{document}

%% Between these brackets you write the title of your paper:
{\large\bf{Toward gas exhaustion in the W51 high-mass protoclusters}}

%% Here comes the author(s) of the paper, please indicate within $^...$
%% the number which corresponds to the institute of each author.
{\bf{ 

Adam Ginsburg{\eso},
W.~M. Goss{\nrao}, Ciriaco Goddi{\radboud$^{,}$\allegro}, Roberto Galv{\'a}n-Madrid{\morelia}, James E. Dale\excellence, John Bally{\casa}, Cara D.
Battersby{\cfa}, Allison Youngblood{\lasp}, Ravi Sankrit{\sofia}, Rowan Smith{\jodrell}, Jeremy Darling\casa, J.~M.~Diederik Kruijssen{\mpia$^{,}$\iah},
Hauyu Baobab Liu{\eso}
}}

%%% Here you write your institute name(s) and address(es),
%%% the number in $^..$ indicates your author number, for example:
%$^1$ {European Southern Observatory, Casilla 19001, Santiago 19, Chile} \\
%$^2$ {Cerro Tololo Inter-American Observatory, National Optical Astronomy
%     Observatories, Casilla 603, La Serena, Chile} \\
%$^3$ {Las Campanas Observatory, Carnegie Inst. of Washington, Casilla
%     601, La Serena, Chile}

{\eso}{
       \it{
           European Southern Observatory, Karl-Schwarzschild-Stra{\ss}e 2, D-85748 Garching bei München, Germany\\
                  \email{Adam.Ginsburg@eso.org}
           }
       } \\ 
%{\saudi}{\it{Astron. Dept., King Abdulaziz University, P.O. Box 80203,
%Jeddah 21589, Saudi Arabia}}\\
%%{\edmonton}{\it{University of Alberta, Department of Physics, 4-181 CCIS, Edmonton AB T6G 2E1 Canada}} \\ 
%{\yale}{\it{Department of Astronomy, Yale University, P.O. Box 208101, New Haven, CT 06520-8101 USA}} \\ 
%%{\puertorico}{\it{Department of Physical Sciences, University of Puerto Rico, P.O. Box 23323, San Juan, PR 00931}}
%{\mpifr}{\it{Max Planck Institute for Radio Astronomy, auf dem Hugel, Bonn}}
{\nrao}{\it{National Radio Astronomy Observatory, Socorro, NM 87801 USA}}\\
%{\oxford}{\it{Oxford}}
%{\chalmers}{\it{Chalmers}}
{\radboud}{\it{Department of Astrophysics/IMAPP, Radboud University Nijmegen, PO Box 9010, 6500 GL Nijmegen, the Netherlands}} \\
{\allegro}{\it{ALLEGRO/Leiden Observatory, Leiden University, PO Box 9513, NL-2300 RA Leiden, the Netherlands}} \\
{\morelia}{\it{Instituto de Radioastronom{\'i}a y Astrof{\'i}sica, UNAM, A.P. 3-72, Xangari, Morelia, 58089, Mexico}} \\
{\excellence}{\it{University Observatory/Excellence Cluster `Universe' Scheinerstra{\ss}e 1, 81679 M{\"u}nchen, Germany}} \\
{\casa}{\it{CASA, University of Colorado, 389-UCB, Boulder, CO 80309}} \\ 
{\cfa}{\it{Harvard-Smithsonian Center for Astrophysics, 60 Garden
           Street, Cambridge, MA 02138, USA}} \\ 
{\lasp}{\it{LASP, University of Colorado, 600 UCB, Boulder, CO 80309}}\\
{\sofia}{\it{SOFIA Science Center, NASA Ames Research Center, M/S 232-12, Moffett Field, CA 94035, USA}}\\
{\jodrell}{\it{Jodrell Bank Centre for Astrophysics, School of Physics and Astronomy, University of Manchester, Oxford Road, Manchester M13 9PL, UK}} \\
{\mpia}{\it{Max-Planck Institut f{\"u}r Astrophysik, Karl-Schwarzschild-Stra{\ss}e 1, 85748 Garching, Germany}} \\
{\iah}{\it{Astronomisches Rechen-Institut, Zentrum f{\"u}r Astronomie der Universit{\"a}t Heidelberg, M{\"o}nchhofstra{\ss}e 12-14, 69120 Heidelberg, Germany}}

%% Here you may write the e-mail address of one or more of the authors
%% who will act as contact person for preprint requests etc, for example:

{E-mail contact: adam.g.ginsburg@gmail.com}


  %% IF YOU USE ANY PERSONAL LATEX COMMANDS IN YOUR ABSTRACT,
  %% PLEASE INCLUDE THEIR DEFINITIONS HERE!
  %% AND PLEASE INCLUDE ONLY THOSE YOU NEED FOR THE ABSTRACT.


%% Within the following brackets you place your text:

{
We present new JVLA observations of the high-mass cluster-forming region W51A
from 2 to 16 GHz with resolution
$\theta_{fwhm}\approx0.3-0.5\arcsec$.  The data reveal a wealth of
observational results:
(1) Currently-forming, very massive (proto-O) stars are traced by \ortho\ 
\twotwo\  emission, suggesting that this line can be used efficiently as a
massive protostar tracer.
(2) There is a spatially distributed population of $\lesssim$mJy continuum sources,
including hypercompact H{\sc II} regions and candidate colliding wind binaries,
in and around the W51 proto-clusters.  
(3) There are two clearly detected protoclusters, W51e and W51 IRS2, that are
gas-rich but may have most of their mass in stars within their inner $\lesssim0.05$
pc.  The
majority of the bolometric luminosity in W51 most likely comes from a third
population of OB stars between these clusters.
The presence of a substantial population of exposed O-stars coincident with
a population of still-forming massive stars, along with a direct measurement
of the low mass loss rate via ionized gas outflow from W51 IRS2, together imply
that feedback is ineffective at halting star formation in massive
protoclusters.  Instead, feedback may shut off the large-scale accretion of
diffuse gas onto the W51 protoclusters, implying that they are evolving towards
a state of gas exhaustion rather than gas expulsion. Recent theoretical models
predict gas exhaustion to be a necessary step in the formation of
gravitationally bound stellar clusters, and our results provide an
observational validation of this process.
}


% Here you write which journal accepted your paper, for example:

{ Accepted by A\&A }

%% If preprints are available on the WWW you can give the web
%% direction here.



\end{document}
